\section{六元杂环化合物}

\subsection{结构和性质}

\begin{center}
    \chemfig{*6(-N=-=-=)}
\end{center}

$\ce{N}$ 碱性、亲和性、氧化反应

$\beta$ 位可以发生亲电取代,$\alpha$位可发生亲核取代。

\subsection{物理性质}

\subsubsection{沸点}

\begin{center}
    \chemfig{*6(-=-=-=)} < \chemfig{*6(-N=-=-=)}
\end{center}

N的存在一定程度上破坏了对称结构,会导致分子产生极性。

\subsubsection{溶解性}

吡啶有一定的极性,可以溶解大部分有机物质,是良好的溶剂。

\subsection{化学性质}

\subsubsection{碱性}

脂肪胺 > 吡啶 > 苯胺 > 吡咯

\begin{center}
    \chemfig{*6(-N(-H)-----)} > \chemfig{*6(-N=-=-=)} > \chemfig{*6(-=-=(-NH_2)-=)} > \chemfig{[:-17]*5(-N(-H)-=-=)}
\end{center}

\begin{center}
    \small
    \schemestart
    \chemfig{*6(-N=-=-=)} \arrow{0}[,0] \+ $\ce{HCl}$ \arrow{->} \chemfig{*6(-\chemabove{N}{\oplus}(-H|\chembelow{Cl}{\ominus})=-=-=)}
    \schemestop
\end{center}

\subsubsection{亲核性}

可以作为亲核试剂和 $\ce{CH3I}$ 反应。


\begin{center}
    \small
    \schemestart
    \chemfig{*6(-N=-=-=)} \arrow{0}[,0] \+ $\ce{CH3I}$ \arrow{->} \chemfig{*6(-\chemabove{N}{\oplus}(-CH_3|\chembelow{I}{\ominus})=-=-=)}
    \schemestop
\end{center}

\subsubsection{亲电取代($\beta$位)}

\paragraph{卤素} 从上到下越来越难
\begin{center}
    \small
    \schemestart
    \chemfig{*6(-N=-=-=)}\arrow{0}[,0] \+ $\ce{Cl2}$ \arrow{->[催化]}[,1.3]\chemfig{*6(-N=-(-Cl)=-=)} + $\ce{HCl}$
    \schemestop
\end{center}

\begin{center}
    \small
    \schemestart
    \chemfig{*6(-N=-=-=)}\arrow{0}[,0] \+ $\ce{Br2}$ \arrow{->[催化]}[,1.3]\chemfig{*6(-N=-(-Br)=-=)} + $\ce{HCl}$
    \schemestop
\end{center}
\paragraph{硝化} 这里用硝酸钾是为了防止 $\ce{HNO3}$ 挥发
\begin{center}
    \small
    \schemestart
    \chemfig{*6(-N=-=-=)}\arrow{0}[,0] \+ $\ce{KNO3}$ \arrow{->[催化]}[,1.3]\chemfig{*6(-N=-(-NO_2)=-=)}
    \schemestop
\end{center}
\paragraph{磺化}
\begin{center}
    \small
    \schemestart
    \chemfig{*6(-N=-=-=)}\arrow{0}[,0] \+ $\ce{H2SO4}$ \arrow{->[催化]}[,1.3]\chemfig{*6(-N=-(-SO_3H)=-=)}
    \schemestop
\end{center}

\paragraph{FC烷基化反应}不能形成对应的共振结构,吡啶一般不发生这种反应。

\paragraph{FC烷基化反应} 不能形成对应的共振结构,吡啶一般不发生这种反应。

\subsubsection{亲核试剂($\alpha$位)}

\begin{center}
    \small
    \schemestart
    \chemfig{*6(-N=-=-=)}\arrow{0}[,0] \+ $\ce{NaNH2}$ \arrow{->}[,1.3]\chemfig{*6(-N=(-NH2)-=-=)}
    \schemestop
\end{center}


\begin{center}
    \small
    \schemestart
    \chemfig{*6(-N=(-Cl)-=-=)}\arrow{0}[,0] \+ $\ce{NaNH2}$ \arrow{->[$\ce{NH3}$][$\Delta$]}[,1.3]\chemfig{*6(-N=(-NH2)-=-=)}
    \schemestop
\end{center}

\subsubsection{氧化反应}

\paragraph{加氧} 改变定位作用,活化吡啶环

% N成为正离子,下面接一个O负离子。

\paragraph{硝酸氧化}

与硝酸反应,$\beta$位的$\ce{-CH2R}$被氧化为$\ce{-COOH}$

\subsubsection{还原反应}


催化加氢


\begin{center}
    \small
    \schemestart
    \chemfig{*6(-N=-=-=)} \arrow{->[$\ce{NH3}$][$\ce{H2, Pt}$]}[,1.3]\chemfig{*6(-\chembelow{N}{H}-----)}
    \schemestop
\end{center}

\paragraph{$\alpha - \ce{H}$的亲核性}

可以和醛酮发生类似羟醛缩合的反应。


