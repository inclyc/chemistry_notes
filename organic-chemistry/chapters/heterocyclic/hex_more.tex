\section{嗪类}

\begin{center}
    \chemname{\chemfig{*6(=N-N=-=-=)}}{哒嗪}
    \chemname{\chemfig{*6(=N-=-=N-=)}}{嘧啶}
\end{center}


\subsection{物理性质}

易溶于水

\subsection{化学性质}

\subsubsection{亲电取代(难)}

如果环上连有多个供电基,如氨基、羟基、甲基。才能发生亲电取代

\begin{center}
    \small
    \chemname{\chemfig{*6((-OH)=N-=-(-OH)=N-=)}}{尿嘧啶}
    \chemname{\chemfig{*6((-OH)=N-=(-CH_3)-(-OH)=N-=)}}{胸腺嘧啶}
    \chemname{\chemfig{*6((-OH)=N-=-(-NH_2)=N-=)}}{胞嘧啶}
\end{center}

这些物质的 \chemfig{=[:30](-[:90]OH)-[:-30]} 都有互变异构 \chemfig{-[:30](=[:90]O)-[:-30]}。


\subsubsection{亲核取代}

通常比\chemfig{*6(-N=-=-=-)}简单。

\begin{center}
    \small
    \schemestart
    \chemfig{*6(=N-=-=N-=)} \arrow{->[$\ce{NaNH2}$]}[,1.3]\chemfig{*6((-NH_2)=N-=-=N-=)}
    \schemestop
\end{center}

\subsubsection{氧化反应}

\begin{center}
    \small
    \schemestart
    \chemfig{*6(=N-=-=N-=)} \arrow{->[$\ce{H2O2}$][$\ce{CH3COOH}$]}[,1.5] \chemfig{*6(=\chemabove{N}{\oplus}(-\chembelow{O}{\ominus})-=-=N-=)} \arrow{->[$\ce{H2SO4}$][$\ce{HNO3}$]}[,1.4]\chemfig{*6(=\chemabove{N}{\oplus}(-\chembelow{O}{\ominus})-=(-NO_2)-=N-=)} \arrow{->[$\ce{PCl3}$]}\chemfig{*6(=N-=(-NO_2)-=N-=)}
    \schemestop
\end{center}

% TODO: hex那里还没写