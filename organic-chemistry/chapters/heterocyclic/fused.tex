\section{稠杂环}

\begin{center}
    \chemname{\chemfig{*6(-=*5(-N(-[,0.5]H)-=-)-=-=)}}{吲哚}
\end{center}

\subsection{结构}

整个分子成平面结构,具有$\Pi_9^{10}$型大$\Pi$键。吲哚环的芳香性使得它容易发生亲电取代。

\subsection{物理性质}

吲哚难溶于水,常温下是固体(晶体)。

\subsection{化学性质}


\paragraph{亲电取代}

一般在比吡咯环的$\beta$位发生取代反应。亲电取代的难易程度:

\begin{center}
    \small
    \schemestart
    \chemfig{[:-17]*5(-N(-H)-=-=)} \subscheme{\arrow{0}[,0.5]}  > \subscheme{\arrow{0}[,0.5]} \chemfig{*6(-=*5(-N(-[,0.5]H)-@{b}=@{a}-)-=-=)} \subscheme{\arrow{0}[,0.5] }> \subscheme{\arrow{0}[,0.5] }\chemfig{*6(-=-=-=)}
    \schemestop
    \chemmove{
        \draw (a)+(0.5,0.5) -- (a);
        \node[above = 3pt of a, red] (dummy1) {$\beta$};
        \node[right = 2pt of b, red] (b) {$\alpha$};
    }
\end{center}

\begin{center}
    \small
    \schemestart
    \chemfig{*6(-=*5(-N(-[,0.5]H)-=-)-=-=)} \subscheme{\arrow{0}[,0.1]} \+ $\ce{Br2}$ \arrow{->[\scriptsize\chemfig{[:30]*6(--O---O-)}]}[,1.5]\chemfig{*6(-=*5(-N(-[,0.5]H)-=(-Br)-)-=-=)}
    \schemestop
\end{center}

\begin{center}
    \small
    \schemestart
    \chemfig{*6(-=*5(-N(-[,0.5]H)-=-)-=-=)} \subscheme{\arrow{0}[,0.1]} \arrow{->[$0^\circ \mathrm{C}$][$\ce{CH3COONO2}$]}[,1.9]\chemfig{*6(-=*5(-N(-[,0.5]H)-=(-NO_2)-)-=-=)}
    \schemestop
\end{center}

\begin{center}
    \small
    \schemestart
    \chemfig{*6(-=*5(-N(-[,0.5]H)-=-)-=-=)} \subscheme{\arrow{0}[,0.1]} \arrow{->[\scriptsize\chemname[-7.7ex]{\chemfig{[:90]*6(-\chemabove{N}{\oplus}(-\chembelow{S}{\ominus}O_3)=-=-=)}}{三氧化硫吡啶}]}[,1.9]\chemfig{*6(-=*5(-N(-[,0.5]H)-=(-NO_2)-)-=-=)}
    \schemestop
\end{center}
