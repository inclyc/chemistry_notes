\chapter{含氮有机化合物}

\section{硝基化合物}

\subsection{硝基化合物的物理性质}

一般以固体形式存在,难溶于水


\subsection{硝基化合物的化学性质}

\subsubsection{脂肪族}

$\ce{CH3NO2}$ 酸性、亲核性、还原反应

$\ce{O2N-CH2-}$可以作为很好的亲核试剂$\ce{Nu-}$,包括与羟基化合物的缩合,有点类似羟醛缩合反应。

硝基会减弱碱性、降低亲电取代的活性,增加亲核取代的活性(例如氯苯转变为苯酚)

\subsubsection{芳香族硝基化合物}


\begin{center}
    \small
    \schemestart
    \chemfig{*6(-=-(-NO_2)=-=)}
    \schemestop
\end{center}


\section{胺}

\subsection{概述}

\subsubsection{分类}


可以按照与和胺相连的烃基分类为脂肪族和芳香族的胺化合物。根据\textbf{氨基上连有的烃基的数目},可以将胺分为伯胺,仲胺,叔胺。注意这个分类的方法和普通的按照碳原子分了的伯仲叔不同。


\subsection{物理性质}

\subsubsection{熔沸点}
伯、仲胺都有极化的氢。N的电负性没有 $\ce{-OH}$ 那么大,他们可以形成分子间的氢键但是比羧酸形成的分子间氢键要弱得多。伯、仲胺的熔沸点通常比一般的烷烃高,比醇、羧酸低。甲胺、乙胺为气体、其他胺类为液体。

\subsubsection{溶解度}

低级胺可溶,高级胺难溶。


\subsection{化学性质}

\subsubsection*{成键与杂化}

脂肪族 $\ce{-NH2}$ 一般采用sp3杂化。 $\ce{Ph-NH2}$ 一般采用$sp^2$杂化,然而,这里的$sp^2$ 也有一定的$sp^3$特征。


\subsubsection*{手性氨}


当氮原子上连接三个不同的基团时,为手性氮。然而,手性氮的两个手性对映体的势垒差很低,在常温下可以迅速相互转化,目前技术上还不能分离出手性物质。

\subsubsection{碱性}

\begin{center}
    $\ce{RNH2 + H+ -> RN^+H3}$ 
\end{center}

\paragraph{电子效应} 氨基可以和一个氢离子结合形成铵盐。氮原子上连有给电子基团时碱性增强,连有吸电子基团时碱性减弱。因此,在气态中

\begin{center}
    $\ce{(CH3)3N}$ > $\ce{(CH4)_2NH}$ > $\ce{CH3NH2}$ > $\ce{NH3}$   
\end{center}

然而,胺在溶剂中需要考虑溶剂化效应、空间位阻效应等其他影响因素。表\ref{tab:effect}中列出了这些影响。

\begin{table}[h]
    \centering
    \begin{tabular}{ll}
        \toprule
        \textbf{影响因素} & \textbf{胺的碱性影响} \\ 
        \midrule
        电子效应 & 叔 > 仲 > 伯 \\ 
        溶剂化效应 & 伯 > 仲 > 叔 \\ 
        空间位阻效应 & 伯 > 仲 > 叔 \\
        \bottomrule
    \end{tabular}
    \caption{不同的影响因素对伯、仲、叔胺的碱性影响}
    \label{tab:effect}
\end{table}


\subsubsection{亲核性}


\paragraph{烃基化反应}

\begin{center}
    \small
    \schemestart
    \chemfig{RNH_2} + $\ce{R-X}$ \arrow{->} \chemfig{R-NH(-[:90]R')}
    \schemestop
\end{center}


\paragraph{酰基化反应}

\begin{center}
    \small
    \schemestart
    \chemfig{RNH_2} + \chemfig{R-C(=[:90]O)-Cl} \arrow{->} \chemfig{R-C(=[:90]O)-NH-R'}
    \schemestop
\end{center}

叔胺可以进一步在碱的作用下形成季铵盐。

$\ce{-OH}$ 保护

这里利用的是酸酐来制备酰胺,而不是直接利用酰基化反应制备。酸酐不会使得酚羟基发生酰基化反应。

\begin{center}
    \scriptsize
    \schemestart
    \chemfig{*6(-(-NH_2)=-=(-OH)-=)} \+ \chemfig{-[:30](=[:90]O)-[:-30]O-[:30](=[:90]O)-[:-30] } \arrow{->} \chemfig{*6(-(-NH-C(-[:-150]CH_3)=[:-30]O)=-=(-OH)-=)}
    \schemestop
\end{center}

$\ce{-NH2}$ 保护

酰胺不容易被氧化,因此如果有需要保护的氨基可以先用酸酐,将氨基氧化化成酰胺,然后再用氧化剂氧化。

\paragraph{磺酰化反应(Hinsberg)反应}

这个反应可以区分伯胺、仲胺、叔胺,并分离他们三者。

\begin{center}
    $\ce{RNH2 + Ph-SO2Cl -> Ph-SO2NHR}$ (苯磺酰酯,固体)$\ce{->[NaOH]} \mbox{溶解}$ 

    $\ce{R2NH + Ph-SO2Cl -> Ph-SO2NR2}$  $\ce{->[NaOH]} \mbox{不溶解}$ 
    
    $\ce{R3N + Ph-SO2Cl ->} \mbox{不反应}$ 
\end{center}

\subsubsection{与亚硝酸反应}

胺的级数不同,这个反应将会有不同的性质(表\ref{tab:hinsberg_level})。

\begin{table}[h]
    \centering
    \begin{tabular}{lc}
        \toprule
        胺 & 反应 \\
        \midrule
        伯胺 & $\ce{RNH2 + O=N-OH ->[-H2O] R-N^+#N-OH-}$ \\
        仲胺 & $\ce{R2NH + HO-NO ->[-H2O]R2N-N=O}$ \\ 
        叔胺 & 不易反应 \\
        \bottomrule
    \end{tabular}
    \caption{不同级数的胺的反应}
    \label{tab:hinsberg_level}
\end{table}


\paragraph{伯胺} 脂肪胺的产物不稳定,可以发生进一步分解,定量产生 $\ce{N2}$ 气体,可以进一步分析。芳香胺可以产生重氮盐,产物比较稳定,不易分解。

\paragraph{仲胺} 脂肪胺生成N-亚硝胺,是换色的油状物。芳香胺生成黄色的固体。

\paragraph{叔胺} 脂肪胺不易反应,芳香胺中,由于 $\ce{-N(R)3}$ 的给电子作用,芳环的亲电能力增强,可以进一步发生亲电反应。