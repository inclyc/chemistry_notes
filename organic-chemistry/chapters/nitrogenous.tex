\chapter{含氮有机化合物}
\section{硝基化合物}

\subsection{硝基化合物的物理性质}

一般以固体形式存在,难溶于水


\subsection{硝基化合物的化学性质}

\subsubsection{脂肪族}

$\ce{CH3NO2}$ 酸性、亲核性、还原反应

$\ce{O2N-CH2-}$可以作为很好的亲核试剂$\ce{Nu-}$,包括与羟基化合物的缩合,有点类似羟醛缩合反应。

硝基会减弱碱性、降低亲电取代的活性,增加亲核取代的活性(例如氯苯转变为苯酚)

\subsubsection{芳香族硝基化合物}


\begin{center}
    \small
    \schemestart
    \chemfig{*6(-=-(-NO_2)=-=)}
    \schemestop
\end{center}

\input{chapters/nitrogenous/amonia.tex}

\section{季铵盐和季铵碱}

\subsection{季铵盐}

\subsubsection{制法}

\begin{center}
    $\ce{R3N + RX -> R4N+X-}$ 
\end{center}

\subsubsection{主要作用}

阳离子表面活性剂、可做乳化剂、杀菌剂。长链季铵盐做相转移催化剂(PTR)。

\subsection{季铵碱}

\subsubsection{制法}

季铵盐和强碱反应,生成季铵盐。

\subsubsection{热不稳定性}

\begin{table}[h]
    \centering
    \begin{tabular}{ll}
        不含有$\beta - H$ &  $\mathrm{S_N2}$取代 $\ce{CH4N+OH- -> CH3OH + (CH3)3N}$  \\ 
        有$\beta - H$ (一种) &  发生$\beta$消除反应 \\
        有$\beta - H$ (二种) &  按Hoffman规则发生$\beta$消除反应\\
    \end{tabular}
\end{table}


环胺的结构推导可以利用这一反应。

\begin{center}
    \small
    \schemestart
    \chemfig{[:-18]*5(-\chembelow{N}{H}----)} \arrow{->[2$\ce{CH3I}$]} \chemfig{[:-18]*5(-\chemabove{N}{+}(-[:-30]CH_3)(-[:-150]H_3C)----)} \arrow{->[$\ce{Ag2O}$][$\ce{H2O}, \Delta$]} \chemfig{(-[:-60]N|{(CH3)_2})-[:30]-[:-30]=[:30]}
    \arrow(@c1--){0}[-90,1.2] {}
    \arrow{->[$\ce{CH3I}$]} {} \arrow{->[$\ce{Ag2O}$][$\ce{H2O}, \Delta$]}[,1.4] \chemfig{=[:30]-[:-30]=[:30]}
    \schemestop
\end{center}

% \chemfig{[:30]*6(-----(-)=)}

\section{重氮和偶氮化合物}


\begin{table}[h]
    \centering
    \begin{tabular}{ll}
        偶氮化合物 & $\ce{R-N=N-R}$ \\ 
        重氮化合物 & $\ce{R-N#N+ \quad Cl-}$ \\
    \end{tabular}
\end{table}


\subsection{偶氮化合物}

很容易发生均裂。产生自由基,是很好的链引发剂。

\subsection{重氮化合物}


\subsubsection{制备}

\begin{center}
    \small
    \schemestart
    \chemfig{*6(-=-=(-NH_2)-=)} + $\ce{NaNO2}$ + $\ce{HCl}$   \arrow{->[$0-5^\circ C$]} \chemfig{*6(-=-(-\chemabove{N}{\oplus}~[:0]NCl^{-})=-=)}
    \schemestop
\end{center}


\subsubsection{芳香重氮盐的性质}

\paragraph{被$\ce{H}$取代}

意义:生成氨基,同时又把氨基去掉了。可以合成一些不满足苯环上的定位规律的物质。

\begin{center}
    \small
    \schemestart
    \chemfig{*6(-=-=-=)} \arrow{->} \chemfig{*6(-(-Br)=-(-Br)=-(-Br)=)}
    \schemestop
\end{center}

\begin{center}
    \small
    \schemestart
    \chemfig{*6(-=-=-=)} \arrow{->} \chemfig{*6(-=(-NO_2)-=(-CH_3)-=)}
    \schemestop
\end{center}