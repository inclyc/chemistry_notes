    
    \section{化学性质}
    
    \subsection{$\alpha-\ce{H}$键}
    \scriptsize \chemfig{-C(-[:90])(-[:-90]H)-C=O} \normalsize 由于双键的氧加强了$\alpha-\ce{H}$上的碳氢键的极性。醛酮可以在$\alpha-\ce{H}$ 异裂后产生阴离子,可以发生缩合反应和异裂取代反应。

    \subsection{醛基的氧化反应}

    解释醛基的氧化还原性质可以通过与$\alpha-\ce{H}$的比较来进行。

    \subsection{亲核加成}

    \subsubsection{与弱酸的加成}

    与HCN,得到重要的合成材料$\alpha-$羟基腈,这个物质对于合成$\alpha-$羟基羧酸很有用。

    \begin{figure}[H]
        \scriptsize
        \centering
        \schemestart
        \chemfig{C(-[:-135])(-[:135])=O} \+ HCN \arrow \chemfig{-C(-[:90]OH)(-[:-90])-CN} 
        \schemestop
    \end{figure}

    有机玻璃的制备:

    \begin{figure}[H]
        \scriptsize
        \centering
        \schemestart
        \chemfig{C(-[:135]H_3C)(-[:-135]H_3C)=O} \+ HCN \arrow{->[NaOH]} \chemfig{C(-[:135]H_3C)(-[:-135]H_3C)(-[:45]CN)(-[:135]H_3C)(-[:-45]OH)}
        \schemestop
    \end{figure}

    
    \subsubsection{与$\ce{ROH}$的加成}


    \begin{figure}[H]
        \scriptsize
        \centering
        \schemestart
        \chemfig{C(=[:90]@{c1o}\charge{150=\:,30=\:}{O})(-[:-150]H_3C)(-[:-30]CH_3)} \arrow(c1--){0}[45,0.5] \chemfig{@{c1h}H-[@{c1hb}]@{c1b}B} \arrow(@c1--){<=>} \chemfig{C(=[:90]\charge{90:1pt=+}{O}H)(-[:-150]H_3C)(-[:-30]CH_3)} \+ \chemfig{CH_3OH} \arrow{<=>} \chemfig{CH_3-C(-[:90]OH)(-[:-90]OCH_3(-[:-90]H))-CH_3}
        \schemestop
        \chemmove[]{
            \draw (c1o) -- (c1b);
            % \draw (c1hb) .. controls +(-90:5mm) and +(-150:5mm) .. (c1b);
        }
    \end{figure}

    \paragraph{活性}

    $\ce{H+}$催化,体积效应比较大,醛、环酮的活性高

    \paragraph{二元醇}

    \begin{center}
        \scriptsize
        \schemestart
        \chemfig{[:-30]*6(---(=O)---)} \+ \chemfig{CH_2(-[:-90]CH_2-[:0]OH)-OH} \arrow{<=>[$\ce{H+}$]} \chemfig{[:-30]*6(---(-[:45]O-[:0]C?H_2)(-[:-45]O-[:0]C?H_2)---)}
        \schemestop
    \end{center}

    
    \paragraph{应用}

    \begin{enumerate}
        \item 缩醛或缩酮在碱性条件、氧化条件下是稳定的。
        \item 在酸性条件下可逆,可以变回原来的醛或者酮。
    \end{enumerate}

    可以用于保护羰基$\ce{C=O}$官能团

    \subsubsection{与$\ce{NaHSO3}$的反应}

    醛、甲基酮可以与亚硫酸氢钠饱和溶液发生加成反应,生成结晶的$\alpha-$羟基磺酸钠

    \begin{figure}[H]
        \scriptsize
        \centering
        \schemestart
        \chemfig{@{c}C(-[:120]R)(-[:-120]H_3C)=O} \+ \chemfig{@{ss}\charge{90=\:}{S}(=[:-90]O)(-[:150]OH)-[:30]ONa} \arrow{<=>} \chemfig{C(-[:135]R)(-[:-135]H_3C)(-[:45]ONa)(-[:-45]SO_3H)} \arrow{<=>} \chemfig{C(-[:135]R)(-[:-135]H_3C)(-[:45]OH)(-[:-45]SO_3Na)}
        \schemestop
        \chemmove[shorten <= 5pt, shorten >= 5pt]{
            \draw (ss) .. controls +(90:20mm) and +(60:3mm) .. (c);
        }
    \end{figure}


    \paragraph{活性} 体积反应,醛、环酮、甲基酮反应

    \paragraph{应用} 现象:鉴别,可逆:制备提纯


    \subsubsection{与含氮化合物及其衍生物}

    \begin{center}
        \scriptsize
        \schemestart
        \chemfig{C(-[:135])(-[:-135])=O} \+ \chemfig{NH_2Y} \arrow{<=>} \chemfig{-[,0.5]C(-[:90]OH)(-[:-90,0.5])-NHY} \arrow{->[-$\ce{H2O}$]} \chemfig{C(-[:135])(-[:-135])=NY} 
        \schemestop
    \end{center}

    \paragraph{历程} 先加成,后消除,pH=4~5 催化。不用强酸催化的原因是:如果酸性过强,会导致生成铵盐,失去电子,不反应。

    \paragraph{应用} 用于鉴定醛和酮。

    \begin{center}
        \scriptsize
        \schemestart
        \chemfig{*6((-O_2N)-=(-NO_2)-(-NHNH_2)=-=)} \arrow{->[醛、酮]} \chemfig{*6((-O_2N)-=(-NO_2)-(-NHN=[:0]C(-[:-90]R)-H)=-=)}
        \schemestop
    \end{center}

    醛、酮的分离、纯化。结晶多为晶形沉淀。加成产物可以在稀酸条件下可以水解成原来的醛或者酮。


    \subsubsection{与金属有机化合物$\ce{RMgX}$}

    \begin{center}
        \scriptsize
        \schemestart
        \chemfig{C(-[:135])(-[:-135])=O} + RMgX \arrow{<=>} \chemfig{-[,0.5]C(-[:90]OMgX)(-[:-90,0.5])-R} \arrow{->[$\ce{H+}$][$\ce{H2O}$]}  \chemfig{-[,0.5]C(-[:90]OH)(-[:-90,0.5])-R}
        \schemestop
    \end{center}

    \paragraph{应用} 制醇

    \subsubsection{与金属有机化合物$\ce{RLi}$}

    烷基锂的亲核性极强,而且几乎不受到体积效应的影响。要求反应必须绝对完全无水。

    \subsection{还原反应}

    \subsubsection{催化加氢}

    \begin{center}
        \scriptsize
        \schemestart
        \chemfig{C(-[:135])(-[:-135])=O} \arrow{->[$\ce{Ni}$][$\ce{H2}$]} \chemfig{-[,0.5]C(-[:90]H)(-[:0])-OH}
        \schemestop
    \end{center}

    \subsubsection{$\ce{NaBH4}$}
    \begin{center}
        \scriptsize
        \schemestart
        \chemfig{C(-[:135])(-[:-135])=O} \+ $\ce{H-}$ \arrow{->} \chemfig{-C(-[:-90]H)-\charge{80:2pt=-}{O}} \arrow{->[$\ce{H+}$]} \chemfig{-C(-[:-90]H)-OH}
        \schemestop
    \end{center}

    \begin{center}
        \scriptsize
        \schemestart
        \chemfig{[:-30]*6(-=-(=O)---)} \arrow{->} \chemfig{[:-30]*6(-=-(-OH)---)}
        \schemestop
    \end{center}

    对\chemfig{C=[,0.5]C},\chemfig{C~[,0.5]C}不反应,可以有选择性地加成。这一选择性主要是相对Raney Ni而言,这些金属催化加氢的方法会同时将双键还原,而用金属氧化物不会


    \subsubsection{Clemmensen 还原 以及 Wolff-Kishner 反应}

    Clemmensen 还原 和 Wolff-Kishner还原的反应条件不一样,反应结果均为将羰基还原为亚甲基。

    \begin{center}
        \scriptsize
        \schemestart
        \chemfig{C(-[:135])(-[:-135])=O} \arrow{->[$\ce{Zn, Hg, HCl}$][$\ce{\Delta}$, 回流]}[,1.5] \chemfig{-[,0.4]CH_2-[,0.4]}
        \schemestop
    \end{center}

    \begin{center}
        \scriptsize
        \schemestart
        \chemfig{C(-[:135])(-[:-135])=O} \arrow{->[$\ce{NH2NH2}$][$\ce{OH-}, \Delta$]}[,1.5] \chemfig{-[,0.4]CH_2-[,0.4]}
        \schemestop
    \end{center}

    黄鸣龙对Wolff-Kishner还原反应作出了改进,这个反应也叫黄鸣龙还原法。


    \subsubsection{Cannizzaro歧化反应}

    不含有$\alpha - \ce{H}$的醛在浓碱性条件下可以发生分子间的氧化还原反应,一分子的醛被氧化成羧酸,另一分子的醛被还原为醇,该反应称为Cannizzaro反应。

    \begin{center}
        \scriptsize
        \schemestart
        \chemfig{H@{a}CHO} \+ \chemfig{@{b}O\charge{60=-}{H}} \arrow{->[][][]} \chemfig{H-C(-[:90]O^{-})(-[:-90]OH)-H} 
        \schemestop

        \chemmove[]{
            \draw (b) .. controls +(90:10mm) and +(60:5mm) .. (a); 
        }
    \end{center}

    \subsubsection{$\alpha - \ce{H}$反应}

    卤代反应

    
    \begin{center}
        \scriptsize
        \schemestart
        \chemfig{R-C(=[:90]O)-CH_3} \+ \chemfig{Br_2} \arrow{<=>[$\ce{H+}$ / $\ce{OH-}$]}[,1.5] \chemfig{R-C(=[:90]O)-CH_2Br} \+ \chemfig{HBr}
        \schemestop
    \end{center}

    碱催化的条件下难以停留在第一取代阶段,会继续催化。

    \begin{center}
        \scriptsize
        \schemestart
        \chemfig{R-C(=[:90]O)-CH_3} \arrow{<=>[$\ce{OH-}$]} \chemfig{R-C(=[:90]O)-C\charge{90=-}{H}_2}  \+ \chemfig{Br_2} \arrow{<=>[]}[,1.5] \chemfig{R-C(=[:90]O)-CH_2Br} + \chemfig{HBr}
        \schemestop
    \end{center}


    酸催化:

    \begin{center}
        \scriptsize
        \schemestart
        \chemfig{R-C(=[:90]O)-CH_3} \arrow{<=>[$\ce{H+}$]} \chemfig{R-C(=[:90]O\charge{60=+}{H})-CH_3}  \+ \chemfig{Br_2} \arrow{<=>[]}[,1.5] \chemfig{R-C(=[:90]O)-CH_2Br} + \chemfig{HBr}
        \schemestop
    \end{center}
    

    可以停留在第一阶段。

    \subsubsection{碘仿反应}

    这个反应生成特殊气味的黄色结晶。

    \begin{center}
        \scriptsize
        \schemestart
        \chemfig{R-C(=[:90]\charge{90=\:,180=\:}{O})-CH_3} \arrow{->[$\ce{I2}$]} \chemfig{R-C(=[:90]\charge{90=\:,180=\:}{O})-CI_3} \arrow{<=>[$\ce{OH-}$]} \chemfig{R-C(=[:90]\charge{90=\:,180=\:}{O})-\charge{60=-}{O}} + \chemfig{CHI_3}
        \schemestop
    \end{center}

    \subsubsection{羟醛缩合反应}

    羟醛缩合的活性:醛 > 酮,这个反应主要以醛为主,酮比较难反应(体积效应)。羟醛缩合为我们提供了一个比较重要的大量增加碳链的方式

    \subsection{氧化反应}

    \begin{itemize}
        \item  Tollen试剂,硝酸银的氨溶液。这个反应条件比较苛刻,不易成功。
        \item  Fehling试剂 硫酸铜与酒石酸钾钠的氢氧化钠溶液。这种试剂只能氧化脂肪醛、不能氧化芳香醛
    \end{itemize}

    酮一般很难被氧化,但酮可以与过氧酸发生Bayer-Villiger反应,生成插氧产物(酯)。

    原理上需要先生成互变异构,再氧化烯醇式双键。
