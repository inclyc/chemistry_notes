
    \section{不饱和的醛或者酮}

    不饱和羰基化合物,就是分子中除了羰基外,还含有不饱和键的化合物。

    \subsection{$\alpha, \beta$ 不饱和醛酮}

    $\ce{C=C}$上可以亲电加成$\ce{C=O}$上可以亲核加成。

    \subsubsection{氧化反应}

    弱氧化剂下,将醛氧化为羧酸(盐)。强氧化剂下,将双键和醛基都氧化,生成两种羧酸(或进一步反应)。

    \subsubsection{还原反应}

    如果是$\ce{NaBH4}$ 双键不变,将醛还原成醇(可以得到烯醇)。如果是催化加氢,则还原为饱和的醇。

    \subsubsection{双烯合成反应}

    $\alpha, \beta$ 不饱和醛的双烯合成反应极易进行。双烯体上存在供电子体、单烯体上有吸电子基团,有利于双烯合成反应。

    \subsubsection{亲电加成 $\ce{HBr}$}

    这种加成有两种进攻方法,分别导致1, 2加成和1, 4加成。但是这两种加成的结果是一样的,都相当于碳碳双键的加成。$\ce{H+}$ 会进攻$\alpha$-C,$\ce{Br}$最终和$\beta-$C相连。

    \subsubsection{亲核加成 $\ce{HCN}$}

    1, 2加成相当于对羰基的加成,1, 4加成相当于对碳碳双键的加成。1, 2加成和1, 4加成的影响因素:

    \begin{enumerate}
        \item 强亲核试剂有利于1, 2加成,弱亲核试剂有利于1, 4加成。
        \item 体积效应:大体积有利于1, 4加成。
    \end{enumerate}

    \subsubsection{缩合反应}

    不含有$\gamma$-H的不饱和醛酮,与含有$\alpha$-H的酮的反应。注意,这里没提$\alpha$-H醛,是因为$\alpha$-H醛可以自己和自己发生缩合反应。
    \begin{centering}
        
    \end{centering}

    含有$\gamma$-H的不饱和醛,可以自己和自己进行1, 2加成


