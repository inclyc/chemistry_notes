\chapter{醌}

    \section{概述}

    苯醌、萘醌、蒽醌。更多体现芳酮的性质。

    通常是存在很大的离域$\pi$键,会存在跃迁。醌也具有很好的对称性。对苯醌、萘醌、蒽醌,三种物质。苯醌到蒽醌的性质会从$\alpha, \beta$不饱和酮逐渐过渡到类似芳酮。

    \section{化学性质}

    \subsection{亲电加成}

    \subsubsection{与$\ce{X2}$的加成}

    \begin{center}
        \scriptsize
        \schemestart
        \chemfig{*6(-(=O)-=-(=O)-=)} \+ $\ce{Br2}$ \arrow \chemfig{*6(-(=O)-(-Br)-(-Br)-(=O)-=)} \arrow{->[-$\ce{HBr}$]} \chemfig{*6(-(=O)-=(-Br)-(=O)-=)} 
        \schemestop
    \end{center}

    \subsection{亲核加成}

    \subsubsection{与$\ce{HCN}$}

    \begin{center}
        \scriptsize
        \schemestart
        \chemfig{*6(-(=O)-=-(=O)-=)} \+ $\ce{CN-}$ \arrow \chemfig{*6(-(=O)-(-[:-30]CN)(-[:30]H)-=(-O)-=)}
        \schemestop
    \end{center} % 1,4 加成,互变异构

    \subsubsection{对$\ce{NH2OH}$的加成}
    
    更容易发生1,2加成

    \subsubsection{与$\ce{RMgX}$的加成}


    \begin{center}
        \scriptsize
        \schemestart
        \chemfig{*6(-(=O)-=-(=O)-=)} \arrow{->[$\ce{RM}$]} \chemfig{*6(-(-[:-150]R)(-[:-30]OM)-=-(=O)-=)} \arrow{->[$\ce{H3O+}$]} \chemfig{*6(-(-[:-150]R)(-[:-30]OH)-=-(=O)-=)}
        \schemestop
    \end{center}

    这个反应可以生成醌醇。醌醇可以重排位羟基取代的苯二酚。

    \begin{center}
        \scriptsize
        \schemestart
        \chemfig{*6(-(-[:-150]R)(-[:-30]OH)-=-(=O)-=)} \arrow{->} \chemfig{*6((-R)-(-OH)=-=(-OH)-=)}
        \schemestop
    \end{center}


    \section{蒽醌的性质}

    \subsection{亲电取代(难)}

    可以跟硫酸进行磺化反应,注意这些反应条件会越来越困难。

    \subsection{还原反应}

    在$\ce{Zn}$,$\ce{Hg}$的作用下可以还原他的氧元素

    \begin{center}
        \scriptsize
        \schemestart
        \chemfig{*6(-=*6(-(=O)-*6(-=-=-=)--(=O)-)-=-=)} \arrow \chemfig{*6(-=*6(--*6(-=-=-=)---)-=-=)}
        \schemestop
    \end{center}

