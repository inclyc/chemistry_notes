\chapter{酚}
\section{概述}
酚(phenols, $\ce{Ar-OH}$)
羟基直接与苯环\textit{$sp^2$}碳原子相连接的化合物称为酚,苯酚(\scalebox{0.6}[0.6]{\chemfig{OH-[:180,,1]=_[:240]-[:180]=_[:120]-[:60]=_(-[:300])}})最为常见,另外还有萘酚等。

\subsection{分类}
\begin{enumerate}
    \item 芳环类型
    \item $\ce{-OH}$的个数
\end{enumerate}
\subsection{命名}
不考,没啥用,知道大概啥顺序就行

\section{物理性质}
\begin{enumerate}
    \item 形成分子间氢键,沸点较高
    \item 对称性较好,熔点较高,常温下是固体
    \item 水溶性 苯酚在冷水中微溶,加热时可以无限溶解 \par
    与醇相比,苯酚和水形成氢键的能力比较差。($sp^3$和$sp^2$)\footnote{$sp^3$杂化距离更远,跟适合形成氢键。酚只有一对$sp^2$电子,比醇还少一对}
    \item 酚本身无色,容易被空气中的氧气氧化而略呈红色或褐色
    \item 有毒
\end{enumerate}

\section{化学性质}
\subsection{$\ce{C-O}$ 键}
不容易断。酚具有一定的离域,具有一定的双键性质。

\subsection{$\ce{O-H}$ 键}
\begin{enumerate}
    \item 酸性
    \item 含有未成对的电子---碱性、亲核性
    \item 配位性,可以和铁发生显色反应
\end{enumerate}

\subsection{苯环}
\begin{enumerate}
    \item 亲电取代,$\ce{-OH}$是邻对位定位基,可以使苯环活化
    \item 氧化(复杂,机理不甚明确)
\end{enumerate}

\subsection{酚的酸性}

\subsubsection{不同类}
$\ce{H2CO3} > \ce{Ph-OH} > \ce{H2O} > \ce{ROH}$

若芳香基团上有吸电子基团,则酚酸性增强

\subsubsection{同类}

\begin{enumerate}
    \item 吸电子基 酸性增加 \par
    \begin{figure}[H]
        \scriptsize
        \centering
        \chemfig{*6(=-=-(-OH)=-)} < \chemfig{*6(=-(-NO_2)=-(-OH)=-)} < \chemfig{*6(=(-NO_2)-=-(-OH)=-)}
    \end{figure}
    \item 供电子基 酸性降低 \par
    \begin{figure}[H]   
        \scriptsize
        \centering
        \chemfig{*6(=-=-(-OH)=-)} < \chemfig{*6(=-(-CH_3)=-(-OH)=-)} < \chemfig{*6(=(-CH_3)-=-(-OH)=-)}
    \end{figure}
    \item 卤素 吸电子,酸性增加
\end{enumerate}

\subsection{显色反应}
可以与铁形成紫色复合物

$\ce{6C6H6OH + FeCl3 -> H_3[Fe(OC6H5)] + 3HCl}$

\subsection{亲核性}

亲核性 只和易发生亲核反应的物质反应

如 
\begin{figure}[H]
    \scriptsize
    \centering
    \schemestart
    \chemfig{R-C(=[:90]O)-Cl} \+ \chemfig{*6(-=-=(-OH)-=)} \arrow
    \schemestop
\end{figure}

\subsection{亲电取代}

\subsubsection{卤代反应}

\begin{figure}[H]
    \scriptsize
    \centering
    \schemestart
    \chemfig{*6(-=-=(-OH)-=)}  \+ $\ce{Br2}$ \arrow \chemfig{*6(-(-Br)=-(-Br)=(-OH)-(-Br)=)}
    \schemestop
\end{figure}
原因:水中存在 \scriptsize\chemfig{*6(-=-=(-{O^{-}})-=)}\normalsize,要想控制反应进度,可以在有机溶剂中进行:

\begin{figure}[H]
    \scriptsize
    \centering
    \schemestart
    \chemfig{*6(-=-=(-OH)-=)} \arrow{->[$\ce{Br}$][$\ce{CS2}, 0^\circ \mathrm{C}$]} \chemfig{*6(-(-Br)=-=(-OH)-=)} + \chemfig{*6(-=-(-Br)=(-OH)-=)}
    \schemestop
\end{figure}

\subsubsection{磺化反应}

\begin{figure}[H]
    \scriptsize
    \centering
    \schemestart
    \chemfig{*6(-=-=(-OH)-=)} \+ $\ce{H2SO4}$ \arrow \chemfig{*6(-=-(-SO_3)=(-OH)-=)} \+ \chemfig{*6(-(-SO_3)=-=(-OH)-=)}
    \schemestop
\end{figure}
\subsubsection{硝化反应}

一般来讲不要直接硝化,不然产率会很低

\begin{figure}[H]
    \scriptsize
    \centering
    \schemestart
    \chemfig{*6(-=-=(-OH)-=)} \arrow{->[$\ce{NaNO2}$][$\ce{H+}$]} \chemfig{*6(-(-NO)=-=(-OH)-=)} \arrow{->[$\ce{HNO3}$]} \chemfig{*6(-(-NO_2)=-=(-OH)-=)}
    \schemestop
\end{figure}

\subsubsection{傅氏反应}

\begin{figure}[H]
    \scriptsize
    \centering
    \schemestart
    \chemfig{*6(-=-=(-OH)-=)} \+ $\ce{RCl}$ \arrow{->[HF]} \chemfig{*6(-(-R)=-=(-OH)-=)}
    \schemestop
\end{figure}

\subsubsection{氧化反应}

\begin{figure}[H]
    \scriptsize
    \centering
    \schemestart
    \chemfig{*6(-=-=-=)} \arrow{->[$\mbox{[O]}$]} \chemfig{*6(-(=O)-=-(=O)-=)}
    \schemestop
\end{figure}

\subsection{酚的制法(了解)}

