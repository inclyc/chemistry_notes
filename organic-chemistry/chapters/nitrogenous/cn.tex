\section{腈和异氰酸酯}

\subsection{腈类}

低级腈通常易溶于水,可以形成分子间氢键。

\subsubsection{催化加氢}

\begin{equation*}
    \ce{RCN ->[H2][Ni] RCH2NH2} \qquad \mbox{存在副反应}
\end{equation*}


\subsection{亲核加成反应}

腈类物质可以水解变成羧酸。

\subsubsection{与$\ce{RMgX}$的反应}

可以与格氏试剂生成酮。

\begin{center}
    \scriptsize
    \schemestart
    \chemfig{R-C~N} \+ \chemfig{R-MgX} \arrow{->} \chemfig{R-C(-[:-90]R)=N-MgX} \arrow{->} \chemfig{R-C(=[:-90]O)-R}
    \schemestop
\end{center}

\subsubsection{$\alpha - \ce{H}$的反应}

和羟醛缩合相似,$\ce{CH3-CN}$可以形成碳负离子:\chemfig{\charge{90:3pt=$\ominus$}{C}H_2-CN},这个碳负离子可以作为亲核试剂和其他物质反应。


% 克莱森缩合,这是什么?

\subsection{异腈}

\subsubsection{结构$\ce{R-NC}$}

\begin{equation*}
    \ce{RNH2 + CHCl3 + 3KOH -> RNC + 3KCl + 3H2O}
\end{equation*}

% 化学性质
% 异腈类物质在手性合成上有非常好的性质。

% 异腈是很重要的原料,或者是某一有机合成步骤的前体

\subsubsection{水解}

\begin{equation*}
    \ce{RN#C + H2O ->[\mbox{亲核加成}] R-NH=C-OH ->[H+] RNH2 + HCOOH}
\end{equation*}

\subsubsection{加氢}
\begin{equation*}
    \ce{RN#C + H2 ->[Ni] RNHCH3}
\end{equation*}

\subsubsection{异构化}
\begin{equation*}
    \ce{RNC ->[\Delta] RCN}
\end{equation*}
\subsubsection{氧化}

\begin{equation*}
    \ce{Ph-NC + HgO -> Ph-N=C=O}
\end{equation*}

\subsection{异氰酸酯}

\begin{table}[h]
    \centering
    \begin{tabular}{ll}
        氰酸   & $\ce{HO-CN}$   \\
        异氰酸 & $\ce{H-N=C=O}$ \\
    \end{tabular}
\end{table}

异氰酸苯酯比较稳定。苯环可以起到稳定作用

\subsubsection{亲核加成反应}

注意,加成加成在羰基上。

\begin{center}
    \small
    \schemestart
    \chemfig{-N=\charge{90:3pt=$\delta+$}{C}=\charge{90:3pt=$\delta-$}{O}}
    \schemestop
\end{center}

\begin{center}
    \scriptsize
    \schemestart
    \chemfig{*6(-=-(-N=[:0]C=[:0]O)=-=)} \+ $\ce{H2O}$ \arrow{->[加成]} \chemfig{*6(-=-(-N=[:0]C(-[:90]OH)-[:0]OH)=-=)}
    \arrow(@c1--){0}[-90, 1.5] {}
    \arrow{->[重排]} \chemfig{*6(-=-(-NH-[:0]C(=[:90]O)-[:0]OH)=-=)}
    \schemestop
\end{center}


