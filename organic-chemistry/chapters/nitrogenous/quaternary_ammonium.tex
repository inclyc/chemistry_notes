
\section{季铵盐和季铵碱}

\subsection{季铵盐}

\subsubsection{制法}

\begin{center}
    $\ce{R3N + RX -> R4N+X-}$ 
\end{center}

\subsubsection{主要作用}

阳离子表面活性剂、可做乳化剂、杀菌剂。长链季铵盐做相转移催化剂(PTR)。

\subsection{季铵碱}

\subsubsection{制法}

季铵盐和强碱反应,生成季铵盐。

\subsubsection{热不稳定性}

\begin{table}[h]
    \centering
    \begin{tabular}{ll}
        不含有$\beta - H$ &  $\mathrm{S_N2}$取代 $\ce{CH4N+OH- -> CH3OH + (CH3)3N}$  \\ 
        有$\beta - H$ (一种) &  发生$\beta$消除反应 \\
        有$\beta - H$ (二种) &  按Hoffman规则发生$\beta$消除反应\\
    \end{tabular}
\end{table}


环胺的结构推导可以利用这一反应。

\begin{center}
    \small
    \schemestart
    \chemfig{[:-18]*5(-\chembelow{N}{H}----)} \arrow{->[2$\ce{CH3I}$]} \chemfig{[:-18]*5(-\chemabove{N}{+}(-[:-30]CH_3)(-[:-150]H_3C)----)} \arrow{->[$\ce{Ag2O}$][$\ce{H2O}, \Delta$]} \chemfig{(-[:-60]N|{(CH3)_2})-[:30]-[:-30]=[:30]}
    \arrow(@c1--){0}[-90,1.2] {}
    \arrow{->[$\ce{CH3I}$]} {} \arrow{->[$\ce{Ag2O}$][$\ce{H2O}, \Delta$]}[,1.4] \chemfig{=[:30]-[:-30]=[:30]}
    \schemestop
\end{center}

% \chemfig{[:30]*6(-----(-)=)}
