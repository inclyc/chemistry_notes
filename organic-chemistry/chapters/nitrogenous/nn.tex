\section{重氮和偶氮化合物}


\begin{table}[h]
    \centering
    \begin{tabular}{ll}
        偶氮化合物 & $\ce{R-N=N-R}$ \\ 
        重氮化合物 & $\ce{R-N#N+ \quad Cl-}$ \\
    \end{tabular}
\end{table}


\subsection{偶氮化合物}

很容易发生均裂。产生自由基,是很好的链引发剂。

\subsection{重氮化合物}

\subsubsection{制备}

\begin{center}
    \small
    \schemestart
    \chemfig{*6(-=-=(-NH_2)-=)} + $\ce{NaNO2}$ + $\ce{HCl}$   \arrow{->[$0-5^\circ C$]} \chemfig{*6(-=-(-\chemabove{N}{\oplus}~[:0]NCl^{-})=-=)}
    \schemestop
\end{center}


\subsection{芳香重氮盐的性质}

\subsubsection{被$\ce{H}$取代}

意义:生成氨基,同时又把氨基去掉了。可以合成一些不满足苯环上的定位规律的物质。

\begin{center}
    \small
    \schemestart
    \chemfig{*6(-=-=-=)} \arrow{->} \chemfig{*6(-(-Br)=-(-Br)=-(-Br)=)}
    \schemestop
\end{center}

\begin{center}
    \small
    \schemestart
    \chemfig{*6(-=-=-=)} \arrow{->} \chemfig{*6(-=(-NO_2)-=(-CH_3)-=)}
    \schemestop
\end{center}

\subsubsection{被$\ce{-OH}$取代}

\subsubsection{被卤素取代}

\begin{center}
    \small
    \schemestart
    \chemfig{*6(-=-(-N_2^+|Cl)=-=)} \arrow{->[$\ce{CuCl + HCl}$]}[,1.6] \chemfig{*6(-=-(-Cl)=-=)} \+ $\ce{N2}$
    \schemestop
\end{center}

\subsubsection{被$\ce{-CN}$取代}


\begin{center}
    \small
    \schemestart
    \chemfig{*6(-=-(-N_2^+Cl)=-=)} \arrow{->[$\ce{CuCN + KCN}$]}[,1.6] \chemfig{*6(-=-(-CN)=-=)} \+ $\ce{N2}$
    \schemestop
\end{center}


\subsubsection{还原反应}


\begin{center}
  $\ce{-N2^+ Cl}$ > $\ce{-NO2}$ > $\ce{-CHO}$
\end{center}


还原生成$\ce{Ph-NHNH2}$

\subsubsection{偶联反应}

\begin{center}
  \small
  \schemestart
  \chemfig{*6(-=-(-N_2^+Cl)=-=)} \+ \chemfig{*6(-=-(-Y)=-=)} \arrow{->} 
  \schemestop
\end{center}

对于亲电试剂
\begin{itemize}
  \item 芳环如果有吸电子基团,反应速率增大
  \item 芳环如果有给电子基团,反应速率减小
\end{itemize}

对于偶联剂
\begin{itemize}
  \item 芳环上的电子云密度越大,越容易偶联
  \item 芳环上有吸电子基,不偶联
  \item 偶联位置:邻对位取代基的对位(居多)或邻位
\end{itemize}


\subsection{重氮甲烷}

\begin{equation*}
  \ce{C^+H3-N^+#N}
\end{equation*}


对甲基苯磺酰胺和氢氧化钠反应,可以得到重氮甲烷。

\subsubsection{性质}

有毒,受热易分解,易爆炸。见光易分解,生成$\ce{N2}$和$\ce{=CH2}$(卡宾)。

\paragraph{醛酮} 和酮反应,生成比原来的酮多一个碳的酮。和醛反应可以生成甲基酮。

\paragraph{酰卤} 生成烯酮(wolff重排),生成其他衍生物。


