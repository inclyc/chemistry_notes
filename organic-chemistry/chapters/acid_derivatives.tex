\chapter{羧酸的衍生物}

\section{物理性质}


\subsection{酰卤}

\begin{center}
    \chemfig{R-[:30]C(=[:90]O)-[:-30]X}
\end{center}

不能形成分子间的氢键。

\subsection{酯、酸酐}

\begin{center}
    \chemfig{R-[:30]C(=[:90]O)-[:-30]OR}
\end{center}

\begin{center}
    \chemfig{R-[:30]C(=[:90]O)-[:-30]O-[:30]C(=[:90]O)-[:-30]R}
\end{center}

也不能形成氢键,低级酯一般为液体,高级酯一般为固体。
\section{化学性质}

\subsection{亲核取代}

衍生物的亲核取代反应机理上为先加成、再消去。


\begin{center}
    \small
    \centering
    \schemestart
    \chemfig{R-[:30]C(=[:90]O)-[:-30]L} \arrow{->[\ch{Nu\mch}]} \chemfig{R-C(-[:90]\ch{O\mch})(-[:-90]\ch{Nu})-L} \arrow{->[-L]} \chemfig{R-[:30]C(=[:90]O)-[:-30]Nu}
    \schemestop
\end{center}

\subsubsection{酰卤的水解}

水解反应的机理为亲核取代,不同衍生物的反应活性是:

\begin{equation*}
    \mbox{酰卤} > \mbox{酸酐} > \mbox{酯} > \mbox{酰胺}
\end{equation*}

酸酐的水解需要酸催化,酯还需要加热,酰胺还需要回流。酸催化的作用时增加碳上的正电荷。使得反应物更容易被水中的\ch{OH\mch}进攻。

\begin{center}
    \scriptsize
    \centering
    \schemestart
    \chemfig{R-[:30]C(=[:90]O)-[:-30]L} \arrow{->[\ch{H\pch}]} \chemfig{R-[:30]C(=[:90]OH\ch{\pch})-[:-30]L} \arrow{->[\ch{OH\mch}]} \chemfig{R-C(-[:90]OH\ch{\pch})(-[:-90]OH\ch{\mch})-L} 
    \arrow{->[\ch{-HL}][-\ch{H\pch}]} \chemfig{R-[:30]C(=[:90]O)-[:-30]OH}
    \schemestop
\end{center}


\subsubsection{酰卤制备酚酯}

酚酯一般难以用酸和酚直接制备,需要使用酰卤。

\begin{center}
    \centering
    \scriptsize
    \schemestart
    \chemfig{R-[:30]C(=[:90]O)-[:-30]Cl} \+ \chemfig{*6(-=-(-OH)=-=)} \arrow{->}
    \schemestop
\end{center}


\subsubsection{氨解反应和氨基的保护}

\begin{center}
    \centering
    \scriptsize
    \schemestart
    \chemfig{*6(-=-=(-NH_2)-=)} \arrow{->[\chemfig{CH_3-C(=[:90]O)-Cl}][或酸酐]}[,2.3] \chemfig{*6(-=-=(-NH-[:0]C(=O)-[:0]CH_3)-=)}
    \schemestop
\end{center}

\subsubsection{与\ch{RMgX}的反应}

\begin{center}
    \centering
    \scriptsize
    \schemestart
    \chemfig{R-[:30]C(=[:90]O)-[:-30]Cl} \arrow{->[\ch{RMgCl}]}[,1.2] \chemfig{R-C(-[:90]OMgCl)(-[:-90]R)-Cl} \arrow{->[-\ch{MgCl2}]}[,1.4] \chemfig{R-[:30]C(=[:90]O)-[:-30]R}
    \schemestop
\end{center}

生成的酮可以进一步和格氏试剂反应,但是第一步反应的活性更强,因此可以通过控制格氏试剂的量来控制反应的数量级。

如果是酯与格氏试剂反应,则更容易变为最终的醇。因为第一步酯和格氏试剂的反应本身比酰卤困难得多。如果想让酯停留在第一个取代的阶段,则需要控制空间位阻效应。

\subsection{还原反应}

另外,这些衍生物还可以发生还原反应


\section{卤代酸}

\subsection{$\alpha$-卤代酸}

卤原子受到羰基的影响,反应活性增强,因此易与各种亲核试剂发生反应,生成$\alpha$-取代羧酸。


\subsubsection{酸性}

\begin{center}
    $\ce{RCH2COOH}$ < \small\chemfig{R-CH(-[:90]Cl)-COOH} > \chemfig{R-CH(-[:90]OH)-COOH}
\end{center}


中间的羟基吸电子能力没有 $\ce{-OH}$ 强。这是因为 $\ce{-OH}$ 的氧原子还连着一个 $\ce{H}$ ,削弱了 $\ce{O}$  吸电子的能力。


\subsubsection{亲核取代}

\begin{center}
    \small
    \schemestart
    \chemfig{CH_2=CH_2} \arrow \chemfig{CH_2(-[:30]COOC_2H_5)-[:-30]COOC_2H_5}
    \schemestop
\end{center}

\subsection{$\beta$ - 卤代酸}


性质上和$\alpha$卤代酸的性质差不多。外加可以进行消去反应。


\section{羟基酸}



