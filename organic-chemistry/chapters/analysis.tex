\chapter{物理化学分析方法}

\section{紫外光谱}

紫外吸收光谱是由于分子中价电子的跃迁而产生的。

\subsection{UV的特点}

200nm - 400nm 波长。峰形状较宽,信号较少。

\subsection{跃迁方式}

\subsubsection{$\sigma - \sigma*$}

对应不含有$\pi$键的物质,例如饱和烷烃。$\lambda < 150 \ \mathrm{nm}$,在常见的紫外吸收光谱没有吸收。

\subsubsection{$n \rightarrow \sigma*$}

不含有$\pi$键,含有 $\ce{-X}$ $\ce{-OH}$ $\ce{-NH2}$ 等杂原子的饱和烃化合物

$150 \ \mathrm{nm} < \lambda < 200 \ \mathrm{nm}$

\subsubsection{$\pi \rightarrow \pi*$}

$\lambda > 200 \ \mathrm{nm}$ 一般用来表征共轭烯烃。

\subsection{生色基、助色基}

\subsubsection{生色基}
主要是一些大$\Pi$键,例如芳环、共轭烯烃中的$\Pi$键。
\subsubsection{助色基}
助色基团可以使紫外光谱的波长改变(或强度改变),例如苯环上的羟基。

\subsection{超共轭效应}
\begin{center}
    \small
    \schemestart
    \chemname{\chemfig{=[:30]-[:-30]=[:30]}}{$217 \ \mathrm{nm}$} \chemname{\chemfig{-[:-30]=[:30]-[:-30]=[:30]}}{$222 \ \mathrm{nm}$}
    \schemestop
\end{center}
\subsection{共轭效应}
苯环上连有的原子若有$p$轨道,则可以与苯环发生$p-\pi$共轭。
\subsection{空间效应}
指不同结构产生的大$\pi$键会导致不同的紫外吸收。