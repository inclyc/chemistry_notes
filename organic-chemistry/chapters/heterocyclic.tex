\chapter{杂环化合物}
\section{概述}
\subsection{分类}
\subsubsection{单环}
五元杂环、六元杂环
\subsubsection{稠杂环}
\section{含有一个杂原子的五元杂环化合物}

\begin{center}
    \chemname{\chemfig{[:-17]*5(-O-=-=-)}}{呋喃}
    \chemname[2.3ex]{\chemfig{[:-17]*5(-\chembelow{N}{H}-=-=-)}}{吡咯}
    \chemname{\chemfig{[:-17]*5(-S-=-=-)}}{噻吩}
\end{center}

\subsection{亲电反应活性}

\begin{center}
    \chemfig{[:-17]*5(-\chembelow{N}{H}-=-=-)} >  \chemfig{[:-17]*5(-O-=-=-)} > \chemfig{[:-17]*5(-S-=-=-)} > \chemfig{*6(-=-=-=)}
\end{center}

\subsection{环的稳定性}

离域能越大,越稳定。所有的这些五元杂环化合物都可以用硝化反应开环,同时,也不能用硝酸直接硝化,因为环不稳定。


\subsection{物理性质}

% 思考题:为什么呋喃难溶,而四氢呋喃可溶。
无色液体,通常不溶于水。而四氢呋喃可溶解。

\subsection{化学性质}

\subsubsection{亲电取代反应}

\paragraph{与$\ce{Cl2}$反应(亲电特性)}

\begin{center}
    \small
    \schemestart
    \chemfig{[:-17]*5(-\chembelow{N}{H}-=-=-)} \+ $\ce{Cl2}$ \arrow{->[反应难控制]}[,1.4]
    \arrow(@c1--){0}[-90,1.2] {}
    \chemfig{[:-17]*5(-O-=-=-)} \+ $\ce{Cl2}$ \arrow{->}[,1.4] \chemfig{[:-17]*5(-O-(-Cl)=-=-)}
    \schemestop
\end{center}

\paragraph{与$\ce{Br}$反应}

\begin{center}
    \small
    \schemestart
    \chemfig{[:-17]*5(-O-=-=-)} + $\ce{Br2}$ \arrow{->} \chemfig{[:-17]*5((-Br)-O-(-Br)=(-Br)-(-Br)=-)}
    \schemestop
\end{center}

\paragraph{硝化反应}

用温和的硝化试剂乙酰基硝酸酯$\ce{CH3COONO2}$硝化,且控制在低温条件(为了保持硝化试剂的稳定性)。

\paragraph{磺化}

用 \chemname[2.7ex]{\chemfig{*6(-\chemabove{N}{\oplus}(-\chembelow{S}{\ominus}O_3)=-=-=)}}{三氧化硫吡啶} 或 95\%的硫酸磺化
