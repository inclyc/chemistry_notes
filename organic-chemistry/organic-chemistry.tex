\documentclass[a4paper]{ctexrep}
\usepackage{mathtools}
\usepackage{tikz}
\usepackage{chemfig}
\usepackage{mhchem}


\author{Y.C. Long}
\title{有机化学 -- 孙净雪}


\begin{document}
    \maketitle
    \tableofcontents
    

    % 孙净雪 - 13384608260
    % 期末闭卷70分,出勤10分,课堂小测验10分,作业10分。作业交了就行。
    % 老师是两批老师,所以这个课会用一点时间过渡一下
    \chapter*{学期过渡部分}
    \section*{基本反应回顾}
    \begin{itemize}
        \item 烷烃--自由基引发,取代
        \item 烯烃--加成,亲电,$\pi$电子云是裸露在外的
        \item P--X 卤代烃 没有不饱和度,被亲核试剂进攻
        \item 芳烃 亲电试剂进攻
    \end{itemize}

    \section*{亲核试剂和亲电试剂}
    
    \subsection*{亲电试剂}

    \begin{enumerate}
        \item 单质 $\ce{X2}$
        \item 强酸 $\ce{HCl}$
        \item 缺电子化物 
    \end{enumerate}

    \subsection*{亲核试剂}

    \begin{enumerate}
        \item 弱酸、弱酸眼 $\ce{HCN}$,$\ce{NaCN}$
        \item 含有未成键电子对化合物 $\ce{NH3}$
        \item 格氏试剂 $\ce{RMgX}$
    \end{enumerate}
    \chapter{酚、醌}
    \section{概述}
    酚(phenols, $\ce{Ar-OH}$)
    羟基直接与苯环\textit{$sp^2$}碳原子相连接的化合物称为酚,苯酚(\scalebox{0.6}[0.6]{\chemfig{OH-[:180,,1]=_[:240]-[:180]=_[:120]-[:60]=_(-[:300])}})最为常见,另外还有萘酚等。
    
    \subsection{分类}
    \begin{enumerate}
        \item 芳环类型
        \item $\ce{-OH}$的个数
    \end{enumerate}
    \subsection{命名}
    不考,没啥用,知道大概啥顺序就行

    \section{物理性质}
    \begin{enumerate}
        \item 形成分子间氢键,沸点较高
        \item 对称性较好,熔点较高,常温下是固体
        \item 水溶性 苯酚在冷水中微溶,加热时可以无限溶解 \par
        与醇相比,苯酚和水形成氢键的能力比较差。($sp^3$和$sp^2$)\footnote{$sp^3$杂化距离更远,跟适合形成氢键。酚只有一对$sp^2$电子,比醇还少一对}
        \item 酚本身无色,容易被空气中的氧气氧化而略呈红色或褐色
        \item 有毒
    \end{enumerate}

    \section{化学性质}
    \subsection{$\ce{C-O}$ 键}
    不容易断。酚具有一定的离域,具有一定的双键性质。

    \subsection{$\ce{O-H}$ 键}
    \begin{enumerate}
        \item 酸性
        \item 含有未成对的电子---碱性、亲核性
        \item 配位性,可以和铁发生显色反应
    \end{enumerate}

    \subsection{苯环}
    \begin{enumerate}
        \item 亲电取代,$\ce{-OH}$是邻对位定位基,可以使苯环活化
        \item 氧化(复杂,机理不甚明确)
    \end{enumerate}
\end{document}