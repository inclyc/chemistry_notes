\documentclass[a4paper]{ctexrep}
\usepackage{mathtools}
\usepackage{tikz}
\usepackage{chemfig}
\usepackage[version=4]{mhchem}
\usepackage{float}


\author{Y.C. Long}
\title{有机化学 -- 孙净雪}


\begin{document}
    \maketitle
    \tableofcontents
    

    % 孙净雪 - 13384608260
    % 期末闭卷70分,出勤10分,课堂小测验10分,作业10分。作业交了就行。
    % 老师是两批老师,所以这个课会用一点时间过渡一下
    \chapter*{学期过渡部分}
    \section*{基本反应回顾}
    \begin{itemize}
        \item 烷烃--自由基引发,取代
        \item 烯烃--加成,亲电,$\pi$电子云是裸露在外的
        \item P--X 卤代烃 没有不饱和度,被亲核试剂进攻
        \item 芳烃 亲电试剂进攻
    \end{itemize}

    \section*{亲核试剂和亲电试剂}
    
    \subsection*{亲电试剂}

    \begin{enumerate}
        \item 单质 $\ce{X2}$
        \item 强酸 $\ce{HCl}$
        \item 缺电子化物 
    \end{enumerate}

    \subsection*{亲核试剂}

    \begin{enumerate}
        \item 弱酸、弱酸眼 $\ce{HCN}$,$\ce{NaCN}$
        \item 含有未成键电子对化合物 $\ce{NH3}$
        \item 格氏试剂 $\ce{RMgX}$
    \end{enumerate}
    \chapter{酚}
    \section{概述}
    酚(phenols, $\ce{Ar-OH}$)
    羟基直接与苯环\textit{$sp^2$}碳原子相连接的化合物称为酚,苯酚(\scalebox{0.6}[0.6]{\chemfig{OH-[:180,,1]=_[:240]-[:180]=_[:120]-[:60]=_(-[:300])}})最为常见,另外还有萘酚等。
    
    \subsection{分类}
    \begin{enumerate}
        \item 芳环类型
        \item $\ce{-OH}$的个数
    \end{enumerate}
    \subsection{命名}
    不考,没啥用,知道大概啥顺序就行

    \section{物理性质}
    \begin{enumerate}
        \item 形成分子间氢键,沸点较高
        \item 对称性较好,熔点较高,常温下是固体
        \item 水溶性 苯酚在冷水中微溶,加热时可以无限溶解 \par
        与醇相比,苯酚和水形成氢键的能力比较差。($sp^3$和$sp^2$)\footnote{$sp^3$杂化距离更远,跟适合形成氢键。酚只有一对$sp^2$电子,比醇还少一对}
        \item 酚本身无色,容易被空气中的氧气氧化而略呈红色或褐色
        \item 有毒
    \end{enumerate}

    \section{化学性质}
    \subsection{$\ce{C-O}$ 键}
    不容易断。酚具有一定的离域,具有一定的双键性质。

    \subsection{$\ce{O-H}$ 键}
    \begin{enumerate}
        \item 酸性
        \item 含有未成对的电子---碱性、亲核性
        \item 配位性,可以和铁发生显色反应
    \end{enumerate}

    \subsection{苯环}
    \begin{enumerate}
        \item 亲电取代,$\ce{-OH}$是邻对位定位基,可以使苯环活化
        \item 氧化(复杂,机理不甚明确)
    \end{enumerate}

    \subsection{酚的酸性}

    \subsubsection{不同类}
    $\ce{H2CO3} > \ce{Ph-OH} > \ce{H2O} > \ce{ROH}$

    若芳香基团上有吸电子基团,则酚酸性增强

    \subsubsection{同类}

    \begin{enumerate}
        \item 吸电子基 酸性增加 \par
        \begin{figure}[H]
            \scriptsize
            \centering
            \chemfig{*6(=-=-(-OH)=-)} < \chemfig{*6(=-(-NO_2)=-(-OH)=-)} < \chemfig{*6(=(-NO_2)-=-(-OH)=-)}
        \end{figure}
        \item 供电子基 酸性降低 \par
        \begin{figure}[H]   
            \scriptsize
            \centering
            \chemfig{*6(=-=-(-OH)=-)} < \chemfig{*6(=-(-CH_3)=-(-OH)=-)} < \chemfig{*6(=(-CH_3)-=-(-OH)=-)}
        \end{figure}
        \item 卤素 吸电子,酸性增加
    \end{enumerate}

    \subsection{显色反应}
    可以与铁形成紫色复合物

    $\ce{6C6H6OH + FeCl3 -> H_3[Fe(OC6H5)] + 3HCl}$

    \subsection{亲核性}

    亲核性 只和易发生亲核反应的物质反应

    如 
    \begin{figure}[H]
        \scriptsize
        \centering
        \schemestart
        \chemfig{R-C(=[:90]O)-Cl} \+ \chemfig{*6(-=-=(-OH)-=)} \arrow
        \schemestop
    \end{figure}

    \subsection{亲电取代}

    \subsubsection{卤代反应}

    \begin{figure}[H]
        \scriptsize
        \centering
        \schemestart
        \chemfig{*6(-=-=(-OH)-=)}  \+ $\ce{Br2}$ \arrow \chemfig{*6(-(-Br)=-(-Br)=(-OH)-(-Br)=)}
        \schemestop
    \end{figure}
    原因:水中存在 \scriptsize\chemfig{*6(-=-=(-{O^{-}})-=)}\normalsize,要想控制反应进度,可以在有机溶剂中进行:

    \begin{figure}[H]
        \scriptsize
        \centering
        \schemestart
        \chemfig{*6(-=-=(-OH)-=)} \arrow{->[$\ce{Br}$][$\ce{CS2}, 0^\circ \mathrm{C}$]} \chemfig{*6(-(-Br)=-=(-OH)-=)} + \chemfig{*6(-=-(-Br)=(-OH)-=)}
        \schemestop
    \end{figure}

    \subsubsection{磺化反应}

    \begin{figure}[H]
        \scriptsize
        \centering
        \schemestart
        \chemfig{*6(-=-=(-OH)-=)} \+ $\ce{H2SO4}$ \arrow \chemfig{*6(-=-(-SO_3)=(-OH)-=)} \+ \chemfig{*6(-(-SO_3)=-=(-OH)-=)}
        \schemestop
    \end{figure}
    \subsubsection{硝化反应}

    一般来讲不要直接硝化,不然产率会很低

    \begin{figure}[H]
        \scriptsize
        \centering
        \schemestart
        \chemfig{*6(-=-=(-OH)-=)} \arrow{->[$\ce{NaNO2}$][$\ce{H+}$]} \chemfig{*6(-(-NO)=-=(-OH)-=)} \arrow{->[$\ce{HNO3}$]} \chemfig{*6(-(-NO_2)=-=(-OH)-=)}
        \schemestop
    \end{figure}

    \subsubsection{傅氏反应}

    \begin{figure}[H]
        \scriptsize
        \centering
        \schemestart
        \chemfig{*6(-=-=(-OH)-=)} \+ $\ce{RCl}$ \arrow{->[HF]} \chemfig{*6(-(-R)=-=(-OH)-=)}
        \schemestop
    \end{figure}

    \subsubsection{氧化反应}

    \begin{figure}[H]
        \scriptsize
        \centering
        \schemestart
        \chemfig{*6(-=-=-=)} \arrow{->[$\mbox{[O]}$]} \chemfig{*6(-(=O)-=-(=O)-=)}
        \schemestop
    \end{figure}

    \subsection{酚的制法(了解)}


    \chapter{醛、酮}

    \section{概述}

    \subsection{分类}

    \begin{enumerate}
        \item 脂肪型
        \item 芳香型
    \end{enumerate}

    \subsection{羰基的结构特点 -- 极性不饱和基团}

    碳和氧都采用$sp^2$杂化,碳氧双键中,成键电子云分布不均匀,而是偏向氧原子。

    \section{物理性质}

    极性分子之间的偶极-偶极相互作用,醛酮的沸点比相对分子量相近的烷烃和醚高,低级醛通常易溶于水,高级醛通常微溶或不溶于水

    \section{化学性质}
    
    \subsection{$\alpha-\ce{H}$键}
    \scriptsize \chemfig{-C(-[:90])(-[:-90]H)-C=O} \normalsize 由于双键的氧加强了$\alpha-\ce{H}$上的碳氢键的极性。醛酮可以在$\alpha-\ce{H}$ 异裂后产生阴离子,可以发生缩合反应和异裂取代反应。

    \subsection{醛基的氧化反应}

    解释醛基的氧化还原性质可以通过与$\alpha-\ce{H}$的比较来进行。

    \subsection{亲核加成}

    \subsubsection{与弱酸的加成}

    与HCN,得到重要的合成材料$\alpha-$羟基腈

    \begin{figure}[H]
        \scriptsize
        \centering
        \schemestart
        \chemfig{C(-[:-135])(-[:135])=O} \+ HCN \arrow \chemfig{-C(-[:90]OH)(-[:-90])-CN} 
        \schemestop
    \end{figure}

    有机玻璃的制备:

    \begin{figure}[H]
        \scriptsize
        \centering
        \schemestart
        \chemfig{C(-[:135]H_3C)(-[:-135]H_3C)=O} \+ HCN \arrow{->[NaOH]} \chemfig{C(-[:135]H_3C)(-[:-135]H_3C)(-[:45]CN)(-[:135]H_3C)(-[:-45]OH)}
        \schemestop
    \end{figure}

    
    \subsubsection{与$\ce{ROH}$的加成}


    \begin{figure}[H]
        \scriptsize
        \centering
        \schemestart
        \chemfig{C(=[:90]O)(-[:-150]H_3C)(-[:-30]CH_3)} \arrow{<=>} \chemfig{C(=[:90]\charge{90:1pt=+}{O}H)(-[:-150]H_3C)(-[:-30]CH_3)}m \+ \chemfig{CH_3OH} \arrow{<=>} \chemfig{CH_3-C(-[:90]OH)(-[:-90]OCH_3(-[:-90]H))-CH_3}
        \schemestop
    \end{figure}

    \paragraph{活性}

    $\ce{H+}$催化,体积效应比较大,醛、环酮的活性高

    \paragraph{二元醇}

    \begin{center}
        \scriptsize
        \schemestart
        \chemfig{[:-30]*6(---(=O)---)} \+ \chemfig{CH_2(-[:-90]CH_2-[:0]OH)-OH} \arrow{<=>[$\ce{H+}$]} \chemfig{[:-30]*6(---(-[:45]O-[:0]C?H_2)(-[:-45]O-[:0]C?H_2)---)}
        \schemestop
    \end{center}

    
    \paragraph{应用}

    \begin{enumerate}
        \item 缩醛或缩酮在碱性条件、氧化条件下是稳定的。
        \item 在酸性条件下可逆,可以变回原来的醛或者酮。
    \end{enumerate}

    可以用于保护羰基$\ce{C=O}$官能团

    \subsubsection{与$\ce{NaHSO3}$的反应}

    醛、甲基酮可以与亚硫酸氢钠饱和溶液发生加成反应,生成结晶的$\alpha-$羟基磺酸钠

    \begin{figure}[H]
        \scriptsize
        \centering
        \schemestart
        \chemfig{C(-[:120]R)(-[:-120]H_3C)=O} \+ \chemfig{\charge{90=\:}{S}(=[:-90]O)(-[:150]OH)-[:30]ONa} \arrow{<=>} \chemfig{C(-[:135]R)(-[:-135]H_3C)(-[:45]ONa)(-[:-45]SO_3H)} \arrow{<=>} \chemfig{C(-[:135]R)(-[:-135]H_3C)(-[:45]OH)(-[:-45]SO_3Na)}
        \schemestop
    \end{figure}


    \paragraph{活性} 体积反应,醛、环酮、甲基酮反应

    \paragraph{应用} 现象:鉴别,可逆:制备提纯


    \subsubsection{与含氮化合物及其衍生物}

    \begin{center}
        \scriptsize
        \schemestart
        \chemfig{C(-[:135])(-[:-135])=O} \+ \chemfig{NH_2Y} \arrow{<=>} \chemfig{-[,0.5]C(-[:90]OH)(-[:-90,0.5])-NHY} \arrow{->[-$\ce{H2O}$]} \chemfig{C(-[:135])(-[:-135])=NY} 
        \schemestop
    \end{center}

    \paragraph{历程} 先加成,后消除,pH=4~5 催化。不用强酸催化的原因是:如果酸性过强,会导致生成铵盐,失去电子,不反应。

    \paragraph{应用} 用于鉴定醛和酮。

    \begin{center}
        \scriptsize
        \schemestart
        \chemfig{*6((-O_2N)-=(-NO_2)-(-NHNH_2)=-=)} \arrow{->[醛、酮]} \chemfig{*6((-O_2N)-=(-NO_2)-(-NHN=[:0]C(-[:-90]R)-H)=-=)}
        \schemestop
    \end{center}

    醛、酮的分离、纯化。结晶多为晶形沉淀。加成产物可以在稀酸条件下可以水解成原来的醛或者酮。


    \subsubsection{与金属有机化合物$\ce{RMgX}$}

    \begin{center}
        \scriptsize
        \schemestart
        \chemfig{C(-[:135])(-[:-135])=O} + RMgX \arrow{<=>} \chemfig{-[,0.5]C(-[:90]OMgX)(-[:-90,0.5])-R} \arrow{->[$\ce{H+}$][$\ce{H2O}$]}  \chemfig{-[,0.5]C(-[:90]OH)(-[:-90,0.5])-R}
        \schemestop
    \end{center}

    \paragraph{应用} 制醇

    \subsubsection{与金属邮寄化合物$\ce{RLi}$}

    烷基锂的亲核性极强,而且几乎不受到体积效应的影响。要求反应必须绝对完全无水。

    \subsection{还原反应}

    \subsubsection{催化加氢}

    \begin{center}
        \scriptsize
        \schemestart
        \chemfig{C(-[:135])(-[:-135])=O} \arrow{->[$\ce{Ni}$][$\ce{H2}$]} \chemfig{-[,0.5]C(-[:90]H)(-[:0])-OH}
        \schemestop
    \end{center}

    \subsubsection{$\ce{NaBH4}$}
    \begin{center}
        \scriptsize
        \schemestart
        \chemfig{C(-[:135])(-[:-135])=O} \+ $\ce{H-}$ \arrow{->} \chemfig{-C(-[:-90]H)-\charge{80:2pt=-}{O}} \arrow{->[$\ce{H+}$]} \chemfig{-C(-[:-90]H)-OH}
        \schemestop
    \end{center}

    \begin{center}
        \scriptsize
        \schemestart
        \chemfig{[:-30]*6(-=-(=O)---)} \arrow{->} \chemfig{[:-30]*6(-=-(-OH)---)}
        \schemestop
    \end{center}

    对\chemfig{C=[,0.5]C},\chemfig{C~[,0.5]C}不反应,可以有选择性地加成。

    
\end{document}