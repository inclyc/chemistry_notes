\documentclass[a4paper]{article}

\usepackage{physics}
\usepackage{booktabs}
\usepackage{mathtools}
\usepackage{amsfonts}

\renewcommand{\d}{\mathrm{d}}

\begin{document}
    \section{Linearity of quantum mechanics}
    \subsection{Basic theory elements}
    In Maxwells' theory:
    \begin{equation*}
        \left( \vec{E}, \vec{B}, \rho, \vec{J} \right)
    \end{equation*}

    Also a solution:
    
    \begin{equation*}
        \left( \alpha\vec{E}, \alpha\vec{B}, \alpha\rho, \alpha\vec{J} \right)
    \end{equation*}

    the linear combination of two verified solutions is also a solution:
    
    \subsection{Linear operator}

    \begin{equation*}
        L u \equiv 0
    \end{equation*}

    $u$: unknown vector, $L$: linear operator, $0$: zero vector.


    \begin{equation*}
        L(au) = aLu
    \end{equation*}

    \begin{equation*}
        L(u_1 + u_2) = L(u_1) + L(u_2)
    \end{equation*}

    \begin{equation*}
        L(\alpha u_1 + \beta u_2) = \alpha L(u_1) + \beta L(u_2)
    \end{equation*}


    If $u_1$, $u_2$ are solutions, then $\alpha u_1 + \alpha u_2$ is a solution.
    

    \subsubsection{Example}

    \begin{equation*}
        \frac{\d u}{\d t} + \frac{1}{\tau} = 0
    \end{equation*}
    
    \begin{equation*}
        L = \left( \frac{\d }{\d t} + \frac{1}{\tau} \right)
    \end{equation*}
    
    \subsection{Is classic mechanics linear?}

    No. Consider 1D mechanics Assume that the potential $V(x)$, and $x(T)$, the position is our unknown variable.

    \begin{equation*}
        m \frac{\d ^ 2 x(t)}{\d t^2} = - V' \left( x(t) \right)
    \end{equation*}

    The potential function may not be linear! It may be a function of time, or of position, or both.
    
    \subsection{QM is linear!}

    \begin{equation*}
        \psi \qquad \text{Wave function}
    \end{equation*}

    Schrödinger's equation:

    \begin{equation*}
        i \hbar \frac{\partial \psi}{\partial t} = \hat{H} \psi
    \end{equation*}
    

    $\hat{H}$ is the Hamiltonian, which is the potential function, and the kinetic energy, its a linear operator! This equation is a linear equation!
    
    Here $L$ is: 

    \begin{equation*}
        L =  i \hbar \frac{\partial}{\partial t} - \hat{H}
    \end{equation*}

    \section{Necessity of complex numbers}
    
    \begin{equation*}
        i = \sqrt{i}
    \end{equation*}

    \begin{equation*}
        \psi \in  \mathbb{C} 
    \end{equation*}

    \section{Loss of determinism}

    The light was made of quanta made of photons. In some way, photons are also waves. For photons, the energy is given by:

    \begin{equation*}
        E = h \nu
    \end{equation*}
    
    where $h$ is the Planck constant, and $\nu$ is the frequency. The photons can go through the polarizer and it can only predict probabilities only. 

    We can use dirac's notation and imaging that this a vector. We say that this is a vector because QM is a linear theory that you can scale wave functions and add them together just like vectors.
    
    \begin{equation*}
        \ket{\mbox{photon}, x}, \ket{\mbox{photon}, y} 
    \end{equation*}
    

    \begin{equation*}
        \ket{\mbox{photon}, \alpha} = \cos \alpha \ket{\mbox{photon}, x} + \sin \alpha \ket{\mbox{photon}, y} 
    \end{equation*}
   
    
    
    \section{Supper position in unusual}
    \section{Entanglement}
\end{document}