\documentclass[a4paper]{ctexrep}

\usepackage{tikz}
\usepackage[version=4]{mhchem}
\usepackage{graphicx}
\usepackage{float}
\usepackage{xcolor}
\usepackage{mathtools}

\newcommand{\mol}{\mathrm{mol}}
\renewcommand{\d}{\mathrm{d}}

\author{Y.C. Long}
\title{物理化学 -- 刘婧媛}

\begin{document}
    % 作业 20%
    % 课程报告 10%
    % 期末考试 70%
    \maketitle
    \tableofcontents

    \chapter{电解质溶液}
        \section{前言}
        \subsection{电化学综述}
        内容:$\mbox{化学反应} \leftrightarrow \mbox{电现象}$。既包括热力学问题,也包括动力学问题。电化学的用途主要包括研究化学能和电能的互相转换:
        \[
            \mbox{化学能} \ce{<=>T[\mbox{电池}][\mbox{电解池}]} \mbox{电能}  
        \]

        电化学系统就是电池或电解池,由导体或半导体组成,多相存在,为科学研究以及生产过程提供精确快速地研究测定方法。
        \subsection{电解质溶液}的
        电解质溶液是电池和电解池的重要组成部分。
        \subsubsection{与非电解质溶液的区别}
        \paragraph{导电} 物理动力学 

        \paragraph{热力学} 高度不理想性
        \subsubsection{电解质溶液的特点}
        电解质溶液的概念较多,要以理解为主。

        \section{电化学中的基本概念}

        \subsection{电解质溶液的导电机理(The mechanism of conduction for electrolyte solution)}
        金属(第一类导体)和电解质溶液(第二类导体)的导电机理不同。例如,电解$\ce{CuCl2}$溶液,首先有离子电迁移的物理变化,后有电极反应的化学变化。

        $\ce{Pt}$电极:$\ce{2Cl- -> Cl2 + 2e-}$ (氧化)

        $\ce{Cu}$电极:$\ce{Cu2+ + 2e- -> Cu}$(还原)

        总结果: $\ce{\mbox{电池(-)} ->T[e-] Cu ->T[e-][re] sln ->T[e-][ox] Pt ->T[e-] \mbox{电池(+)}}$

        \subsubsection{电极命名法}
        \paragraph{按电位高低} 电位高 -- 征集, 电位低 -- 负极

        \paragraph{按反应性质} 氧化 -- 阳极,还原 -- 阴极

        \section{Faraday电解定律}
        Faraday 归纳了多次实验结果,于1833年总结出了电解定律:
        \begin{enumerate}
            \item 电解界面上发生的化学变化物质的量与通入的电荷量成正比
            \item 若将几个电解池相连,通入等量的电荷,他们发生化学反应的物质的量相等
        \end{enumerate}

        \subsection{Faraday 常数}
        人们把在数值上等于1 mol元电荷的电量称为Faraday常数。Faraday常数需要记忆:$F \approx 96500 \mathrm{C} \cdot \mbox{mol}^{-1}$

        如果在电解池发生如下反应:
        \[ 
            \ce{M^{z+} + z_{+}e- -> M(s)}
        \]
        电子得失的化学计量数为$z_{+}e-$,则需要的电荷量为:$\ce{z_{+}e-} \cdot \mathrm{F}$

        \subsection{荷电粒子基本单位的选取}

        $n = \frac{N}{L}$,其中$N$为基本单元的个数,所以$n$值与基本单元有关,例如18g水,可以表示为:
        $n \ce{H2O} = 1 \mol$,$n(\ce{2H2O}) = 0.5 \mol$。
        在研究电解质溶液导电性质时,习惯用一个元电荷($\ce{e-}$)为基础指定物质量的基本单元。

        \paragraph{基本阴阳离子的基本单位}

        离子$\ce{M^{z+}}$ 用$n \left( \frac{1}{z_{+}} M^{z+}\right)$描述离子的物质的量。
        
        例 $n(\ce{H+})$,$n(\frac{1}{2}\ce{Cu^{2+}})$,$n(\frac{1}{3}Fe^{3+})$。

        这样做的好处是,$1 \mol$任何离子所带的电量均为$6.023 \times 10^{23} e$。

        \paragraph{电解质的基本单位}
        % TODO 这个地方老师的PPT好像有点问题
        电解质$M_{v+}A_{v-}$,用$n \left( \frac{1}{v_+ v_-} M_{v+}A_{v-} \right)$

        \paragraph{参与氧化还原反应的物质M}

        $\ce{M + ze- ->T[re] M^{z-}}$,以$n \left( \frac{1}{z}M\right)$ 来描述物质M的物质的量

        \section{离子的电迁移率和迁移数}

        \subsection{离子的电迁移率}
        
        \[
            \mbox{电解质} \ce{ ->T[电离] } \mbox{离子B} + \mbox{离子D}  
        \]

        \subsubsection{定义}

        通常讨论在一定$T$,$p$下的某一指定溶液,则离子迁移速率\[r_B = u_B\frac{\d E}{\d l}\]

        \subsubsection{淌度的测量}

        根据定义,设法测定$r_{B}$和$E$

        \subsubsection{离子的极限电迁移率}


        \subsection{离子迁移数}

        电解质溶液导电时离子的具体迁移情况(Hittorf模型),把电解池分为阳极区,中间区,阴极区。离子B的迁移数是离子B运载的电荷量和总电流之比。

        \[
            t_B \overset{\mathrm{def}}{=} \frac{I_B}{I}
        \]

        \subsubsection{结论}

        \paragraph{浓度变化} 中间区的浓度不变,两极区的浓度改变。
        \paragraph{物质的量关系} $n_B(\mbox{迁移}) \neq n_B(\mbox{电极反应})$
        \paragraph{溶液的导电任务由正负离子共同分担}
        \[
            \frac{t_{+}}{t_{-}} = \frac{u_{+}}{u_{-}}  
        \]


    \chapter{可逆电池的电动势及其应用}

    \chapter{电解和极化作用}

    \chapter{化学反应动力学}

    \chapter{表面物理化学}

    \chapter{胶体化学}
\end{document}