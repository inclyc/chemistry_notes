\section{几种典型的复杂反应}

\subsection{对峙反应}

在正反两个方向都能进行的反应称为对峙反应,称为可逆反应。

\begin{equation*}
    \ce{A <->[k_1][k_{-1}] B}
\end{equation*}

解方程可以得到:

\begin{equation*}
    k_1 ta = x_e \ln \left(\frac{a - x_e}{x_e} \right)
\end{equation*}

\subsection{平行反应}

\begin{equation*}
    \ce{A -> } \begin{cases}
        \ce{->[k_1] B} \\
        \ce{->[k_2] C} \\
    \end{cases}
\end{equation*}

\begin{equation*}
    k = k_1 + k_2
\end{equation*}

\subsection{连续反应}

\begin{equation*}
    \ce{A ->[k_1] B ->[k_2] C}
\end{equation*}



\subsection{链反应}

\subsubsection{链反应的基本过程}

\paragraph{链引发}

\paragraph{链传递}

\paragraph{链终结}

