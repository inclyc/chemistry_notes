\section{化学动力学的任务和目的}

\subsection{热力学的作用和没有解决的问题}

热力学回答了反应能量转换、反应的方向、限度,和平衡性质。然而,我们更多时候希望了解:

\begin{itemize}
    \item 反应的快慢 -- 反应速率
    \item 化学反应的机理
\end{itemize}

化学反应速率的表示法

\[
    R \rightarrow P
\]

反应速率会随着时间的变化而不断变化。反应刚开始的时候速度大,然后速率不断减小,体现了反应速率变化的真实情况。反应速度和反应速率之 间通常差一个正负号。速度是``有方向的'',速率是速率的绝对值,没有方向。反应速率定义为

\[
    r = \frac{1}{V} \frac{d \xi}{dt}
\]

对于气相反应,气体的压力更容易测定,因此也可以用压力作为反应速率。
