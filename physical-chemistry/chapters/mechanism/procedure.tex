\section{化学反应的结构和过程}

\subsection{基元反应}

基元反应是一种特殊的反应,它的反应过程只在一步发生。基元反应拥有固定的速率方程性质:

\begin{equation*}
    \ce{aA + bB -> P}
\end{equation*}

\begin{equation*}
    r = k[A]^a[B]^b
\end{equation*}

\subsection{具有简单级数的方程}

\subsubsection{一级反应}

凡是反应速率只与物质浓度的一次方成正比者称为一级反应。

\[
    \ce{A \rightarrow B}
\]

\[
    \frac{ \d x} {(a - x)} = k_1 dt
\]

有

\begin{equation*}
    \ln \frac{a}{a - x} = k_1 t
\end{equation*}

\begin{equation*}
    (a - x) = a e^{-k_1 t}
\end{equation*}


\paragraph{半衰期} 反应的半衰期为 $\frac{\ln 2}{k_1}$

\subsubsection{二级反应}

反应速率和物质浓度的二次方成正比者,称为二级反应。

\begin{equation*}
    \ce{A + B ->[k_1] C}
\end{equation*}

\begin{equation*}
    \frac{\d x}{\d t} = k_2 (a - x) (b - x)
\end{equation*}

\begin{equation*}
    \int \frac{\d x}{(a - x)(b - x)} = \int k_2 \d t
\end{equation*}

\begin{equation*}
    \frac{1}{a - b} \ln \frac{a - x}{b - x} = k_2 t + C
\end{equation*}

\begin{equation*}
    k_2 = \frac{1}{t(a - b)} \ln \left[ \frac{b(a - x)}{a(b - x)} \right]
\end{equation*}

速率常数用浓度表示或用压力表示,两者的数值不相等。假设有某$n$级反应,
\begin{equation*}
    k_c = k_p (RT)^{1 - n}
\end{equation*}
\subsubsection{零级反应}
反应物质的浓度不影响反应速率。常见的零级反应有表面催化反应和酶催化反应。这时反应物总是过量的,反应速率取决于固体催化剂的有效表面活性或酶的浓度。

\begin{equation*}
    r = \frac{\d x}{\d t} = k_0
\end{equation*}

\subsubsection{准级反应}
在速率方程中,若某一物质的浓度远远大于其他反应无的浓度,或是在速率方程中的催化剂浓度项,在反应过程中可以认为没有变化,可以并入速率常数项,这时反应总级数可以相应下降,称为准级反应。

\subsection{测定反应的级数}


\subsubsection{积分法}

先计算$k$值,再根据假设的$k$值进行作图。

\subsubsection{微分法}

\begin{equation*}
    \ln r = \ln \left( - \frac{\d [A]}{\d t} \right) = \ln k + n \ln [A]
\end{equation*}

\subsubsection{半衰期法}

这个方法比较简单,但只适合反应物只有一个,或按照化学计量数投料。

\begin{equation*}
    t_{\frac{1}{2}} = A \cdot a^{1 - n}
\end{equation*}