\section{温度对化学反应速率的影响}


\subsection{Arrhenius 经验式}
\subsubsection{van't Hoff}

温度每升高10K,反应速率大约增加2到4倍。

\begin{equation*}
    \frac{k_{T + 10 \mathrm{K}}}{k_T} = 2 \sim 4
\end{equation*}


\subsubsection{Arrhenius}

Arrhenius 研究了许多气相反应的速率,特别是对蔗糖在水溶液中的转化做了大量的研究工作。他提出了活化能的概念,揭示了反应速率常数与温度的依赖关系,即:

\begin{equation*}
    k = Ae^{-\frac{E_a}{RT}}
\end{equation*}

如果假设$A$与$T$无关,则

\begin{equation*}
    \frac{\d \ln k}{\d T} = \frac{E_a}{RT^2}
\end{equation*}