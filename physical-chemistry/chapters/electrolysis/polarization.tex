

\section{极化作用}


不可逆电极过程中$\phi$偏离平衡位置的现象。

产生极化的原因:当$I \neq 0$时,电极上发生一系列以一定速率进行的过程。每个过程均有阻力,克服阻力需要推动力,就表现为$\phi$的偏离行为。电池极化分为:

\subsection{浓差极化}

由于浓差扩散过程中存在阻力,使得电极附近的溶液与浓度的本体不同,从而使$\phi$值与平衡值产生一定的偏离。

\subsection{电化学极化}


电极反应的某一步反应速率比较慢,需要比较高的活化能。


极化作用结果总是阴极电极电势下降,阳极电极电势上升。
