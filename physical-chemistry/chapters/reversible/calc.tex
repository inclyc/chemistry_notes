
    \section{电动势的计算}

    \subsection{直接加和}

    将电池中各种界面的电势差加起来,为电动势。

    \subsection{通过Nernst方程计算}

    \begin{reaction*}
        Pt | H2(p1) | HCl(a) | Cl2(p2) | Pt
    \end{reaction*}

    \begin{equation*}
        \Delta_r G_m = \Delta_r G_m ^\ominus + RT\ln \prod _{B} a_B^{\nu_B}
    \end{equation*}

    根据$\Delta_r G_m = -zFE$

    \begin{equation*}
        E = E^\ominus - \frac{RT}{zF} \ln \prod _{B} a_B^{\nu_B}
    \end{equation*}

    这里的$z$取多少并不会影响最后电动势的大小,整体变化$z$的话,系数会在对数部分得到还原。

    \begin{reaction*}
        Pt | H2( $p$ ) | H2SO4 | O2(p) | Pt
    \end{reaction*}


    \begin{reactions*}
        H2 - 2 e- &-> 2 H+ \\
        1/2 O2 + 2 e- + 2 H+ &-> H2O
    \end{reactions*}

    \subsection{电极电势}

    认为氢标准电极的电势为0,其他电极的电势是相对的电势。电极电势为负数时表示容易进行氧化反应,电极电势为正数时,容易发生还原反应。

    $\phi^{\ominus}$ 标准电极电势,在标准状态下的材料制备时的$\phi$。$\phi$是与氢标准电极形成原电池下的$E$。$\phi$的计算与$E$相同。

    \begin{equation*}
        \Delta_r G_m = -zF\phi
    \end{equation*}

    \begin{equation*}
        \Delta_r G_m ^\ominus = -zF\phi ^\ominus
    \end{equation*}

    \begin{reaction*}
        Zn^{2+} (a1) + 2 e- -> Zn
    \end{reaction*}

    \begin{equation*}
        \phi_{ \left( \ch{Zn^{2+} | Zn} \right) } = \phi^\ominus - \frac{RT}{zF} \ln \frac{1}{a1}
    \end{equation*}


    计算电动势的例子
    \begin{equation*}
        \ch{H2}\left( p = 90 \mathrm{kPa} \right) | \ch{H\pch} \left( a_{\ch{H\pch}} = 0.01 \right)  \| \ch{Cu\pch[2]} \left( a_{\ch{Cu\pch[2]}} = 0.10 \right)| \ch{Cu}
    \end{equation*}

    \begin{equation*}
        p^\ominus = 100 \mathrm{kPa} \qquad \phi(\ch{Cu\pch[2]}, \ch{Cu}) = 0.337 \mathrm{V}
    \end{equation*}

    $a_{\ch{H2}}$ 为 

    \begin{equation*}
        \frac{p}{p^\ominus} = 0.9
    \end{equation*}

    \begin{equation*}
        E = \phi(\ch{Cu\pch[2]}, \ch{Cu}) - \frac{RT}{zF} \cdot \ln \frac{a_{\ch{H\pch}}^2}{a_{\ch{Cu\pch[2]} }a_{\ch{H2}}}
    \end{equation*}

    \begin{align*}
        R &= 8.314 \\
        T &= 298 \\ 
        F &= 96500 \\ 
    \end{align*}

    代入上式

    \begin{align*}
        E &= \phi(\ch{Cu\pch[2]}, \ch{Cu}) - \frac{RT}{zF} \cdot \ln \frac{a_{\ch{H\pch}}^2}{a_{\ch{Cu\pch[2]} }a_{\ch{H2}}} \\ 
        &= 0.337 - 0.0256 \cdot \ln \frac{0.01^2}{0.9 * 0.1} \\ 
        &= 0.424 \quad (\mathrm{V} )
    \end{align*}
