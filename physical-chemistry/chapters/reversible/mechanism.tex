\section{电动势产生的机理}

    \subsection{电极和溶液之间的界面电势差的定义}

    由于水合作用,电极和溶液之间会产生电位差。

    \subsection{接触电势}

    电子逸出功,电子从金属表面逸出时,为了克服表面势垒所做的功。相互接触时,逸出功的大小不一致,导致逸出功


    \subsection{液体接界电势}

    在含有不同溶液的形成的界面上,或同一种溶液不同浓度所形成的的界面上,由于阴阳离子迁移的速度不一致,会导致界面形成微小的电动势。这一电动势被称为液接电势。

    \subsubsection{盐桥的作用}

    液接电势对电动势产生干扰,可以用\textbf{盐桥}来抵消液接电势。


    \section{电动势的计算}

    \subsection{直接加和}

    将电池中各种界面的电势差加起来,为电动势。

    \subsection{通过Nernst方程计算}

    \begin{reaction*}
        Pt | H2(p1) | HCl(a) | Cl2(p2) | Pt
    \end{reaction*}

    \begin{equation*}
        \Delta_r G_m = \Delta_r G_m ^\ominus + RT\ln \prod _{B} a_B^{\nu_B}
    \end{equation*}

    根据$\Delta_r G_m = -zFE$

    \begin{equation*}
        E = E^\ominus - \frac{RT}{zF} \ln \prod _{B} a_B^{\nu_B}
    \end{equation*}

    这里的$z$取多少并不会影响最后电动势的大小,整体变化$z$的话,系数会在对数部分得到还原。

    \begin{reaction*}
        Pt | H2( $p$ ) | H2SO4 | O2(p) | Pt
    \end{reaction*}


    \begin{reactions*}
        H2 - 2 e- &-> 2 H+ \\
        1/2 O2 + 2 e- + 2 H+ &-> H2O
    \end{reactions*}

    \subsection{电极电势}

    认为氢标准电极的电势为0,其他电极的电势是相对的电势。电极电势为负数时表示容易进行氧化反应,电极电势为正数时,容易发生还原反应。

    $\phi^{\ominus}$ 标准电极电势,在标准状态下的材料制备时的$\phi$。$\phi$是与氢标准电极形成原电池下的$E$。$\phi$的计算与$E$相同。

    \begin{equation*}
        \Delta_r G_m = -zF\phi
    \end{equation*}

    \begin{equation*}
        \Delta_r G_m ^\ominus = -zF\phi ^\ominus
    \end{equation*}

    \begin{reaction*}
        Zn^{2+} (a1) + 2 e- -> Zn
    \end{reaction*}

    \begin{equation*}
        \phi_{ \left( \ch{Zn^{2+} | Zn} \right) } = \phi^\ominus - \frac{RT}{zF} \ln \frac{1}{a1}
    \end{equation*}