\section{利用电动势计算难求的其他物理量}

\begin{equation*}
    \Delta G = -z EF
\end{equation*}

\begin{equation*}
    K^\ominus = \exp \left( \frac{zEF}{RT}  \right)
\end{equation*}

可利用电化学的方法求 $K^\ominus_{sp}$,$K^\ominus_w$

\subsection{求$K^\ominus_{sp}$}

设计电池,使电池反应为

\begin{reaction*}
    AgCl(s) -> Ag\pch (a_{Ag\pch}) + Cl\mch (a_{Cl\mch})
\end{reaction*}


\begin{equation*}
    E^\ominus = \phi^\ominus_{\ce{Cl-} | \ce{AgCl} | \ce{Ag}} - \phi^\ominus_{\ce{Ag+} | \ce{Ag}}
\end{equation*}

这种题型的特点是需要从非氧化还原反应构造原电池。可以在两边同时加上一个数字来完成。

\begin{equation*}
    K^\ominus_{sp} = \exp \left( \frac{zE^\ominus F}{RT} \right)    
\end{equation*}


\subsection{求$\Delta_r S_m$}

\begin{equation*}
    \Delta_r S_m = zF \left( \frac{\partial E}{\partial p}  \right)_p 
\end{equation*}

$\left( \frac{\partial E}{\partial p}  \right)_p $ 称为温度系数,可以通过测量一系列$E$,$T$的关系来求得。

\subsection{求$\Delta_r H_m$}

\begin{equation*}
    \Delta_r H_m = -zFE + zFT \left(  \frac{\partial E}{\partial T} \right)_p 
\end{equation*}

\subsubsection{测量方法}

需要测量$E$和$\left(  \frac{\partial E}{\partial T} \right)_p $

\subsubsection{优势}

电动势法测量与量热法比较,精度高。

\subsubsection{热效应}


等温情况下,可逆反应的热效应为

\begin{equation*}
    Q_R = T \Delta_r S_m = zFT \left(  \frac{\partial E}{\partial T} \right)_p
\end{equation*}

\subsection{求电解质溶液的平均活度因子}

\subsubsection{已知标准电极电势}

例子:

\begin{reaction*}
    Pt | H2 "($p^\ominus$)" | HCl (m) | AgCl(s) | Ag(s)
\end{reaction*}

\begin{reaction*}
    1/2 H2 + AgCl -> Ag + {Cl\mch} ( $a_{\ce{Cl-}}$ ) + H\pch ( $a_{\ce{H+}}$ )
\end{reaction*}


电动势计算公式

\begin{equation*}
    E = \left( \phi^\ominus_{\ce{Cl-} | \ce{AgCl} | \ce{Ag+}}   - \phi^\ominus_{\ce{H+} | \ce{H2}} \right)  - \frac{RT}{F} \ln a_{\ce{H+}} a_{\ce{Cl-}}
\end{equation*}

\begin{equation*}
    a_{\ce{H+}} a_{\ce{Cl-}}= \left(\gamma_\pm  \frac{m_{\ce{HCl} }}{m^\ominus} \right)^2
\end{equation*}

\begin{equation*}
    E = \phi^\ominus_{\ce{Cl-} | \ce{AgCl} | \ce{Ag+}} - \frac{2RT}{F} \ln \frac{m_{\ce{HCl}}}{m^\ominus} - \frac{2RT}{F} \ln \gamma_\pm
\end{equation*}

从电极电势表查到$\phi^\ominus_{\ce{Cl-} | \ce{AgCl} | \ce{Ag+}}$的值和测得不同浓度$\ce{HCl}$溶液的电动势$E$,就可以求出不同浓度的$\gamma_\pm$值。如果平均活度因子可以根据Debye-Hückel公式计算,则可以求得$\phi^\ominus_{\ce{Cl-} | \ce{AgCl} | \ce{Ag+}}$的值。

\subsubsection{未知电动势}

对于$1 - 1$型电解质,有$I = m_B, z_+ = \| z_- \| = 1$,则 Debye-Hückel 公式为

\begin{equation*}
    \ln \gamma_\pm = - A' \left\lvert z_+ z_- \right\rvert \sqrt{I} = -A' \sqrt{m_B}
\end{equation*}


\begin{equation*}
    \phi^\ominus_{\ce{Cl-} | \ce{AgCl} | \ce{Ag+}} = E + \frac{2RT}{F} \ln \frac{m_{\ce{HCl}}}{m^\ominus} - \frac{2RTA'}{F} \sqrt{m_{\ce{HCl}}}
\end{equation*}

可以设上面这个式子右边各项的和为$E'$,$E'$对$m_{\ce{HCl}}$作图或$\sqrt{m_{\ce{HCl}}}$作图,在稀溶液范围内可以得到一条直线。这样通过求截距就能求出$\phi^\ominus_{\ce{Cl-} | \ce{AgCl} | \ce{Ag+}}$。
