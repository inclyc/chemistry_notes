
    \section{可逆电池及可逆电极}

    \subsection{电池的习惯表示方法}

    规定:

    \begin{enumerate}
        \item 阳极在左,阴极在右
        \item 物质必须注明状态(定量计算)
        \item $\vert$ ,相界面 $\|$ ,盐桥
    \end{enumerate}

    \subsection{双复原}

    物质和能量的转换都必须能够可逆进行。

    \subsection{可逆电池的类型}

    \begin{enumerate}
        \item 金属、阳离子电池
        \item 氢电极
        \item 氧电极
        \item 卤素电极
        \item 汞齐电极
    \end{enumerate}

    \begin{enumerate}
        \item 金属-难溶盐
        \item 金属-
    \end{enumerate}

    % $\ce{H2 - 2e- -> 2H+}$

    % $\ce{Sb2O3 + 6e- + 6H+ -> 2Sb + 3H2O}$

    \subsection{根据反应设计电池}

    电池与反应的互译,一般先寻找阳极和阴极,在复核。

    \subsubsection{非氧化还原反应的电极反应设计}

    \begin{reaction*}
        AgCl(s) + I- (a1) -> AgI(s) + Cl- (a2)
    \end{reaction*}

    可以给两边都加一个$\ch{Ag}$

    \begin{reaction*}
        AgCl(s) + I- (a1) + Ag -> AgI(s) + Ag + Cl- (a2)
    \end{reaction*}
