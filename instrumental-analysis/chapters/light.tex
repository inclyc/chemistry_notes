\chapter{光分析法}

\section{光分析法导论}

\subsection{吸收、发射、荧光光谱}

\subsubsection{吸收光谱法}

物质吸收了辐射能,使辐射能的强度降低,这样的光谱被称为吸收光谱。

吸收光谱要求的物质状态:液态、固态、均质固态

\subsubsection{发射光谱法}

以火焰、电弧、等离子炬等作为能量源,使气态源自的外层电子受激发射出特征光谱进行定性定量分析的方法。

\subsubsection{荧光光谱法}

也称为光致光谱,其光源为辐射能,物质受外界光能作用被激发,退激发的时候以辐射的形式表现出来,即为荧光光谱。


\subsection{原子光谱和分子光谱}

原子光谱为分立的线状光谱,价电子能级跃迁。分子光谱为连续的带状光谱,分子能级跃
迁。

\section{紫外可见吸收光谱}

\subsection{紫外可见吸收光谱法及其原理}

\subsubsection{电子跃迁与分子吸收光谱}

\paragraph{电子能级} 此类吸收能量差 $\Delta E$ 较大 $1 \sim 2 \mathrm{eV}$。电子跃迁产生的吸收光谱在紫外 $-$ 可见光区,称为紫外 $-$ 可见光谱或分子的电子光谱。

\paragraph{振动能级} 一般能量差 $\Delta E$ 约为 $ 0.05 \sim 1 \mathrm{eV}$,跃迁产生的吸收光谱位于红外区,使用红外光谱或分子转动光谱。

\paragraph{转动能级} $\Delta E$ 为 $0.005 \sim 0.050 \mathrm{eV}$,跃迁产生吸收光谱位于远红外区,使用远红外光谱或分子转动光谱。

\subsubsection{用途}

\begin{enumerate}
    \item 吸收光谱的波长分布是由产生谱带的跃迁能级间的能量差所决定,反映了分子内部能级的分布状况,是物质定性分析的依据。
    \item 吸收谱带的强度与分子偶极矩的变化,跃迁几率油光,也提供分子结构信息,将在最大吸收波长处测得的摩尔吸光系数 $\epsilon_{\max}$ 也作为定性分析的依据。
    \item 吸收谱带强度与该分子吸收的光子数成正比,可以作为定量分析的依据。
\end{enumerate}

紫外可见吸收光谱一般通过分子价电子能级跃迁产生,为连续的带状光谱。光谱的波长范
围一般是 $10 - 800 \ \mathrm{nm}$。测量透过率 $T$ 和吸收系数 $A$,可用于结构鉴
定和定量分析(既可以定性又可以定量)。

透过率 Transmittance $T$

\begin{equation}
    T = \frac{I_t}{I_0} \quad 0 \leq T \le 1
\end{equation}

吸光度 Absorbance $A$

\begin{equation}
    A = - \lg T
\end{equation}

\subsection{有机物电子跃迁与分子吸收光谱}

有机化合物的紫外 $-$ 可见吸收光谱是三种电子跃迁的结果:

$\sigma, \pi, n$ 电子跃迁。

当外层电子吸收紫外光或可见光辐射后,就从基态向激发态(反键轨道)跃迁。主要有四种跃迁,所需的能量 $\Delta E$ 大小顺序为:

\begin{equation}
    \sigma \rightarrow \sigma^* > n \rightarrow \sigma^* > \pi \rightarrow \pi^* > n \rightarrow \pi^*
\end{equation}

\subsubsection{$\sigma \rightarrow \sigma^*$ 跃迁}

此类跃迁所需能量最大,$\sigma$ 电子只有吸收远紫外光的能量才能发生跃迁,吸收波长 $\lambda < 200 nm$。饱和烷烃的分子吸收光谱出现在远紫外区,例如,$\ce{CH4}$ 的 $\lambda_{\max} = 125 \mathrm{nm}$,$\ce{CH3CH3}$ $\lambda_{\max} = 135 \mathrm{nm}$。只能被真空紫外分光光度计检测到,作为溶剂使用。

\subsubsection{$n \rightarrow \sigma^*$ 跃迁}

吸收波长为 $ 150 \sim 250 \mathrm{nm} $,大部分在远紫外区,近紫外区仍不易观察到,所需的能量较大。

含非键电子的饱和烃衍生物(含 $\ce{N}$、$\ce{O}$、$\ce{S}$和卤素等杂原子均呈现 $n \rightarrow \sigma^*$ 跃迁。

末端吸收:

醇类末端吸收不大,可以用作溶剂。

\begin{table}[H]
    \centering
    \caption{各化合物对应的 $\lambda_{\max}$ 和 $\epsilon_{\max}$}
    \begin{tabular}{ccc}
        \toprule
        化合物           & $\lambda_{\max} (\mathrm{nm})$ & $ \epsilon_{\max}$ \\
        \midrule
        $\ce{H2O}$    & 167                            & 1480               \\
        $\ce{CH3OH}$  & 184                            & 150                \\
        $\ce{CH3Cl}$  & 173                            & 200                \\
        $\ce{CH3I}$   & 258                            & 365                \\
        $\ce{CH3NH2}$ & 215                            & 600                \\
        \bottomrule
    \end{tabular}
\end{table}

有一些含有 $n$ 电子的基团,例如 $\ce{-OH}$、$\ce{-OR}$,$\ce{-NH2}$、$-NHR$、$-X$ 等。他们本身不能吸收 $\lambda > 200 \mathrm{nm}$ 的光,但当他们与母核相连时,吸收波长向长波方向移动,且吸收强度增加,此类基团被成为\textbf{助色团}。

\subsubsection{$\pi \rightarrow \pi^*$ 跃迁}

这种跃迁所需的能量比较小,吸收波长处于远紫外区的近紫外端或近紫外区,$\epsilon_{\max}$ 一般在 $10^4 \cdot \mathrm{L} \cdot \mathrm{mol}^{-1} \cdot \mathrm{cm}^{-1}$ 以上,属于强吸收。

主要是双键中的吸收带:

\begin{itemize}
    \item K带:共轭非封闭体系的 $\pi \rightarrow \pi^*$ 跃迁
    \item E带:乙烯型谱带
    \item B带:苯型谱带
\end{itemize}

\subsubsection{$n \rightarrow \pi^*$ 跃迁}

羰基化合物共轭烯烃中的 $n \rightarrow \pi^*$。

R带:带有杂原子的不饱和键形成的吸收带。

\subsubsection{生色团和助色团}

\paragraph{生色团:B、E、K、R带}

最有用的紫外 $-$ 可见光谱是由 $\pi \rightarrow \pi^*$ 和 $n \rightarrow \pi^*$ 跃迁产生的。这两种跃迁均要求有机物分子中含有不饱和基团。这类含有 $\pi$ 键的不饱和基团称为\textbf{生色团}。简单的生色团由双键或三键体系组成。

\paragraph{助色团}

含有 $n$ 电子的基团,例如 $\ce{-OH}$、$\ce{-OR}$,$\ce{-NH2}$、$-NHR$、$-X$ 等。他们本身不能吸收 $\lambda > 200 \mathrm{nm}$ 的光,但当他们与母核相连时,吸收波长向长波方向移动,且吸收强度增加,此类基团被成为\textbf{助色团}。

\subsection{无机化合物的紫外可见吸收光谱}

\subsubsection{配位场跃迁}

$\mathrm{d} - \mathrm{d}$ 电子跃迁

主要分为四面体配位场和八面体配位场,如果有配位场存在,$\mathrm{d} - \mathrm{d}$ 会发生分裂。

\subsubsection{电荷转移跃迁}

辐射作用下,分子中原本定域在金属 $N$ 轨道上的电荷转移到配体 $L$ 的轨道上,或按相反方向转移,产生的吸收光谱。

产生电荷转移跃迁的必要条件是:

络合物的组分之一具有电子给予体的特性,而另一组分具有电子受体的特性。

\begin{equation}
    \ce{{[Fe^{3+}CNS-]}^{2+} ->[h\nu] {[Fe^{2+}CNS]}^{2+}}
\end{equation}

$\ce{Fe^{2+}}$ 与邻菲罗啉的紫外吸收光谱属于这种情况。

\subsubsection{Beer $-$ Lambert law}

\begin{equation}
    A = \lg \frac{I_0}{I_t} = kbc
\end{equation}

一束平行单色光通过均匀、透明的吸光介质时,其吸光度与吸光质点的浓度和吸收层的乘积成正比。

\subsection{仪器的基本组成}

\subsubsection{光源}

在整个紫外光区域或可见光谱区可以发射\textbf{连续光谱},具有足够的辐射强度和稳定性,有较长的使用寿命。

\paragraph{可见光区} \textbf{钨灯},其辐射波长范围在 $320 \sim 2500 \mathrm{nm}$。

\paragraph{紫外区} \textbf{氢灯、氘灯}。发射 $184 \sim 400 \mathrm{nm}$ 的连续光谱。

\subsection{紫外可见吸收光谱的应用}

\subsubsection{定性分析}

对于有机物,如果有足够的生色团和助色团,可以直接测定,另外也可以利用显色反应。

对于无机物,主要靠利用显色反应分析,例如金属阳离子 $\ce{Fe}$ / 阴离子。

\subsubsection{工作/标准曲线法:测定溶液浓度}

\subsubsection{标准加入法}

如果试样基体组成较复杂,又没有纯净的基体空白,很难配置相类似的标准溶液时,使用标准加入法是适合的。

取相同体积的样品空白和待测样品溶液分别移入试管中 ($C_x$),然后将含有待测元素的标准溶液 ($C_0$) 按比例顺序加入待测样品的试管中。

定容后浓度依次为:

\begin{equation}
    C_x, C_x + C_0, C_x + 2C_0, C_x + 3C_0, C_x + 4C_0, \dots
\end{equation}

\subsection{红外吸收光谱}

\subsection{原子吸收光谱}

\section{红外吸收光谱}

具有\textbf{偶极矩}的分子才可以产生红外光谱。

\subsection{红外光谱的基团频率}

与一定结构单元相联系的、在一定范围内出现的化学键振动频率 -- 基团特征频率。

常见的有机化合物基团频率出现在范围:$4000 \sim 670 \ \mathrm{cm}^{-1}$ 依据基团的振动形式可以分为四个区:

\begin{enumerate}
    \item $4000 - 2500$ $\ce{X-H}$ 伸缩振动区
    \item $2500 - 1900$ 三键、累积双键伸缩振动区
    \item $1900 - 1200$ 双键伸缩振动区
    \item $1200 - 670$ $\ce{X-Y}$伸缩、$\ce{X-H}$变形振动区
\end{enumerate}


\subsection{分子结构与吸收峰}

\subsubsection{$\ce{X-H}$ 伸缩振动区 ($4000 - 2500 \ \mathrm{cm}^{-1}$)}


\paragraph{\ce{-O-H}} $3650 - 3200 \ \mathrm{cm}^{-1}$ 确定 醇、酚、酸

在非极性溶剂中,浓度较小时,峰形尖锐,强吸收。浓度较大时,发生缔合作用,峰形较宽。

需要注意区分:

\ce{-NH} 伸缩振动: $3500 \sim 3100 \ \mathrm{cm}^{-1}$

\vspace{0.5em}


\paragraph{不饱和碳原子上的 $\ce{=C-H}$}

$3000 \ \mathrm{cm}^{-1}$ 以上

\paragraph{饱和碳原子上的 $\ce{-C-H}$}

$3000 \ \mathrm{cm}^{-1}$ 以下

例如 $\ce{-CH3}$、$\ce{-CH2-}$、$\ce{-C-H}$。

\subsubsection{三键 $\ce{C#C}$ 振动区 ($2500 - 1900 \ \mathrm{cm}^{-1}$)}

该区域出现的峰较少,主要有 $\ce{RC#CH}$、$\ce{RC#CR^'}$、$\ce{RC#N}$。如果仅含有C、H、N时,峰较强且尖锐。如果含有 O  原子,O越靠近 $\ce{RC#N}$ 峰越弱。


\subsubsection{双键伸缩振动区 ($1900 - 1200 \ \mathrm{cm}^{-1}$)}

\paragraph{$\ce{C=O}$} $1850 - 1600 \ \mathrm{cm}^{-1}$

羰基的特征峰,峰强很大(通常是红外最强峰),同时尖锐。

饱和醛酮 $1740 - 1720 \ \mathrm{cm}^{-1}$ 强、尖、不饱和向低波数移动。

\paragraph{苯}

苯的衍生物在 $1650 - 2000 \ \mathrm{cm}^{-1}$ 出现 $\ce{C-H}$ 和 $\ce{C=C}$ 键的面内变形振动的泛频吸收(强度弱),可用来判断取代基的位置。


\paragraph{\ce{RC=CR^'}} $1620 - 1680 \ \mathrm{cm}^{-1}$

强度弱,对称时无红外活性。

% TODO: 影响峰位变化的因素?(不知道是否要考)
