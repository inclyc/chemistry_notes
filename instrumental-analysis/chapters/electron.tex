\chapter{电化学分析法}

\section{离子选择性电极与膜电极}

\subsubsection{离子选择性电极的原理和结构}

\paragraph{特点} 仅对溶液中特定离子有选择性的响应。关键器件为选择膜的敏感元件:
单晶、混晶、液膜、功能膜及生物膜构成。

\paragraph{膜电位} 膜内外被测离子活度的不同而产生电位差

\begin{equation}
    E = E' \pm \frac{RT}{nF} \ln a_i
\end{equation}


\subsubsection{玻璃膜(非晶体膜)电极}

非晶体膜电极,玻璃膜的组成不同可制成对不同阳离子响应的玻璃电极。

$\ce{H+}$ 响应的玻璃膜电极:敏感膜厚度约为 $0.05 \mathrm{mm}$

$\ce{SiO2}$ 基质中加入 $\ce{Na2O}$、$\ce{Li2O}$ 和 $\ce{CaO}$ 烧结而成的特殊玻
璃膜。水浸泡后,表面的 $\ce{Na+}$ 与水中的 $\ce{H+}$ 交换,表面形成水合硅胶层。
玻璃电极使用前,必须在水溶液中浸泡。生成三层结构,中间的干玻璃层和两边的水化硅胶层。

水化硅胶层厚度:$0.01 - 10 \mu \mathrm m$。在水化层,玻璃上的 $\ce{Na+}$ 与溶液
中的 $\ce{H+}$ 发生离子交换而产生相界电位。

水化层的表面可视作阳离子交换剂。溶液中的 $\ce{H+}$ 经水化层扩散至干玻璃层,干玻
璃层的阳离子向外扩散以补偿溶出的离子,离子的相对移动产生扩散电位,两者之和构成
膜电位。


\subsubsection{氟电极}


\begin{equation}
    E = K \pm \frac{RT}{nF} \ln \left[ a_i + K_{ij} (a_j)^{\frac{n_i}{n_j}}
        \right]
\end{equation}


对阳离子响应的电极,$K$ 后取正号,对负离子响应的电极,$K$ 后取负号。其中,
$K_{ij}$ 被称为\textbf{电极的选择性系数}。其意义为:在向相同的测定条件下,待测
离子和干扰离子产生相同的膜电位时待测离子的活度 $a_i$ 与干扰离子活度 $a_j$ 的比
值。

\begin{equation}
    K_{ij} = \frac{a_i}{a_j^{(n_i/n_j)}}
\end{equation}


$K_{ij}$ 的值越小,电极的选择性越高。$K_{ij} = 0.001$ 时,意味着干扰离子 $j$ 的
活度比待测离子 $i$ 的活度大 1000 倍时,两者产生相同的电位。

\subsection{电位分析法的应用}

\subsubsection{pH 测定原理和方法}

\begin{itemize}
    \item 指示电极:pH玻璃膜电极,负极
    \item 参比电极:饱和甘汞电极,正极
\end{itemize}

电池电动势为:

\begin{equation}
    E = E_{SCE} - E_{g}
\end{equation}

\begin{equation}
    E = K' + \frac{\ln 10}{F}\mathrm{pH}
\end{equation}

实际电位法测定 pH 的依据

\begin{align}
    E_x & = K_x + 0.059 \mathrm{pH}_x \\
    E_s & = K_S + 0.059 \mathrm{pH}_x
\end{align}

若测量条件相同,则有 $K_x = K_S$

\subsubsection{定量测定离子活度的方法}

将离子指示性电极和参比电极插入试液,可以组成测定各种离子活度的电池,此电池的电
动势为:

\begin{equation}
    \Delta E_{M} = K \pm \frac{RT}{nF} \ln a_i
\end{equation}

\begin{itemize}
    \item 该式是测量离子活度的通式
    \item 对阳离子响应的电极取正号,对阴离子响应的电极取负号
    \item 实际测量时,通常采取标准曲线法和标准加入法
\end{itemize}


\paragraph{标准曲线法} $a = \gamma c$

\begin{equation}
    E = K \pm \frac{RT}{nF} \ln a
\end{equation}

纯待测离子配制一系列不同浓度的标准溶液,用 TISAB (总离子强度调节缓冲溶液,Total
Ionic Strength Adjustment Buffer),分别测定。

要使 $K$ 保持恒定,必须保证:

\begin{enumerate}
    \item 实验条件恒定
    \item 电极条件恒定
    \item TISAB 条件恒定
\end{enumerate}

\paragraph{标准加入法} 如果试样的组成比较复杂,用标准曲线法有困难,此时可以采用
标准加入法

方法:将小体积的标准溶液加入到已知体积的未知试液中,根据加标样前后电池电动势的
变化计算试液中被测离子的浓度。

标准加入法的步骤:

\begin{enumerate}
    \item 测定工作电池的电动势 $E_1$:设某一试液的体积为 $V_0$,其待测离子的浓
          度为 $c_\chi$,则此工作电池的电动势为:

          \begin{equation}
              E_1 = K \pm \frac{RT}{nF} \ln c_\chi
          \end{equation}
    \item 向试液中准确加入浓度为 $c_s$ (约为 $c_\chi$ 的 100 倍),体积为 $V_s$
          (大约为 $V_0$ 的 1/100) 的用待测离子的纯物质配置的标准溶液。由于 $V_0 >>
              V_s$,可以认为溶液体积基本不变,浓度增量为 $\Delta c = c_s \frac{V_s}
              {V_0} $,此时再次测量供电电池的电动势:

          \begin{equation}
              E_2 - E_1 = K \pm \frac{RT}{nF} \ln \left( 1 + \frac{m_s}{m_x}
              \right)
          \end{equation}
          \begin{equation}
              \Rightarrow E_2 - E_1 = \frac{RT}{nF} \ln (1 + \frac{\Delta c}
              {c_\chi})
          \end{equation}
\end{enumerate}
