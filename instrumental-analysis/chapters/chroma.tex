\chapter{色谱分析法}

\section{色谱法的提出、分类、特点}

\subsection{色谱法的提出}

\subsection{色谱法的分类}

\subsection{色谱法的特点}

\section{色谱分离的过程}

色谱分离过程是在色谱柱中内完成的。填充柱色谱:气-固(液-固)色谱和气-液(液-液)
色谱,两者的分离机理不同。

\paragraph{\{气,液\}固色谱的分离机理} 组分在两相间吸附和脱附的不断重复的过程。

\paragraph{\{气,液\}液色谱的分离机理} 组分在两相间溶解-挥发过程的反复多次分配过
程。

\subsection{色谱的分离过程}

\begin{enumerate} \item 当试样由流动相携带进入色谱柱与固定相接触时,被固定相溶
          解或吸附。\item 随着流动相的不断通入,被溶解或吸附的组分又从固定相中挥发或
          脱附。\item 挥发或脱附下的组分随着流动相向前移动又再次被固定相溶解或吸附。
    \item 随着载气的流动、溶解、挥发,或吸附、脱附的过程反复多次进行,不同物质得以
          分离。\end{enumerate}


\subsection{分配系数 $K$}

组分在固定相和流动相之间发生吸附-脱附,或溶解-挥发的过程叫做\textbf{P分配过程}。
在一定温度下,组分在两相间分配达到平衡时的浓度比,称为分配系数,用 $K$ 来表示

\begin{equation} K = \frac{c_s}{c_M} \end{equation}

其中,$c_s$ 代表组分在固定相中的浓度,$c_M$ 代表组分在流动相中的浓度。分配系数
是色谱分离的依据,各组分分配系数的差异越大,分离效果越好。



\begin{itemize} \item 不同组分在固定相上的分配系数 $K$ 不同,组分确定时,$K$ 主
          要取决于固定相的性质。\item 选择适宜的固定相,使组分间分配系数的差异增大,
          可改善分离效果。\item 试样中的各组分具有不同的 $K$ 值是分离的基础,$K$ 值小
          的先出峰,$K$ 值大的后出峰。\item $K = 0$ 时,不被固定相保留,最先流出
\end{itemize}

\subsection{分配比 $k$}

在实际工作中,也常用分配比来表征色谱分配平衡过程。分配比指在一定温度下,组分在两相间达到平衡时的质量比。

\begin{equation}
    k = \frac{m_s}{m_M}
\end{equation}

分配比也称为容量因子或容量比。

\begin{itemize}
    \item 分配系数与分配比都是组分及固定相的热力学性质有关的常数,随着分离住的温度、柱压的改变而变化。
    \item 都是衡量色谱柱对组分保留能力的参数,数值越大,该组分保留时间越长
    \item 分配比可以由实验测得
\end{itemize}

分配比与分配系数的关系:

\begin{equation}
    k = \frac{m_s}{m_m} = \frac{c_s}{c_m} \cdot \frac{V_s}{V_m} = \frac{K}{\beta}
\end{equation}


\section{色谱流出曲线及术语}

\paragraph{基线} 无试样通过检测器时,检测到的信号即位基线

\paragraph{保留值} 用时间表示的保留值

保留时间 $t_R$:组分从进样到柱后出现浓度极大值时所需的时间。

死时间 $t_M$:不与固定相作用的组分保留时间

调整保留时间 $t_R' = t_R - t_M$

\paragraph{相对保留值$r_{21}$} 组分 2 与组分 1 调整保留时间之比

\begin{equation}
    r_{21} = \frac{t_{R2}'}{t_{R1}'}
\end{equation}

\begin{equation}
    k = \frac{t_R'}{t_m}
\end{equation}


相对保留值只与柱温和固定相性质有关,与其他色谱操作条件无关,它表示了固定相对这
两种组分的选择性。


\section{色谱柱分离的效能}

\subsection{塔板理论 - 色谱柱分离效能指标}

色谱柱长:$L$, 虚拟的塔板间的距离:理论塔板高度 $H$,组分在色谱柱中分配平衡次数:
理论塔板数 $n$

则三者的关系为:

\begin{equation}
    n = \frac{L}{H}
\end{equation}

理论塔板数与色谱参数之间的关系为:

\begin{equation}
    n = 5.54 \left( \frac{t_R}{Y_{1/2}} \right)^2 = 16 \left(\frac{t_R}{W_b}\right)^2
\end{equation}

\subsection{速率理论 -- 影响柱效的因素}

速率方程为:

\begin{equation}
    H = A + \frac{B}{u} + C u
\end{equation}

\begin{table}[H]
    \caption{速率方程 (a.k.a Van Deemter's equation) 中各项的意义}
    \centering
    \begin{tabular}{ccc}
        \toprule
        符号  & 意义     & 备注          \\
        \midrule
        $H$ & 理论塔板高度 & $H$ 越小,柱效越高 \\
        $u$ & 载气的线速度 & 存在最佳流速      \\
        $A$ & 涡流扩散项  & 常量          \\
        $B$ & 分子扩散项  & 常量          \\
        $C$ & 传质阻力项  & 常量          \\
        \bottomrule
    \end{tabular}
\end{table}


想办法减小 $A$、$B$、$C$三项可以提高柱效。

\subsubsection{$A$ - 涡流扩散项}

\begin{equation}
    A = 2 \lambda d p
\end{equation}

$d p$ : 固定相的平均颗粒直径

$\lambda$:固定相的填充不均匀因子

固定相颗粒越小,$d p \downarrow$,填充的越均匀,$A \downarrow$, $H \downarrow$,
柱效 $n \uparrow$。表现在涡流扩散所引起的色谱峰变宽的现象减轻,色谱峰较窄。

\subsubsection{$B/u$ 分子扩散项}

\begin{equation}
    B = 2 v D_g
\end{equation}

$v$: 弯曲因子,填充柱 $v < 1$,空心毛细管柱$v = 1$

$D_g$: 试样组分分子在气相中的扩散系数 $(\mathrm{cm}^2 \cdot \mathrm{s}^{-1})$

\begin{enumerate}
    \item 存在浓度差,产生纵向扩散
    \item 扩散导致色谱峰变宽,$H \uparrow (n\downarrow)$,分离变差
    \item 分子扩散项与流速有关,流速 $\downarrow$,滞留时间 $\uparrow$,扩散 $\uparrow$
    \item 扩散系数 $D_g \propto \left(M_{g}\right)^{-\frac{1}{2}}$
\end{enumerate}

\subsection{传质阻力项}

传质阻力包括气相传质阻力 $C_g$ 和液相传质阻力 $C_L$

\begin{align}
    C   & = C_g + C_l                                                     \\
    C_g & = \frac{0.01k}{(1 + k)^2} \cdot \frac{d_p^2}{D_g}               \\
    C_L & = \frac{2}{3} \cdot \frac{k}{(1 + k)^2} \cdot \frac{d_f^2}{D_L}
\end{align}

$k$ 为容量因子,$D_L$ 为液相扩散系数,$d_f^2$ 为液膜厚度。减少担体粒度 $d p$,
减小液膜厚度 $d f$,选择小分子量的气体作载气,可以降低传质阻力。

\subsubsection{分离度 $R$}

\begin{equation}
    R = \frac{2\left(t_{R(2)} - t_{R(1)}\right)}{W_{b(2)} + W_{b(1)}}
\end{equation}

当 $W_{b(2)} = W_{b(1) = W_b}$,可以引入相对保留值和塔板数,可以导出:

\begin{equation}
    R = \frac{(r_{21} - 1)}{r_{21}} \sqrt{\frac{n}{16}}
\end{equation}

\paragraph{分离度与柱效} 分离度与柱效 $n$ 的平分根成正比,$r_{21}$ 一定时,增加
柱效,可提高分离度,但组分保留时间增加且峰扩展,分析时间加长。

\paragraph{分离度与 $r_{21}$}

\begin{enumerate}
    \item 增大 $r_{21}$ 是提高分离度的最有效方法
    \item 增大 $r_{21}$ 的最有效方法是选择合适的固定相
\end{enumerate}

\subsection{分离条件的选择}

\subsubsection{载气流速的选择}

\begin{align}
    H                               & = A + \frac{B}{u} + C \cdot u \\
    \frac{\mathrm d H}{\mathrm d u} & = - \frac{B}{u^2} + C
\end{align}

可见,当 $ u_0 = \frac{B}{C} $ 时,取最大值。

\subsubsection{柱温的选择}

\begin{enumerate}
    \item 首先应使柱温控制在固定液的最高使用温度和最低使用温度范围之内
    \item 柱温 $\uparrow$,分离度 $\downarrow$,色谱峰变窄变高。柱温 $\uparrow$,
          被测组分的挥发度 $\uparrow$,被测组分在气相中的浓度 $\uparrow$,低沸点
          峰易产生重叠
    \item 柱温 $\downarrow$,分离度 $\uparrow$,分析时间 $\uparrow$。对于难分离
          的物质对,降低柱温虽然可在一定程度上使分离得到改善,但是不可能使之完全
          分离,这是由于两组分的相对保留值增大的同时,两组分的峰宽也在增加,当后
          者的增加速度大于前者时,两峰的交叠更严重
    \item 柱温一般选择在接近或低于组分平均沸点时的温度。组分复杂,沸程宽的试样,
          应该采用\textbf{程序升温}
\end{enumerate}

\subsubsection{柱长和内径的选择}

增加柱长对提高分离度有利(分离度 $R^2$ 正比与柱长 $L$)但组分的保留时间 $t_R
    \uparrow$ 且阻力 $\uparrow$ ,不便操作。

柱长的选用原则是在能满足分离目的的前提下,尽量可能选用较短的柱,有利于缩短分析
时间。填充色谱柱的柱长通常为 $1 - 3 \mathrm{m}$
