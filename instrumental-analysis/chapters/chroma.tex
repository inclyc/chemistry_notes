\chapter{色谱分析法}

\section{色谱法的提出、分类、特点}

\subsection{色谱法的提出}

\subsection{色谱法的分类}

\subsection{色谱法的特点}

\section{色谱分离的过程}

色谱分离过程是在色谱柱中内完成的。填充柱色谱:气-固(液-固)色谱和气-液(液-液)
色谱,两者的分离机理不同。

\paragraph{\{气,液\}固色谱的分离机理} 组分在两相间吸附和脱附的不断重复的过程。

\paragraph{\{气,液\}液色谱的分离机理} 组分在两相间溶解-挥发过程的反复多次分配过
程。

\subsection{色谱的分离过程}

\begin{enumerate} \item 当试样由流动相携带进入色谱柱与固定相接触时,被固定相溶
          解或吸附。\item 随着流动相的不断通入,被溶解或吸附的组分又从固定相中挥发或
          脱附。\item 挥发或脱附下的组分随着流动相向前移动又再次被固定相溶解或吸附。
    \item 随着载气的流动、溶解、挥发,或吸附、脱附的过程反复多次进行,不同物质得以
          分离。\end{enumerate}


\subsection{分配系数 $K$}

组分在固定相和流动相之间发生吸附-脱附,或溶解-挥发的过程叫做\textbf{P分配过程}。
在一定温度下,组分在两相间分配达到平衡时的浓度比,称为分配系数,用 $K$ 来表示

\begin{equation} K = \frac{c_s}{c_M} \end{equation}

其中,$c_s$ 代表组分在固定相中的浓度,$c_M$ 代表组分在流动相中的浓度。分配系数
是色谱分离的依据,各组分分配系数的差异越大,分离效果越好。



\begin{itemize} \item 不同组分在固定相上的分配系数 $K$ 不同,组分确定时,$K$ 主
          要取决于固定相的性质。\item 选择适宜的固定相,使组分间分配系数的差异增大,
          可改善分离效果。\item 试样中的各组分具有不同的 $K$ 值是分离的基础,$K$ 值小
          的先出峰,$K$ 值大的后出峰。\item $K = 0$ 时,不被固定相保留,最先流出
\end{itemize}

\subsection{分配比 $k$}

在实际工作中,也常用分配比来表征色谱分配平衡过程。分配比指在一定温度下,组分在两相间达到平衡时的质量比。

\begin{equation}
    k = \frac{m_s}{m_M}
\end{equation}

分配比也称为容量因子或容量比。

\begin{itemize}
    \item 分配系数与分配比都是组分及固定相的热力学性质有关的常数,随着分离住的温度、柱压的改变而变化。
    \item 都是衡量色谱柱对组分保留能力的参数,数值越大,该组分保留时间越长
    \item 分配比可以由实验测得
\end{itemize}

分配比与分配系数的关系:

\begin{equation}
    k = \frac{m_s}{m_m} = \frac{c_s}{c_m} \cdot \frac{V_s}{V_m} = \frac{K}{\beta}
\end{equation}


\section{色谱流出曲线及术语}

\paragraph{基线} 无试样通过检测器时,检测到的信号即位基线

\paragraph{保留值} 用时间表示的保留值

保留时间 $t_R$:组分从进样到柱后出现浓度极大值时所需的时间。

死时间 $t_M$:不与固定相作用的组分保留时间

调整保留时间 $t_R' = t_R - t_M$

\paragraph{相对保留值$r_{21}$} 组分 2 与组分 1 调整保留时间之比

\begin{equation}
    r_{21} = \frac{t_{R2}'}{t_{R1}'}
\end{equation}

\begin{equation}
    k = \frac{t_R'}{t_m}
\end{equation}


相对保留值只与柱温和固定相性质有关,与其他色谱操作条件无关,它表示了固定相对这
两种组分的选择性。
