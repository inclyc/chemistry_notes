\chapter{高效液相色谱法}

\section{高效液相色谱概述及基本理论}

特点:高压,高效,告诉,高灵敏度,可制备。用于分析高沸点、热不稳定有机及生化试样的高效分离分析方法。

\subsubsection{分离理论}

对于液相色谱,塔板理论和速率理论同样适用,由于流动相为液体,粘度大,分子扩散项的影响较小,而传质阻力项影响较大。

\begin{equation}
    H = A + \frac{B}{u} + Cu
\end{equation}

\begin{enumerate}
    \item 涡流扩散项 $A = 2 \lambda dp$
    \item 分子扩散项 $B / u = C d D m / u$ 可忽略
    \item 传质阻力项 $Cu$
\end{enumerate}

传质阻力项可分为:

固体相传质阻力

\begin{equation}
    H_s = (C_s df^2 / D_s)u
\end{equation}

流动相传质阻力

\begin{equation}
    H_m + H_{sm}
\end{equation}

综合考虑,$H = A + Cu$,$dp \downarrow$,$df \downarrow$ 有利于提高柱效


