\chapter{气相色谱分析法}

\section{气相色谱固定相}

\subsection{气固色谱固定相}

具有一定活性的吸附剂,种类有限。分离对象主要是永久性气体和气态烃类物质。

气固色谱固定相的特点

\begin{enumerate}
    \item 性能与制备和活化条件有很大关系
    \item 同一种固定相,不同厂家或不同的活化条件,分离效果差异较大
    \item 种类有限,能分离的对象不多
    \item 使用方便
\end{enumerate}

\subsection{气液色谱固定相}

气液色谱固定相:担体(支持体) + 固定液

\subsubsection{担体}

一种化学惰性、多孔型的固定颗粒,提供一个大的惰性表面,用以承担固定液。

担体的要求:

\begin{itemize}
    \item 多孔性,孔径分布均匀
    \item 化学惰性,有较好的浸润性
    \item 热稳定性好
    \item 一定的机械强度
\end{itemize}

担体(硅藻土)主要分为红色担体和白色担体。

\paragraph{红色担体} 孔径较小,表孔密集,比表面积大,机械强度好。缺点是表面存有
活性吸附中心点。适宜分离非极性或弱极性组分的试样。

\paragraph{白色担体} 煅烧前原料加入了少量助熔剂(碳酸钠)。颗粒疏松,孔径较大。
比表面积小,机械强度较差。但吸附性显著减小,适宜分离极性组分的试样。


\subsubsection{固定液}

高沸点难挥发的有机物或聚合物。

固定液的要求:

\begin{enumerate}
    \item 挥发小,具有较低的蒸气压
    \item 热稳定性好
    \item 熔点不能太高
    \item 化学稳定性好
    \item 对样品中的各组分有适当的溶解度
\end{enumerate}

固定液的选择,主要根据极性来选。

\paragraph{非极性组分分离} 选用非极性固定液,组分出峰的顺序由蒸气压决定,沸点高
保留时间长

\paragraph{中等极性组分分离} 中等极性固定相,沸点与分子间作用力同时起作用

\paragraph{强极性组分分离} 强极性固定相,分子间作用力起作用,按极性大小出峰

\paragraph{极性 + 非极性组分分离} 极性固定相

\section{气相色谱检测器}

\subsection{热导池检测器}

电路是惠斯通电桥,做出参考臂和测量臂,利用气体导热系数不一样产生电阻差异,测量
电桥不平衡来检测。需要物质有高导热系数,适合 $\ce{H2}$ 、${\ce{He}}$

\subsection{氢火焰检测器}

做有机物的分析,在火焰的作用下解离,产生带电离子,进而测定成分。

\subsection{电子捕获检测器}

针对电负性物质,射线源照射载气电离形成稳定基流,含电负性的原子捕获电子产生稳定
负离子并与载气正离子结合,使基流信号下降,由此可检测组分。

\subsection{火焰光度检测器}

\section{气相色谱法的定性分析}

\subsection{利用纯物质定性的方法}

\subsection{利用文献保留值定性}

\subsection{利用保留指数}

\subsection{与其他分析一起联用的定性方法}


\section{气相色谱法的定量分析}


\subsection{基本依据}

在一定条件下,色谱峰的峰高 $h_i$ 或峰面积 $A_i$ 与所测组分的浓度 $c_i$ 或质量
$m_i$ 成正比。

\begin{align}
    m_i & = f_i \cdot A_i \\
    c_i & = f_i \cdot h_i
\end{align}

一般来说,对浓度敏感性检测器,常用于峰高定量。对质量敏感性的检测器,常用于峰面
积定量。

$f_i$ 是\textbf{定量矫正因子}。

\subsection{峰面积的测量}

\subsubsection{半定量}

\begin{align}
    A & = 1.064 h \cdot Y_{1/2}             \\
    A & = h \cdot (Y_{0.15} + Y_{0.85}) / 2 \\
    A & = h b t_R
\end{align}

\subsubsection{定量}

自动积分 / 计算机积分

\subsection{定量方法}

\subsubsection{归一化法}

以样品中被测组分经校正过的峰面积(或峰高)占样品中各组分经矫正过的峰面积(或峰
高)的总和的比例,来表示样品中各组分含量的定量方法。

\begin{equation}
    w_i = \frac{m_i}{\sum m_i}
\end{equation}

特点和要求:

\begin{enumerate}
    \item 归一化法简便、准确
    \item 进样量的准确性和操作条件的变动对测定结果影响不大
    \item 仅适用于\textbf{试样中所有组分全出峰的情况}
\end{enumerate}

\subsubsection{外标法(标准曲线法)}

\begin{enumerate}
    \item 外标法不使用校正因子,准确性较高
    \item 操作条件变化对结果准确性影响较大
    \item 对进样量的准确性控制要求较高,适用与大批量试样的快速分析
\end{enumerate}

\subsubsection{内标法}

将一定量的标准物作为内标物,加入到准确称取的试样当中,根据被测物和内标物的重量
及其在色谱图上相应的峰面积比,求出某组分的含量

\begin{equation}
    \frac{m_i}{m_s} = \frac{f_i' A_i}{f_s A_S} \quad
    m_i = m_s \frac{f_i' A_i}{f_s' A_s}
\end{equation}


特点和要求:

\begin{enumerate}
    \item 试样中所有组分不能全都出峰,或者只需要测定试样中某几个组分。
    \item 准确性较高,操作条件和进样量的稍许变动对定量结果影响不大。
    \item 每个试样的分析,都要进行两次称量,不适合大批量试样的快速分析。
\end{enumerate}


内标物要满足的要求:

\begin{enumerate}
    \item 试样中不存在纯物质
    \item 与被测组分性质比较接近
    \item 不与试样发生化学反应
    \item 出峰位置应位于被测物质附近
\end{enumerate}

\subsubsection{标准加入法}

特殊的内标法。将\textbf{欲测定组分的纯物质}作为内标物,加入到待测样品中,测定加
入欲测定组分的纯物质前后的峰面积,求出组分的含量。
