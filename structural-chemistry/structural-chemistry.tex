\documentclass[a4paper]{ctexrep}

\usepackage{tikz}
\usepackage[version=4]{mhchem}
\usepackage{graphicx}
\usepackage{float}
\usepackage{xcolor}
\usepackage{mathtools}

\newcommand{\mol}{\mathrm{mol}}
\renewcommand{\d}{\mathrm{d}}


\title{结构化学B -- 那永}
\author{Y.C. Long}


\begin{document}
    \maketitle
    \chapter{绪论}
    \section{发展历史与研究内容}

    用量子力学原理研究院子结构,化学键理论,分子结构,各种光谱和电子能谱(核磁的位置、红外吸收峰的位置),化学反应理论。

    量化计算:各种无机有机分子间作用力,超分子,生物大分子,纳米材料的结构与性能关系的科学。

    \section{学习目的和意义}

    结构化学是四大化学的基础。

    \begin{enumerate}
        \item 建立物质结构观点,更深层次了解物质结构及其反应本质
        \item 增强物质结构的意识
    \end{enumerate}

    物质结构基础 $\Rightarrow$ 量子化学 $\Rightarrow$ 量化计算(预测、模拟)

    \chapter{量子力学基础}

    三个著名的实验用经典力学解释不了,才有了量子力学的提出。这三个实验主要是:黑体辐射、光电效应、氢原子光谱。

    \section{微观粒子的波粒二象性}

    \subsection{光的波粒二象性}

    \[
        h = 6.63 \times 10^{-34} \ \mathrm{m^2 kg / s}  
    \]

    \[
        p = mc = \frac{h}{\lambda} 
    \]

    \subsection{德布罗意波}

    \[
        p = mc = \frac{h}{\lambda} 
    \]

    \subsubsection{电子的衍射实验}

    1927年,美国科学家戴维逊--电子衍射实验。用电子射线发生器通过金属箔。

    \subsection{不确定性关系}

    有一些成对的可观测量,要同时测定他们的任意精确值是不可能的。

    \[
        \Delta x \cdot \Delta p_x \geq h  
    \]

    \[
        \Delta y \cdot \Delta p_y \geq h  
    \]

    \section{量子力学的基本假设}

    \subsection{波函数与几率}

    \textit{假设一:微观粒子的运动状态可以用波函数来表示。}
    
    含时波函数$\psi(x, y, z, t)$,定态波函数$\psi(x, y, z)$。用复值平面波来代表电子的运动状态。

    \[
        p = |\psi|^2 \times V   
    \]

    波函数没有物理意义,$|\psi|^2$有物理意义,代表几率密度。在空间内电子的运动规律是其几率密度。

    \subsubsection{波函数的重要性质}

    $c\psi$与$\psi$描述同一状态。波函数前乘以一个系数,描述的状态不发生改变。

    \subsection{力学量与算符}

    算符就是表示运算过程中的符号。$\sqrt{}$ , $\frac{\mathrm{d}}{\mathrm{d}x}$,$\frac{\partial}{\partial x}$, $\frac{\partial^2}{\partial x^2}$

    \[\hat{F}\psi = F \times \psi \]

    \subsubsection{动量算符}
    
    \[\hat{p} = \frac{h}{2 \pi i} \frac{\partial}{\partial x}\]

    \textbf{举例:}
    
    设
    
    \[\psi = e^{ax}, \hat{F}\psi = \frac{\d \psi}{\d x} \]

    \[
        \psi(x) = Ae^{\left(\frac{-2\pi i}{\hbar}\right)Et} e^{\frac{i}{\hbar}5x}
    \]

    \[
        \hat{p} \psi = 5 \times \psi   
    \]

    \[
        p_x = 5  
    \]

    \subsubsection{动能算符}

    量子力学中,动能和动量的关系依然不变:
    
    \[
        T = \frac{p^2}{2m}  
    \]

    \begin{align*}
        \Rightarrow \hat{T}_x \psi &= \frac{p^2}{2m} \psi  \\
        &= \frac{1}{2m}(p \cdot p \cdot psi) \\ 
        &= \frac{1}{2m} \left[\hat{p} \left( p \psi \right) \right] \left(
            \psi ' = p \cdot \psi
         \right) \\ 
        &= \frac{1}{2m} \left[ \hat{p} \cdot \left(
            \hat{p} \psi
         \right) \right] \\
        &= \frac{1}{2m} \frac{\d}{\d x} \left( \frac{\d \psi}{\d x} \right) \left( -i \hbar \cdot i \hbar \right) \\
        &= - \frac{\hbar^2}{2m} \frac{\d ^2}{\d x^2}
    \end{align*}

    \subsubsection{坐标算符和势能算符}

    \[
        \hat{x} = x \cdot  
    \]

    \[
        \hat{x} \psi = x \cdot \psi  
    \]

    \[
        \hat{V} = V \cdot   
    \]

    \[
        \hat{V} \psi = V \cdot \psi  
    \]


    \subsubsection{哈密顿算符和总能量}

    \[
        E_{\mbox{总}} = T + V  
    \]

    \subsection{本征值和本征方程}

    如果按照算符作用到波函数,所求的动量在常数方向上具有本征值。如果算出来的东西是一个不确定量,那么这个算符就没有本征值。

    \subsection{薛定谔方程(假设三)}
    
    \[
        \hat{H} \psi = E \psi  
    \]

    \subsubsection{波函数的合格条件}

    \begin{enumerate}
        \item 单值,保证空间几率密度的唯一化
        \item 二阶可导,否则方程无解
        \item 平方可积(归一化)
    \end{enumerate}

    \[
        \int \psi^{*} \cdot \psi d\tau = 1  
    \]

    或

    \[
        \int \psi^{*} \cdot \psi d\tau = \mathrm{C} 
    \]

    归一化条件

    \[
        \psi' = \frac{1}{\sqrt{k}} \cdot \psi  
    \]

    \subsubsection{平均值(假设四)}

    \[
        \hat{x} \cdot \psi = x \cdot \psi  
    \]

    这种函数求不出本征值,可以考虑求平均值

    \[
        \bar{F} = \frac{\int \psi^* \hat{F} \psi \d \tau}{\int \psi^* \psi \d \tau}
    \]

    \subsubsection{态叠加原理}

    \[
        \psi = c_1   
    \]
\end{document}