\documentclass[a4paper]{ctexrep}

\usepackage{tikz}
\usepackage[version=4]{mhchem}
\usepackage{graphicx}
\usepackage{float}
\usepackage{xcolor}
\usepackage{mathtools}

\newcommand{\mol}{\mathrm{mol}}
\renewcommand{\d}{\mathrm{d}}


\title{结构化学B -- 那永}
\author{Y.C. Long}


\begin{document}
    \maketitle
    \chapter{绪论}
    \section{发展历史与研究内容}

    用量子力学原理研究院子结构,化学键理论,分子结构,各种光谱和电子能谱(核磁的位置、红外吸收峰的位置),化学反应理论。

    量化计算:各种无机有机分子间作用力,超分子,生物大分子,纳米材料的结构与性能关系的科学。

    \section{学习目的和意义}

    结构化学是四大化学的基础。

    \begin{enumerate}
        \item 建立物质结构观点,更深层次了解物质结构及其反应本质
        \item 增强物质结构的意识
    \end{enumerate}

    物质结构基础 $\Rightarrow$ 量子化学 $\Rightarrow$ 量化计算(预测、模拟)

    \chapter{量子力学基础}

    三个著名的实验用经典力学解释不了,才有了量子力学的提出。这三个实验主要是:黑体辐射、光电效应、氢原子光谱。

    \section{微观粒子的波粒二象性}

    \subsection{光的波粒二象性}

    \[
        h = 6.63 \times 10^{-34} \ \mathrm{m^2 kg / s}  
    \]

    \[
        p = mc = \frac{h}{\lambda} 
    \]

    \subsection{德布罗意波}

    \[
        p = mc = \frac{h}{\lambda} 
    \]

    \subsubsection{电子的衍射实验}

    1927年,美国科学家戴维逊--电子衍射实验。用电子射线发生器通过金属箔。

    \subsection{不确定性关系}

    有一些成对的可观测量,要同时测定他们的任意精确值是不可能的。

    \[
        \Delta x \cdot \Delta p_x \geq h  
    \]

    \[
        \Delta y \cdot \Delta p_y \geq h  
    \]

    
\end{document}