\documentclass[a4paper]{ctexrep}

\usepackage{tikz}
\usepackage[version=4]{mhchem}
\usepackage{graphicx}
\usepackage{float}
\usepackage{xcolor}
\usepackage{mathtools}

\newcommand{\mol}{\mathrm{mol}}
\renewcommand{\d}{\mathrm{d}}


\title{结构化学B -- 那永}
\author{Y.C. Long}


\begin{document}
    \maketitle
    \chapter{绪论}
    \section{发展历史与研究内容}

    用量子力学原理研究院子结构,化学键理论,分子结构,各种光谱和电子能谱(核磁的位置、红外吸收峰的位置),化学反应理论。

    量化计算:各种无机有机分子间作用力,超分子,生物大分子,纳米材料的结构与性能关系的科学。

    \section{学习目的和意义}

    结构化学是四大化学的基础。

    \begin{enumerate}
        \item 建立物质结构观点,更深层次了解物质结构及其反应本质
        \item 增强物质结构的意识
    \end{enumerate}

    物质结构基础 $\Rightarrow$ 量子化学 $\Rightarrow$ 量化计算(预测、模拟)

    \chapter{量子力学基础}

    三个著名的实验用经典力学解释不了,才有了量子力学的提出。这三个实验主要是:黑体辐射、光电效应、氢原子光谱。

    \section{微观粒子的波粒二象性}

    \subsection{光的波粒二象性}

    \[
        h = 6.63 \times 10^{-34} \ \mathrm{m^2 kg / s}  
    \]

    \[
        p = mc = \frac{h}{\lambda} 
    \]

    \subsection{德布罗意波}

    \[
        p = mc = \frac{h}{\lambda} 
    \]

    \subsubsection{电子的衍射实验}

    1927年,美国科学家戴维逊--电子衍射实验。用电子射线发生器通过金属箔。

    \subsection{不确定性关系}

    有一些成对的可观测量,要同时测定他们的任意精确值是不可能的。

    \[
        \Delta x \cdot \Delta p_x \geq h  
    \]

    \[
        \Delta y \cdot \Delta p_y \geq h  
    \]

    \section{量子力学的基本假设}

    \subsection{波函数与几率}

    \textit{假设一:微观粒子的运动状态可以用波函数来表示。}
    
    含时波函数$\psi(x, y, z, t)$,定态波函数$\psi(x, y, z)$。用复值平面波来代表电子的运动状态。

    \[
        p = |\psi|^2 \times V   
    \]

    波函数没有物理意义,$|\psi|^2$有物理意义,代表几率密度。在空间内电子的运动规律是其几率密度。

    \subsubsection{波函数的重要性质}

    $c\psi$与$\psi$描述同一状态。波函数前乘以一个系数,描述的状态不发生改变。

    \subsection{力学量与算符}

    算符就是表示运算过程中的符号。$\sqrt{}$ , $\frac{\mathrm{d}}{\mathrm{d}x}$,$\frac{\partial}{\partial x}$, $\frac{\partial^2}{\partial x^2}$

    \[\hat{F}\psi = F \times \psi \]

    \subsubsection{动量算符}
    
    \[\hat{p} = \frac{h}{2 \pi i} \frac{\partial}{\partial x}\]

    \textbf{举例:}
    
    设
    
    \[\psi = e^{ax}, \hat{F}\psi = \frac{\d \psi}{\d x} \]

    \[
        \psi(x) = Ae^{\left(\frac{-2\pi i}{\hbar}\right)Et} e^{\frac{i}{\hbar}5x}
    \]

    \[
        \hat{p} \psi = 5 \times \psi   
    \]

    \[
        p_x = 5  
    \]

    \subsubsection{动能算符}

    量子力学中,动能和动量的关系依然不变:
    
    \[
        T = \frac{p^2}{2m}  
    \]

    \begin{align*}
        \Rightarrow \hat{T}_x \psi &= \frac{p^2}{2m} \psi  \\
        &= \frac{1}{2m}(p \cdot p \cdot psi) \\ 
        &= \frac{1}{2m} \left[\hat{p} \left( p \psi \right) \right] \left(
            \psi ' = p \cdot \psi
         \right) \\ 
        &= \frac{1}{2m} \left[ \hat{p} \cdot \left(
            \hat{p} \psi
         \right) \right] \\
        &= \frac{1}{2m} \frac{\d}{\d x} \left( \frac{\d \psi}{\d x} \right) \left( -i \hbar \cdot i \hbar \right) \\
        &= - \frac{\hbar^2}{2m} \frac{\d ^2}{\d x^2}
    \end{align*}

    \subsubsection{坐标算符和势能算符}

    \[
        \hat{x} = x \cdot  
    \]

    \[
        \hat{x} \psi = x \cdot \psi  
    \]

    \[
        \hat{V} = V \cdot   
    \]

    \[
        \hat{V} \psi = V \cdot \psi  
    \]


    \subsubsection{哈密顿算符和总能量}

    \[
        E_{\mbox{总}} = T + V  
    \]

    \subsection{本征值和本征方程}

    如果按照算符作用到波函数,所求的动量在常数方向上具有本征值。如果算出来的东西是一个不确定量,那么这个算符就没有本征值。

    \subsection{薛定谔方程(假设三)}
    
    \[
        \hat{H} \psi = E \psi  
    \]

    \subsubsection{波函数的合格条件}

    \begin{enumerate}
        \item 单值,保证空间几率密度的唯一化
        \item 二阶可导,否则方程无解
        \item 平方可积(归一化)
    \end{enumerate}

    \[
        \int \psi^{*} \cdot \psi d\tau = 1  
    \]

    或

    \[
        \int \psi^{*} \cdot \psi d\tau = \mathrm{C} 
    \]

    归一化条件

    \[
        \psi' = \frac{1}{\sqrt{k}} \cdot \psi  
    \]

    \subsection{平均值(假设四)}

    \[
        \hat{x} \cdot \psi = x \cdot \psi  
    \]

    这种函数求不出本征值,可以考虑求平均值

    \[
        \bar{F} = \frac{\int \psi^* \hat{F} \psi \d \tau}{\int \psi^* \psi \d \tau}
    \]

    \subsubsection{态叠加原理}

    \[
        \psi = \sum_{i=1}^n c_i \psi_i
    \]

    系数对概率的贡献是平方关系。


    \subsubsection{正交性}

    两个状态间不影响,其积分为0。

    \[
       \int c_ic_j \psi_i \psi_j d\tau = 0
    \]

    \[
       A = \frac{\int (c_1 \psi_x + c_2 \psi_y + c_3 \psi_z) \hat{F} (c_1 \psi_x^* + c_2 \psi_y^* + c_3 \psi_z^*)}{k} 
    \]

    如果$\hat{F}$有本征值,其本征值为$F$,则

    \[
        A = \frac{\sum\limits_{i=1}^n |c_i|^2 F }{ \sum\limits_{i=1}^n |c_i|^2 } 
    \] 
    

    \subsubsection{泡利不相容原理}

    描述多电子体系的完全波函数,交换其中任意两个电子的完全坐标,波函数是反对称的。

    电子必须旋转720度,才能恢复到原来的状态。电子的自旋量子数为$\frac{1}{2}$


    \section{一维势箱}

    \subsection{模型特点}

    \[
        V = \begin{cases}
            0, & 0 < x < l \\
            \infty, & x \le 0, x \ge l \\
        \end{cases} 
    \]

    考虑$0 < x < l$时,$V=0$,薛定谔方程:

    \[
        -\frac{\hbar^2}{2m} \frac{\d^2 \psi}{\d x^2} = E \psi
    \]

    \[
        \psi(x) = D \sin \frac{\sqrt{2mE}}{\hbar} x   
    \]

    \[
        \psi(l) = D \sin \frac{\sqrt{2mE}}{\hbar} l = 0
    \]

    \[
        \sin \sqrt{2mE} \frac{l}{h} = 0
    \]

    \[
        \sqrt{2mE} \frac{l}{\hbar} = n \pi 
    \]

    \begin{align*}
        E &= \frac{n^2\pi^2\hbar^2}{2ml^2} \\
        &= \frac{n^2h^2}{8ml^2} \qquad (n = 1, 2, 3, \dots)
    \end{align*}

    \begin{align*}
        \psi (x) &= D \cdot \sin \sqrt{\frac{2mn^2h^2}{8ml^2} } \cdot \frac{2\pi}{h} x \\
        &= D \cdot \sin \frac{n\pi}{l} x
    \end{align*}

    \subsection{能量的量子化}

    \[
        \Delta E = E_{n + 1} - E_n = \frac{(2n + 1)h^2}{8ml^2}  
    \]

    \section{单电子体系(类氢原子)波函数求解}



    \[
        \left[ \frac{1}{r^2} \frac{\partial}{\partial r} \left( r ^2 \frac{\partial}{\partial r} \right) + \frac{1}{r^2 \sin \theta} \frac{\partial}{\partial \theta} \sin \theta \frac{\partial}{\partial \theta} + \frac{1}{r^2 \sin ^2 \theta} \frac{\partial^2}{\partial \phi^2} \right]   \psi (r, \theta, \phi) + \frac{2m}{\hbar}\left[ E + \frac{Ze^2}{4\pi \epsilon_0 r }  \right] \psi(r, \theta, \phi) = 0
    \]

    分离变量

    \[
       \psi = R(r) \cdot Y(\theta, \phi)  
    \]

    \[
        \frac{Y}{r^2} \frac{\partial}{\partial r} \left( r^2 \frac{\partial R}{\partial r } \right) + \frac{R}{r^2 \sin \theta} \frac{\partial }{\partial \theta} \left(\sin \theta \frac{\partial Y}{\partial \theta} \right) + \frac{R}{r^2 \sin ^2 \theta} \frac{\partial^2 Y}{\partial \phi^2} + \frac{2m}{\hbar}\left[ E + \frac{Ze^2}{4\pi \epsilon_0 r }  \right] RY = 0
    \]

    两边同时乘以 $\frac{r^2}{R \cdot Y}$

    \[
        \frac{1}{R} \frac{\partial}{\partial r} \left( r^2 \frac{\partial R}{\partial r} \right) + \frac{1}{Y \sin \theta} \frac{\partial }{\partial \theta} \left( \sin \theta \frac{\partial Y}{\partial \theta} \right) + \frac{1}{Y \sin ^2 \theta} \frac{\partial ^2 Y}{\partial \phi ^2} + \frac{2m}{\hbar}\left[ E + \frac{Ze^2}{4\pi \epsilon_0 r }  \right] r^2 = 0 
    \]

    现在整个方程就可以分为一个和角度有关,一个和长度有关。我们可以把与角度、长度有关的量分别提出来。

    \[
        \frac{1}{R} \frac{\partial}{\partial r} \left( r^2 \frac{\partial R}{\partial r} \right) + \frac{2m}{\hbar}\left[ E + \frac{Ze^2}{4\pi \epsilon_0 r }  \right] r^2 = k
    \]

    \[
        \frac{1}{Y \sin \theta} \frac{\partial}{\partial \theta} \left( \sin \theta \frac{\partial Y}{\partial \theta} \right) + \frac{1}{Y \sin ^2 \theta} \frac{\partial ^2 Y}{\partial \phi ^2} = -k 
    \]

    先看下面这个方程,我们需要继续分离两个角度,也就是$\theta$和$\phi$

    \[
        Y = \Theta(\theta) \cdot \Phi(\phi)  
    \]

    \[
        - \frac{1}{\Theta \Phi} \left[ \frac{\Phi}{ \sin \theta} \frac{\partial}{\partial \theta} \left( \sin \theta \frac{\partial \Theta}{\partial \theta} \right) + \frac{\Theta}{ \sin ^2 \theta} \frac{\partial ^2 \Phi}{\partial \phi ^2} \right] = k
    \]

    \[
        - \frac{1}{\Theta} \frac{1}{ \sin \theta} \frac{\partial}{\partial \theta} \left( \sin \theta \frac{\partial \Theta}{\partial \theta} \right) - \frac{1}{ \sin ^2 \theta \Phi} \frac{\partial ^2 \Phi}{\partial \phi ^2} = k
    \]

    为了让两边一个只有$\theta$,一个只有$\phi$,同时乘以$\sin ^2   \theta$

    \[
        - \frac{1}{\Theta}  \sin \theta \frac{\partial}{\partial \theta} \left( \sin \theta \frac{\partial \Theta}{\partial \theta} \right) - \frac{1}{\Phi} \frac{\partial ^2 \Phi}{\partial \phi ^2} = k \sin ^2 \theta
    \]

    移项

    \[
        -\frac{1}{\Phi} \frac{\partial ^2 \Phi}{\partial \phi ^2} = k \sin ^2 \theta  + \frac{\sin \theta}{\Theta}   \frac{\partial}{\partial \theta} \left( \sin \theta \frac{\partial \Theta}{\partial \theta} \right)
    \]

    两边必须同时等于一个常数 

    \[
         - \frac{1}{\Phi} \frac{\partial ^2 \Phi}{\partial \phi ^2} = m^2
    \]

    这个方程也就是 $\Phi$ 方程

    \[
        \frac{\partial ^2 \Phi}{\partial \phi ^2} + m^2 \Phi  = 0
    \]

    显然,设$\Phi = A e^{a \phi}$ 

    \[
        a^2 e^{a \phi} + m^2 e^{a \phi} = 0
    \]

    \[
        a = \pm im  
    \]

    \[
        \Phi = \begin{cases}
            Ae^{im\phi} & = A( \cos im\phi + i \sin im \phi)  \\
            Ae^{-im\phi} & =  A( \cos im\phi - i \sin im \phi) \\
        \end{cases}  
    \]

    归一化条件:


    \[
        \int_0^{2\pi} A^2 d\phi = 1 
    \] 

    \[
        A = \frac{1}{\sqrt{2\pi}}  
    \]
    
    \[
        \Phi = \frac{1}{\sqrt{2\pi}} e ^{im\phi}  
    \]

    \[
        e^{im\phi} = e^{im\phi} \cdot e^{im2\pi}  
    \]
    
    \[
        e^{im 2\pi} = 1  \qquad \mbox{m只能取整数} m \in Z
    \]

    
\end{document}