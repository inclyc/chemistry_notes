\chapter{多原子电子结构}

\begin{equation*}
    -\frac{\hbar^2}{2m} \nabla_1^2 - \frac{\hbar^2}{2m} \nabla_2^2 - \frac{ze^2}{4\pi\epsilon_0r_1} - \frac{ze^2}{4\pi\epsilon_0r_2} - \frac{ze^2}{4\pi\epsilon_0r_1r_2} \psi = E\psi
\end{equation*}

\section{中心力场法}

如果忽略电子之间的相互特征,将多电子体系转化为单电子体系,并引入一个以原子核为中心的中心力场。这个中心力场可以表现为\textbf{有效核电荷数} $Z^*$。

\subsection{Slater法则}

将电子分层,|1s| , |2s, 2p| , |3s, 3p|, |3d|。屏蔽常数:$\sigma$,同层为0.35, 1s的为0.3,$n - 1$层的为0.85,$n - 2$以上为1。

有效核电荷数

\begin{equation*}
    Z^* = Z - \sum \sigma
\end{equation*}

例如 $\ce{Na}$ $3s^1$,屏蔽常数为

\begin{equation*}
    0.85 \times 8 + 1 \times 2 = 8.8
\end{equation*}

这样$3s^1$轨道的电子能量为

\begin{equation*}
    E = -13.6 \times \frac{11 - 8.8}{9}
\end{equation*}


\section{电子排布的规律}

必须遵循规则:

\begin{enumerate}
    \item Pauli不相容原理
    \item 能量最低原理
    \item Hund规则
\end{enumerate}

\subsection{能级交错}

多电子体系下会出现能级交错的现象。

\begin{equation*}
    ns < np < nd < nf
\end{equation*}

\subsection{钻穿效应和屏蔽效应}

\subsubsection{钻穿效应}

高轨道的s电子在核附近具有较大的密度,被称为\textbf{钻穿效应}。

\subsubsection{屏蔽效应}

低轨道电子对外层轨道电子有较大的屏蔽作用。


\section{分子轨道理论}


\subsection{$\ce{H2+}$氢分子离子}


\begin{equation*}
    \left(-\frac{\hbar^2}{2m} \nabla_1^2 - \frac{\hbar^2}{2m} \nabla_2^2 - \frac{ze^2}{4\pi\epsilon_0r_1} - \frac{ze^2}{4\pi\epsilon_0r_2} \right) \psi = E\psi
\end{equation*}

\subsection{变分法}


选取变分函数$\psi'$ 得到近似解。


最小能量法:

\begin{equation*}
    E = \frac{\int \psi^* \hat{H} \psi \d \tau}{\int \psi^*  \psi \d \tau} 
\end{equation*}


\subsection{线性变分法}

\begin{equation*}
    \psi ' = c_1 \psi_1 + c_2 \psi_2 + \dots + c_n \psi_n
\end{equation*}


求最小值需要列方程:


\begin{equation*}
    \frac{\partial E}{\partial c_1} = \frac{\partial E}{\partial c_2} = \cdots = \frac{\partial E}{\partial c_n} = 0
\end{equation*}



\subsubsection{线性变分法处理$\ce{H2+}$氢分子离子}

选取变分函数:


如果$r_a \ll r_b$, 则选取 $\psi_A$ 

如果$r_a \gg r_b$,则选取$\psi_B$

\begin{equation*}
    \psi = c_A \psi_A + c_B \psi_B
\end{equation*}

\begin{align*}
    \bar{E} &= \frac{\int \psi^* \hat{H} \psi \d \tau}{\int \psi^*  \psi \d \tau}  \\
    &= \frac{\int (c_A\psi_A + c_B\psi_B) \hat{H} ( c_A\psi_A + c_B\psi_B) \d \tau}{\int (c_A\psi_A + c_B\psi_B)^2 \d \tau} \\ 
    &= \frac{\int (c_A\psi_A + c_B\psi_B) \hat{H} ( c_A\psi_A + c_B\psi_B) \d \tau}{\int (c_A\psi_A + c_B\psi_B)^2 \d \tau} \\ 
\end{align*}


\begin{align*}
    H_{AA} &= \int \psi_A \hat{H} \psi_A \d \tau = \int \psi_B + \hat{H} \psi_B \d \tau \\ 
    H_{AB} &= \int \psi_A \hat{H} \psi_B \d \tau = \int \psi_B + \hat{H} \psi_A \d \tau \\ 
    S_{AB} &= \int \psi_A \psi_B \d \tau
\end{align*}



\begin{align*}
    \left( \bar{E} - H_{AA} \right) c_A^2 + \left( 2S_{AB} \bar{E} - 2H_{AB}   \right)c_Ac_B + \left(  \bar{E} - H_{BB} \right)c_B^2 = 0
\end{align*}


能量最低:

分别对$c_A$,$c_B$求导


\begin{align*}
    2 \left(  \bar{E} - H_{AA}  \right)c_A + 2 \left( S_{AB} \bar{E} - H_{AB}  \right)c_B &= 0 \\
    2 \left(  S_{AB}\bar{E} - H_{AB} \right)c_A + 2 \left( \bar{E} - H_{BB}  \right)c_B &= 0
\end{align*}


\begin{align*}
    \left| \begin{array}{cc}
        \bar{E} - H_{AA} & S_{AB} \bar{E} - H_{AB} \\ 
        S_{AB} \bar{E} - H_{AB} & \bar{E} - H_{BB} 
    \end{array} \right|
\end{align*}



