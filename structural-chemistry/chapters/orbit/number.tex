
\section{量子数的物理意义}

\subsection{主量子数$n$}

\begin{align*}
    E_n &= - \frac{me^4}{8 \pi \epsilon_0^2 h^2} \frac{Z^2}{n^2} \\ 
    &= - 13.6 \frac{Z^2}{n^2} (\mathrm{eV})\\
\end{align*}

$n$的取值决定能量的高低。能量最低时,依然不为0,具有零点能效应。这里的``零点能''指的是$n = 1$的取值,由于一些历史原因,被称为零点能。

\subsection{角量子数$l$}

\begin{equation*}
    \frac{2m r^2}{\hbar ^2} = l \cdot (l + 1)
\end{equation*}

\begin{equation*}
    \frac{8 \pi^2 m r ^2}{h^2} T_{\mbox{转}} = l \cdot (l + 1)
\end{equation*}

\begin{equation*}
    T_{\mbox{转}} = \frac{l \cdot (l + 1)}{8 \pi^2 I}
\end{equation*}

\begin{equation*}
    \mathbf{l} = \sqrt{l(l + 1)}\hbar
\end{equation*}


$l$决定原子轨道的形状、电子云的形状。

\subsection{磁量子数$m$}

\begin{align*}
    \hat{M}_z &= -i\hbar \frac{\partial}{\partial \phi} \psi(r, \theta, \phi) \\ 
    &= - R(r) \Theta(\theta) i \hbar \frac{\partial}{\partial \phi} \left( \frac{1}{\sqrt{2\pi}} e^{im\phi} \right) \\
    &= R(r) \Theta(\theta) \hbar m \left( \frac{1}{\sqrt{2\pi}} e^{im\phi} \right)
\end{align*}

\begin{equation*}
    \mathbf{l}_z = m\hbar = m\frac{h}{2\pi}
\end{equation*}

在磁场中,要考察电子运动产生的磁矩,需要应用磁量子数。

$m$可以决定$M_z$、$\mu_{z}$,原子轨道和电子云的取向。



\begin{align*}
    \hat{M}_z \cdot \psi (r, \theta, \phi) &= -i \hbar R(r) \Theta(\theta) \frac{\partial}{\partial \phi} \sqrt{\frac{1}{\pi}} \cos \phi  \\ 
    &= \cdots \times \sin \phi \\ 
    &\neq \cdots \times \phi(r, \theta, \phi)
\end{align*}


注意 
\begin{enumerate}
    \item $s$轨道没有角动量量子化的问题
\end{enumerate}

\subsubsection{磁矩}

\begin{equation*}
    \hat{\mu} = -\frac{e}{2m_e} \hat{M}
\end{equation*}

电子绕核运动的磁矩

\begin{equation*}
    |\mu| = \left\lvert -\frac{e}{2m_e} \hat{M} \right\rvert
\end{equation*}

玻尔磁子 $\mu_B = \frac{e\hbar}{2m_e}$

\begin{equation*}
    |\mu| = \sqrt{l(l + 1)} \mu_B
\end{equation*}


\subsection{自旋量子数$m_s$}

电子自旋 $s = \frac{1}{2}$ 


\begin{align*}
    M_s &= \sqrt{\frac{1}{2} \cdot \left( 1 + \frac{1}{2} \right)} \cdot \hbar \\ 
    &= \frac{\sqrt{3}}{2} \hbar
\end{align*}


\begin{align*}
    M_{s_z} = m_s \hbar (\pm \frac{1}{2})
\end{align*}


\begin{align*}
    \mu_{s_z} &= \pm \mu_B \\
    &= \pm \frac{e\hbar}{2m_e} \\ 
    &= -\frac{e}{m_e} \hbar (\mp \frac{1}{2}) \\ 
    &= -\frac{e}{m_e} M_{s_z} 
\end{align*}

\begin{align*}
    |u_s| &= -\frac{e}{m_e} \cdot M_s \\ 
    &= \sqrt{3} \mu_B
\end{align*}

$\psi_{n,l,m,m_s}$ 轨道波函数



\subsubsection{赛曼效应}

$\psi(1, 0, 0)$轨道,$\psi(2, 1, \pm 1, 0)$轨道算能量差,由于m有三种取值,导致角动量的方向是量子化的,方向的量子化导致绕核运动产生的磁矩是量子化的。三种量子化产生的磁矩使得能量有升高有降低。本来固定能量差的谱线分为了三条。
