\section{量子力学的基本假设}

\subsection{波函数与几率}

\textit{假设一:微观粒子的运动状态可以用波函数来表示。}

含时波函数$\psi(x, y, z, t)$,定态波函数$\psi(x, y, z)$。用复值平面波来代表电子的运动状态。

\[
    p = |\psi|^2 \times V   
\]

波函数没有物理意义,$|\psi|^2$有物理意义,代表几率密度。在空间内电子的运动规律是其几率密度。

\subsubsection{波函数的重要性质}

$c\psi$与$\psi$描述同一状态。波函数前乘以一个系数,描述的状态不发生改变。

\subsection{力学量与算符}

算符就是表示运算过程中的符号。$\sqrt{}$ , $\frac{\mathrm{d}}{\mathrm{d}x}$,$\frac{\partial}{\partial x}$, $\frac{\partial^2}{\partial x^2}$

\[\hat{F}\psi = F \times \psi \]

\subsubsection{动量算符}

\[\hat{p} = \frac{h}{2 \pi i} \frac{\partial}{\partial x}\]

\textbf{举例:}

设

\[\psi = e^{ax}, \hat{F}\psi = \frac{\d \psi}{\d x} \]

\[
    \psi(x) = Ae^{\left(\frac{-2\pi i}{\hbar}\right)Et} e^{\frac{i}{\hbar}5x}
\]

\[
    \hat{p} \psi = 5 \times \psi   
\]

\[
    p_x = 5  
\]

\subsubsection{动能算符}

量子力学中,动能和动量的关系依然不变:

\[
    T = \frac{p^2}{2m}  
\]

\begin{align*}
    \Rightarrow \hat{T}_x \psi &= \frac{p^2}{2m} \psi  \\
    &= \frac{1}{2m}(p \cdot p \cdot psi) \\ 
    &= \frac{1}{2m} \left[\hat{p} \left( p \psi \right) \right] \left(
        \psi ' = p \cdot \psi
     \right) \\ 
    &= \frac{1}{2m} \left[ \hat{p} \cdot \left(
        \hat{p} \psi
     \right) \right] \\
    &= \frac{1}{2m} \frac{\d}{\d x} \left( \frac{\d \psi}{\d x} \right) \left( -i \hbar \cdot i \hbar \right) \\
    &= - \frac{\hbar^2}{2m} \frac{\d ^2}{\d x^2}
\end{align*}

\subsubsection{坐标算符和势能算符}

\[
    \hat{x} = x \cdot  
\]

\[
    \hat{x} \psi = x \cdot \psi  
\]

\[
    \hat{V} = V \cdot   
\]

\[
    \hat{V} \psi = V \cdot \psi  
\]


\subsubsection{哈密顿算符和总能量}

\[
    E_{\mbox{总}} = T + V  
\]

\subsection{本征值和本征方程}

如果按照算符作用到波函数,所求的动量在常数方向上具有本征值。如果算出来的东西是一个不确定量,那么这个算符就没有本征值。

\subsection{薛定谔方程(假设三)}

\[
    \hat{H} \psi = E \psi  
\]

\subsubsection{波函数的合格条件}

\begin{enumerate}
    \item 单值,保证空间几率密度的唯一化
    \item 二阶可导,否则方程无解
    \item 平方可积(归一化)
\end{enumerate}

\[
    \int \psi^{*} \cdot \psi d\tau = 1  
\]

或

\[
    \int \psi^{*} \cdot \psi d\tau = \mathrm{C} 
\]

归一化条件

\[
    \psi' = \frac{1}{\sqrt{k}} \cdot \psi  
\]

\subsection{平均值(假设四)}

\[
    \hat{x} \cdot \psi = x \cdot \psi  
\]

这种函数求不出本征值,可以考虑求平均值

\[
    \bar{F} = \frac{\int \psi^* \hat{F} \psi \d \tau}{\int \psi^* \psi \d \tau}
\]

\subsubsection{态叠加原理}

\[
    \psi = \sum_{i=1}^n c_i \psi_i
\]

系数对概率的贡献是平方关系。


\subsubsection{正交性}

两个状态间不影响,其积分为0。

\[
   \int c_ic_j \psi_i \psi_j d\tau = 0
\]

\[
   A = \frac{\int (c_1 \psi_x + c_2 \psi_y + c_3 \psi_z) \hat{F} (c_1 \psi_x^* + c_2 \psi_y^* + c_3 \psi_z^*)}{k} 
\]

如果$\hat{F}$有本征值,其本征值为$F$,则

\[
    A = \frac{\sum\limits_{i=1}^n |c_i|^2 F }{ \sum\limits_{i=1}^n |c_i|^2 } 
\] 


\subsubsection{泡利不相容原理}

描述多电子体系的完全波函数,交换其中任意两个电子的完全坐标,波函数是反对称的。

电子必须旋转720度,才能恢复到原来的状态。电子的自旋量子数为$\frac{1}{2}$
