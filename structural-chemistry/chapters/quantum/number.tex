
\section{量子数的物理意义}

\subsection{主量子数$n$}

\begin{align*}
    E_n &= - \frac{me^4}{8 \pi \epsilon_0^2 h^2} \frac{Z^2}{n^2} \\ 
    &= - 13.6 \frac{Z^2}{n^2} (\mathrm{eV})\\
\end{align*}

$n$的取值决定能量的高低。能量最低时,依然不为0,具有零点能效应。这里的``零点能''指的是$n = 1$的取值,由于一些历史原因,被称为零点能。

\subsection{角量子数$l$}

\begin{equation*}
    \frac{2m r^2}{\hbar ^2} = l \cdot (l + 1)
\end{equation*}

\begin{equation*}
    \frac{8 \pi^2 m r ^2}{h^2} T_{\mbox{转}} = l \cdot (l + 1)
\end{equation*}

\begin{equation*}
    T_{\mbox{转}} = \frac{l \cdot (l + 1)}{8 \pi^2 I}
\end{equation*}

\begin{equation*}
    \mathbf{l} = \sqrt{l(l + 1)\hbar}
\end{equation*}


\subsection{磁量子数$m$}

\begin{align*}
    \hat{M}_z &= -i\hbar \frac{\partial}{\partial \phi} \psi(r, \theta, \phi) \\ 
    &= - R(r) \Theta(\theta) i \hbar \frac{\partial}{\partial \phi} \left( \frac{1}{\sqrt{2\pi}} e^{im\phi} \right) \\
    &= R(r) \Theta(\theta) \hbar m \left( \frac{1}{\sqrt{2\pi}} e^{im\phi} \right)
\end{align*}

\begin{equation*}
    \mathbf{l}_z = m\hbar = m\frac{h}{2\pi}
\end{equation*}

在磁场中,要考察电子运动产生的磁矩,需要应用磁量子数。

