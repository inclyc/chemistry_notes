\section{原子状态和光谱能级}

\subsection{电子组态}

\begin{center}
	$\ce{O}: 1s^22s^22p^4$
\end{center}

\subsubsection{概念}

\paragraph{激发态}:基态(最低能量)跃迁到较高能级的轨道

\paragraph{原子的量子态}:考虑原子当中电子的相互作用,反映电子能量状态的一个量子数。

\paragraph{原子的光谱项} 可以表示出电子之间的关系、以及电子的能级。不同的电子之间、轨道和轨道之间,具有相互作用的电子相互作用。

\subsubsection{需要考虑的相互作用}

\paragraph{同一电子} 自旋与轨道。

\paragraph{不同电子之间} 轨道和轨道之间的相互作用,自旋和自旋的相互作用,自旋与轨道的作用。


\subsection{L-S耦合(考虑三种作用)}

按照表\ref{tab:LS}设计角动量。然后根据公式

\begin{equation*}
	\vec{a} + \vec{b} \ce{->[\mbox{耦合}]} \mbox{原子总角动量}
\end{equation*}

\begin{equation*}
	\vec{a} = \sum \vec{l}_i
\end{equation*}

\begin{equation*}
	\vec{b} = \sum \vec{s}_i
\end{equation*}

\begin{table}[h]
	\centering
	\begin{tabular}{ccc}
		\toprule
		向量 & 意义                      & 符号  \\
		\midrule
		a    & 轨道角动量$\vec{l}$向量和 & $M_L$ \\
		b    & 自旋角动量$\vec{s}$向量和 & $M_S$ \\
		\bottomrule
	\end{tabular}
	\caption{角动量设计}
	\label{tab:LS}
\end{table}

\subsubsection{$M_L$}

\begin{equation*}
	\left\lvert M_L \right\rvert = \sqrt{L(L + 1)} \hbar
\end{equation*}

\subsubsection{$M_{L_z}$}

\begin{equation*}
	M_{L_z} = m_L \cdots
\end{equation*}

\begin{equation*}
	m_L = -L, \cdots, +L
\end{equation*}

\subsubsection{$M_S$}

原子的自旋总量子数 $S = |S_1 + S_2|, |S_1 - S_2|$

例:两个电子 s = $1, 0$
三个电子:$\frac{3}{2},\frac{1}{2}$

\begin{equation*}
	M_{S_z} = m_{S} \mu_B
\end{equation*}

$m_S$的取值:$-S, -S + 1, \cdots , S$

\subsubsection{$M_J$}

\begin{equation*}
	\left\lvert M_J \right\rvert = \sqrt{J(J + 1)} \hbar
\end{equation*}

\begin{equation*}
	J = L + S, (L + S - 1), \cdots |L - S|
\end{equation*}

\begin{equation*}
	M_{J_z} = m_J \cdot \mu_B
\end{equation*}

\subsection{原子光谱项}

光谱项的表示方法:
\begin{equation*}
	{^{2S + 1}} \framebox{S P D F}
\end{equation*}

其中,$J$值也会影响能量的大小,但影响程度较小。$J$值的不同使得原子光谱项在外磁场下的分裂不同。这使得每个光谱项可以分裂为个光谱支项,个数与$J$的取值个数相同。每一个$J$值分为$2J + 1$个精细能级。
\begin{equation*}
	^{2S + 1} \framebox{S P D F}_{J}
\end{equation*}

光谱项的意义在表\ref{tab:spectrum}中列出。

\begin{table}[h]
	\centering
	\begin{tabular}{cc}
		\toprule
		项  & 含义                                            \\
		\midrule
		$L$ & 单电子体系轨道亚层的能级                        \\
		$S$ & $\begin{cases}
				       L > S & \mbox{光谱支项为} $(2S + 1)$ \mbox{个} \\
				       L < S & \mbox{光谱支项为} $(2L + 1)$ \mbox{个} \\
			       \end{cases}$ \\
		$J$ & 赛曼分裂为$2J + 1$个                            \\
		\bottomrule
	\end{tabular}
	\caption{原子光谱项}
	\label{tab:spectrum}
\end{table}



\subsubsection{例题}


\paragraph{组态$\ce{B}: 1s^22s^22p^1$}

$2p \rightarrow {^2P_{\frac{3}{2}}} {^2P_{\frac{1}{2}}}$

\paragraph{组态$1s^22p^1$}

$L = 1, S = 1, 0$,所以有光谱项:${^3P}, {^1P}$。代入$J$的计算公式可以得出光谱支项:${^3P_2}, {^3P_1}, {^3P_0}, {^1P_1}$

\paragraph{组态$2p^13p^1$}

$L = 0, 1, 2$,$S = 0, 1$,光谱项:${^1S}, {^1P}, {^1D}$,${^3S}, {^3P}, {^3D}$。考虑$J$值则可列出表\ref{tab:spectrum2p13p1}。

\begin{table}[h]
	\centering
	\begin{tabular}{cccc}
		\toprule
		L & S & J                           & 赛曼分裂数 \\
		\midrule
		2 & 1 & ${^3D_3}, {^3D_2}, {^3D_1}$ & 15         \\
		2 & 0 & ${^1D_2}$                   & 5          \\
		1 & 1 & ${^3P_2}, {^3P_1}, {^3P_0}$ & 9          \\
		1 & 0 & ${^1P_1}$                   & 3          \\
		0 & 1 & ${^3S_1}$                   & 3          \\
		0 & 0 & ${^1S_0}$                   & 1          \\
		\bottomrule
	\end{tabular}
	\caption{$2p^13p^1$的光谱项}
	\label{tab:spectrum2p13p1}
\end{table}

\paragraph{$2p^2$等价电子}

在考虑泡利不相容原理下,设电子数为$k$,总共的状态数为:
\begin{equation*}
	C_{2(2l + 1)}^k
\end{equation*}
例如,$2p^2$电子组态,$l=1, k=2$,电子状态数为$C_6^2 = 15$

\subsubsection{能级}

$S$大的,能级最低。如果$S$相同,$L$大的能级低。

\subsubsection{正光谱与反光谱}

半满为分界线。半满前为正光谱,半满后为负光谱。

