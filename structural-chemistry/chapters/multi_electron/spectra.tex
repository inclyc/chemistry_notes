\section{原子状态和光谱能级}

\subsection{电子组态}

\begin{center}
	$\ce{O}: 1s^22s^22p^4$
\end{center}

\subsubsection{概念}

\paragraph{激发态}:基态(最低能量)跃迁到较高能级的轨道

\paragraph{原子的量子态}:考虑原子当中电子的相互作用,反映电子能量状态的一个量子数。

\paragraph{原子的光谱项} 可以表示出电子之间的关系、以及电子的能级。不同的电子之间、轨道和轨道之间,具有相互作用的电子相互作用。

\subsubsection{需要考虑的相互作用}

\paragraph{同一电子} 自旋与轨道。

\paragraph{不同电子之间} 轨道和轨道之间的相互作用,自旋和自旋的相互作用,自旋与轨道的作用。


\subsection{L-S耦合(考虑三种作用)}

按照表\ref{tab:LS}设计角动量。然后根据公式

\begin{equation*}
	\vec{a} + \vec{b} \ce{->[\mbox{耦合}]} \mbox{原子总角动量}
\end{equation*}

\begin{equation*}
	\vec{a} = \sum \vec{l}_i
\end{equation*}

\begin{equation*}
	\vec{b} = \sum \vec{s}_i
\end{equation*}

\begin{table}[h]
	\centering
	\begin{tabular}{ccc}
		\toprule
		向量 & 意义                      & 符号  \\
		\midrule
		a    & 轨道角动量$\vec{l}$向量和 & $M_L$ \\
		b    & 自旋角动量$\vec{s}$向量和 & $M_S$ \\
		\bottomrule
	\end{tabular}
	\caption{角动量设计}
	\label{tab:LS}
\end{table}

\subsubsection{$M_L$}

\begin{equation*}
	\left\lvert M_L \right\rvert = \sqrt{L(L + 1)} \hbar
\end{equation*}

\subsubsection{$M_{L_z}$}

\begin{equation*}
	M_{L_z} = m_L \cdots
\end{equation*}

\begin{equation*}
	m_L = -L, \cdots, +L
\end{equation*}

\subsubsection{$M_S$}

原子的自旋总量子数 $S = |S_1 + S_2|, |S_1 - S_2|$

例:两个电子 s = $1, 0$
三个电子:$\frac{3}{2},\frac{1}{2}$

\begin{equation*}
	M_{S_z} = m_{S} \mu_B
\end{equation*}

$m_S$的取值:$-S, -S + 1, \cdots , S$

\subsubsection{$M_J$}

\begin{equation*}
	\left\lvert M_J \right\rvert = \sqrt{J(J + 1)} \hbar
\end{equation*}

\begin{equation*}
	J = L + S, (L + S - 1), \cdots |L - S|
\end{equation*}

\begin{equation*}
	M_{J_z} = m_J \cdot \mu_B
\end{equation*}

\subsection{原子光谱项}

\begin{equation*}
	^{2S + 1} \framebox{S P D F}_{J}
\end{equation*}