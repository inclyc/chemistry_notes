\section{分子轨道理论}


\subsection{$\ce{H2+}$氢分子离子}


\begin{equation*}
    \left(-\frac{\hbar^2}{2m} \nabla_1^2 - \frac{\hbar^2}{2m} \nabla_2^2 - \frac{ze^2}{4\pi\epsilon_0r_1} - \frac{ze^2}{4\pi\epsilon_0r_2} \right) \psi = E\psi
\end{equation*}

\subsection{变分法}


选取变分函数$\psi'$ 得到近似解。


最小能量法:

\begin{equation*}
    E = \frac{\int \psi^* \hat{H} \psi \d \tau}{\int \psi^*  \psi \d \tau}
\end{equation*}


\subsection{线性变分法}

\begin{equation*}
    \psi ' = c_1 \psi_1 + c_2 \psi_2 + \dots + c_n \psi_n
\end{equation*}


求最小值需要列方程:


\begin{equation*}
    \frac{\partial E}{\partial c_1} = \frac{\partial E}{\partial c_2} = \cdots = \frac{\partial E}{\partial c_n} = 0
\end{equation*}



\subsubsection{线性变分法处理$\ce{H2+}$氢分子离子}

选取变分函数:


如果$r_a \ll r_b$, 则选取 $\psi_A$

如果$r_a \gg r_b$,则选取$\psi_B$

\begin{equation*}
    \psi = c_A \psi_A + c_B \psi_B
\end{equation*}

\begin{align*}
    \bar{E} & = \frac{\int \psi^* \hat{H} \psi \d \tau}{\int \psi^*  \psi \d \tau}                                                   \\
            & = \frac{\int (c_A\psi_A + c_B\psi_B) \hat{H} ( c_A\psi_A + c_B\psi_B) \d \tau}{\int (c_A\psi_A + c_B\psi_B)^2 \d \tau} \\
            & = \frac{\int (c_A\psi_A + c_B\psi_B) \hat{H} ( c_A\psi_A + c_B\psi_B) \d \tau}{\int (c_A\psi_A + c_B\psi_B)^2 \d \tau} \\
\end{align*}


\begin{align*}
    H_{AA} & = \int \psi_A \hat{H} \psi_A \d \tau = \int \psi_B + \hat{H} \psi_B \d \tau \\
    H_{AB} & = \int \psi_A \hat{H} \psi_B \d \tau = \int \psi_B + \hat{H} \psi_A \d \tau \\
    S_{AB} & = \int \psi_A \psi_B \d \tau
\end{align*}



\begin{align*}
    \left( \bar{E} - H_{AA} \right) c_A^2 + \left( 2S_{AB} \bar{E} - 2H_{AB}   \right)c_Ac_B + \left(  \bar{E} - H_{BB} \right)c_B^2 = 0
\end{align*}


能量最低:

分别对$c_A$,$c_B$求导


\begin{align*}
    2 \left(  \bar{E} - H_{AA}  \right)c_A + 2 \left( S_{AB} \bar{E} - H_{AB}  \right)c_B & = 0 \\
    2 \left(  S_{AB}\bar{E} - H_{AB} \right)c_A + 2 \left( \bar{E} - H_{BB}  \right)c_B   & = 0
\end{align*}


\begin{align*}
    \left| \begin{array}{cc}
               \bar{E} - H_{AA}        & S_{AB} \bar{E} - H_{AB} \\
               S_{AB} \bar{E} - H_{AB} & \bar{E} - H_{BB}
           \end{array} \right|
\end{align*}

\begin{align*}
    \bar{E_1} & = \frac{H_{AA} + H_{AB}}{1 + S_{AB}} \\
    \bar{E_2} & = \frac{H_{AA} - H_{AB}}{1 - S_{AB}} \\
\end{align*}

注意,从上式可得:

\begin{equation*}
    c_A = \pm c_B
\end{equation*}

\begin{align*}
    \psi = c_A \psi_A + c_B \psi_B = \begin{cases}
                                         \psi_1 = c_A(\psi_A + \psi_B) \\
                                         \psi_2 = c_A(\psi_A - \psi_B)
                                     \end{cases}
\end{align*}



\subsubsection{重叠积分$S_{AB}$}

\begin{equation*}
    S_{AB} = \int \psi_A \psi_B \d \tau
\end{equation*}

这个积分被称为重叠积分。
\begin{equation*}
    \begin{cases}
        \psi_A = \psi_B \quad S_{AB} = 1 & \mbox{原子核核间距} R \to 0      \\
        S_{AB} = 0                       & \mbox{原子核核间距} R \to \infty \\
        0 < S_{AB} < 1                   & \mbox{成键}                      \\
    \end{cases}
\end{equation*}

\subsubsection{库仑积分$H_{AA}$}

\begin{align*}
    H_{AA} & = \int \psi_A \hat{H} \psi_A \d \tau                                                                                                                                                                                           \\
           & = \int \psi_A \left( -\frac{\hbar}{2m} \nabla^2 - \frac{ze^2}{4\pi\epsilon_0r_A} - \frac{ze^2}{4\pi\epsilon_0r_B} -  \frac{e^2}{4\pi\epsilon_0R}  \right) \psi_A \d \tau                                                       \\
           & = \int \psi_A \left( -\frac{\hbar}{2m} \nabla^2 - \frac{ze^2}{4\pi\epsilon_0r_A} \right) \psi_A \d \tau  - \int \psi_A \frac{ze^2}{4\pi\epsilon_0r_B} \psi_A \d \tau + \int \psi_A  \frac{e^2}{4\pi\epsilon_0R} \psi_A \d \tau \\
           & = E_A - \int \psi_A \frac{ze^2}{4\pi\epsilon_0r_B} \psi_A \d \tau + \int \psi_A  \frac{e^2}{4\pi\epsilon_0R} \psi_A \d \tau                                                                                                    \\
           & = E_A - J
\end{align*}

其中$E_A$是原子基态的能量,$J$几乎可以抵消。$E_A = E_B \approx H_{AA}$

\subsubsection{交换积分$H_{AB}$}

\begin{align*}
    H_{AB} & = \int \psi_A \hat{H} \psi_B \d \tau                                                                                                                                     \\
           & = \int \psi_A \left( -\frac{\hbar}{2m} \nabla^2 - \frac{ze^2}{4\pi\epsilon_0r_A} - \frac{ze^2}{4\pi\epsilon_0r_B} -  \frac{e^2}{4\pi\epsilon_0R}  \right) \psi_B \d \tau \\
           & = \int \psi_A E_B \psi_B \d \tau - \int \psi_A \left(\frac{ze^2}{4\pi\epsilon_0r_B} - \frac{e^2}{4\pi\epsilon_0R} \right) \psi_B \d \tau                                 \\
           & = E_BS_{AB} +  \left(\frac{ze^2}{4\pi\epsilon_0r_B} - \frac{e^2}{4\pi\epsilon_0R} \right) \int \psi_A \cdot \psi_B \d \tau                                               \\
           & = E_BS_{AB} + K \left( K < 0 \right)                                                                                                                                     \\
\end{align*}
注意,这里
\begin{equation*}
    H_{AB} - E_BS_{AB} = K
\end{equation*}

$E_BS_{AB}$为负值,同时,$\left(\frac{ze^2}{4\pi\epsilon_0r_B} - \frac{e^2}{4\pi\epsilon_0R} \right)$也为负值,总体来看,$H_{AB}$为负值。这个负值有利于体系总体能量的降低,利于成键。这个积分也被称为键积分。在讨论时为了简化,通常也被称为$\beta$积分。


\begin{align*}
    E_1 & = \frac{H_{AA} + H_{BB}}{1 + S_{AB}}                                               \\
        & = \frac{H_{AA} + H_{BB}S_{AB}}{1 + S_{AB}} + \frac{H_{BB} - S_{AB}E_B}{1 + S_{AB}} \\
        & = E_H + \frac{K}{1 + S_{AB}}                                                       \\
        & = E_H + \mbox{负数}
\end{align*}

$\psi_1 = c_A \left( \psi_A + \psi_B \right) \Rightarrow$ 体系能量比单独的氢原子低,表现为成键轨道。

\begin{align*}
    E_2 & = \frac{H_{AA} - H_{BB}}{1 - S_{AB}}                               \\
        & = \frac{H_{AA} - H_{AA}S_{AB} + H_{AA}S_{AB} - H_{AB}}{1 - S_{AB}} \\
        & = H_{AA} + \frac{H_{AA}S_{AB} - H_{AB}}{1 - S_{AB}}                \\
        & = H_{AA} + \frac{E_AS_{AB} - H_{AB}}{1 - S_{AB}}                   \\
        & = H_{AA} + \frac{-K}{1 - S_{AB}}                                   \\
        & = E_H + \mbox{正数}
\end{align*}

$\psi_2 = c_A \left( \psi_A - \psi_B \right) \Rightarrow$ 体系能量比单独的氢原子高,表现为反键轨道。

