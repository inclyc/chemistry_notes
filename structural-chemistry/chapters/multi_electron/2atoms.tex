\section{分子轨道理论和双原子分子结构}

\subsection{分子轨道理论的概念}


分子中的每个电子都可以看作是在各个原子的原子核以及其他电子所组成的平均力场下运动,第$i$个电子的运动状态可以用波函数$\psi_i$来描述。$\psi_i$被称为分子的单电子波函数,又称为分子轨道。分子的总能量为各个电子所处的分子轨道的分子轨道能量之和。


\subsection{分子轨道的形成}

分子轨道$\psi$可以近似地用能级相近的原子轨道线性组合。原子轨道线性组合成分子轨道时,轨道数目不变,轨道能级改变。两个能量相近的原子轨道组合成分子轨道时,能量低于原子轨道能级的称为成键轨道,高于原来两个原子轨道的能级分为反键轨道。

两个分子轨道的形成必须满足以下条件:

\begin{itemize}
    \item 能量相近
    \item 轨道最大重叠
    \item 对称性匹配
\end{itemize}


\subsubsection{能量相近的证明}

设$\phi_a, \phi_b$分别表示两个原子的原子轨道。其中$E_a < E_b$,他们组合成分子轨道。设$H_{aa} = E_a$,$H_{bb} = E_b$,$H_{ab} = \beta$,$S_{ab} = 0$。带入久期行列式,可得:

\begin{equation*}
    (E_a - E)(E_b - E) = \beta^2
\end{equation*}

\begin{align*}
    E_1 & = \frac{1}{2} \left[ (E_a + E_b) - \sqrt{\left(  E_b - E_a \right)^2 + 4\beta^2}  \right] \\
        & = E_a - U
\end{align*}
\begin{align*}
    E_2 & = \frac{1}{2} \left[ (E_a + E_b) + \sqrt{\left(  E_b - E_a \right)^2 + 4\beta^2}  \right] \\
        & = E_b + U
\end{align*}

其中

\begin{align*}
    U = \frac{1}{2} \left[ \sqrt{ \left(E_b - E_a\right)^2 + 4\beta^2} - \left(E_b - E_a\right) \right]
\end{align*}
