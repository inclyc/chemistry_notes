\section{休克尔分子轨道理论(HMO)}

\subsection{适用范围}

只适用于共轭体系,只针对$\pi$电子。

\subsection{$\pi$电子近似}

将所有的化学键看作是原键+$\sigma$键,再加额外的$\pi$电子。


\subsection{线性变分法处理}


\begin{equation*}
    \psi = c_1 \psi_1 + c_2 \psi_2 + c_3 \psi_3 + \cdots + c_n \psi_n
\end{equation*}

久期方程式

\begin{equation*}
    \begin{bmatrix*}
        H_{11} - ES_{11} & H_{12} - ES_{12} & \cdots & H_{1n} - ES_{1n} \\
        H_{21} - ES_{21} & H_{22} - ES_{22} & \cdots & H_{2n} - ES_{2n} \\
        \vdots & \vdots & \ddots & \vdots \\
        H_{n1} - ES_{n1} & H_{n2} - ES_{n2} & \cdots & H_{nn} - ES_{nn}
    \end{bmatrix*} = 0
\end{equation*}

计算过程中做简化处理(表 \ref{tab:simplify}):


\begin{table}[h]
    \centering
    \begin{tabular}{cc}
        库仑积分 & $H_{11} = H_{22} = \alpha$                 \\
        交换积分 & $H_{ij} = H_{ji} = \begin{cases}
                                              \beta & i = j \neq 1    \\
                                              0     & i \neq j \neq 1
                                          \end{cases}$ \\
        重叠积分 & $S_{ij} = \begin{cases}
                                     1 & i = j    \\
                                     0 & i \neq j
                                 \end{cases}$
    \end{tabular}
    \caption{简化处理方法}
    \label{tab:simplify}
\end{table}


\subsection{分子图}

需要计算电荷密度、键级、自由价。

\subsubsection{电荷密度}

电荷密度$\rho_i$第$i$个原子上出现的$\pi$电子数

\begin{equation*}
    \rho_i = \sum_{k} n_k c^2_{ki}
\end{equation*}

\subsubsection{键级}

键级$P_{ij}$表示原子$i$和原子$j$之间成键的强度。

\begin{equation*}
    P_{ij} = \sum_k n_k c_{ki} c_{kj}
\end{equation*}

\subsubsection{自由价}

自由价$F_i$,第$i$个原子剩余成键能力的相对大小。

\begin{equation*}
    F_i = F_{\mathrm{max}} - \sum_i P_{ij}
\end{equation*}


$F_{\mathrm{max}}$的值为$\sqrt{3}$,采用了理论上存在的三次甲基甲烷分子的解。

若要计算三次甲基甲烷的键级,计算久期行列式:

\begin{equation*}
    \begin{vmatrix}
        x & 1 & 1 & 1 \\
        1 & x & 0 & 0 \\
        1 & 0 & x & 0 \\
        1 & 0 & 0 & x \\
    \end{vmatrix}
\end{equation*}


\subsection{丁二烯的HMO法处理}

丁二烯 \begin{scriptsize}
    \chemfig{=[:30]-[:-30]=[:30]}
\end{scriptsize} 的分子轨道为

\begin{equation*}
    \psi = c_1 \psi_1 + c_2 \psi_2 + c_3 \psi_3 + c_4 \psi_4
\end{equation*}

$\psi_i$表示参加共轭的4个C原子的$p_z$轨道。

久期行列式:

\begin{equation*}
    \begin{vmatrix}
        \alpha - E & \beta      & 0          & 0          \\
        \beta      & \alpha - E & \beta      & 0          \\
        0          & \beta      & \alpha - E & \beta      \\
        0          & 0          & \beta      & \alpha - E \\
    \end{vmatrix}
\end{equation*}

\begin{equation*}
    \begin{vmatrix*}
        x & 1 & 0 & 0 \\
        1 & x & 1 & 0 \\
        0 & 1 & x & 1 \\
        0 & 0 & 1 & x \\
    \end{vmatrix*}
\end{equation*}


\begin{equation*}
    x^4 - 3x^2 + 1 = 0
\end{equation*}

解得

\begin{align*}
    x_1 & = -1.618 \\
    x_2 & = -0.618 \\
    x_3 & = 0.618  \\
    x_4 & = 1.618  \\
\end{align*}

于是

\begin{align*}
    E_1 & = \alpha + 1.618\beta \\
    E_2 & = \alpha + 0.618\beta \\
    E_3 & = \alpha - 0.618\beta \\
    E_4 & = \alpha - 1.618\beta \\
\end{align*}

其中$\alpha = E_{p_z} < 0$,$\beta = c_i \sim c_{i+1} (\mbox{键积分}) < 0$。


\begin{equation*}
    E_1 < E_2 < E_3 < E_4
\end{equation*}
