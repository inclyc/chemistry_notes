% set formatoptions+=mB
% set tw=80
\chapter{表面热力学基础}
\section{表面能和表面张力}
\subsection{净吸力}

对于每个液体中的分子而言,液体中的分子都受到一个向液体中心的力。这个力始终指向液
体的内部。

\subsection{表面能}

由于净吸力的存在,在宏观上,整个表面的能量性质趋近于缩小,表面越大,能量越高。

\begin{equation}
    \gamma = \left(\frac{\partial G}{\partial A}\right)_{T, P, n_B}
\end{equation}

系统在温度、压力和组成不变的条件下,可逆地增加表面积时对系统作的非体积功称为表面
功,\textbf{表面功} 与系统增加的面积成正比,即

\begin{equation}
    \d G = \gamma \cdot \d A
\end{equation}


\subsection{表面张力}

将一金属框浇上肥皂液后,再可逆地用力 $F$ 拉动金属框上可移动的边,使之移动 $\d x$
的距离,肥皂膜的表面积扩大$\d A$,因为肥皂膜有两个表面,所以

\begin{equation}
    \d A = 2 l \d x
\end{equation}


\begin{equation}
    F \d x = \gamma \d A
\end{equation}

表面张力是垂直与界面边缘,沿着液面的切线方向单位长度上的收缩力。表面张力的单位是
$\mathrm{N \cdot m^{-1}}$。


\subsection{表面张力大小的影响因素}


表(界)面张力是一种强度性质,纯液体的表面张力通常是指液体与饱和了本身蒸气的空气
而言。影响表(界)面张力的因素主要有

\begin{enumerate}
    \item 物质的本性 \par
          金属键 > 离子键 > 极性共价键 > 非极性共价键
    \item 第二相 ($\beta$ 相)的影响 \par
          温度、压力增加,$\gamma$ 降低
    \item 第三种物质影响(溶质)\par
          无机盐类,例如 $\ce{NaCl}$ \par
          表面活性物质 \par
          表面活性剂:$\ce{C8} \sim \ce{C_{20}}$
    \item 温度的影响
\end{enumerate}


\subsection{Laplace 方程}


\begin{equation}
    \Delta P = \gamma \left(\frac{1}{R_1} + \frac{1}{R_2}\right)
\end{equation}

公式推导:体积功和表面功相等 $\Delta P \d V = \gamma \d A$。
